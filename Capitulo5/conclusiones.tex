\chapter{Conclusiones}\label{chapter5}

%La problemática planteada sobre las fallas en el despacho de gasolina por parte de las gasolineras de la CDMX, es una problemática real que afecta a una gran cantidad de automovilistas (como lo muestran las encuestas realizadas) los cuales no cuentan con las herramientas necesarias para saber, que gasolineras presentan estás fallas y cuales no. 

%El presente trabajo terminal, como se ha expuesto, permite a los automovilistas de la CDMX conocer que gasolineras presentan estos fallos gracias a la elaboración de la clasificación de las mismas. Con lo cual, estos pueden tomar una decisión informada sobre donde cargan gasolina. Como se ha presentado en los capítulos anteriores, el trabajo terminal es factible en cada uno de los ámbitos, tanto técnico, como operativo, como económico. Además de que este, usa tecnología robusta y de vanguardia para así brindar el mejor servicio a los posibles usuarios. Bajo una arquitectura flexible y robusta como la de microservicios, tanto usando una lenguaje de programación que es estándar en el mercado como Java, el trabajo terminal cuenta con todas las características suficientes para satisfaces los requerimientos funcionales establecidos, y por ende, el objetivo del trabajo terminal.

%Finalmente, con en análisis y diseño elaborados, la implementación del sistema resulta como la única actividad pendiente para que el trabajo terminal pueda estar listo para ser sometido a pruebas, tanto unitarias como de integración.
