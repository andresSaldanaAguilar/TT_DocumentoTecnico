\chapter{Trabajo futuro}\label{chapter6}
Para desarrollar el Trabajo Terminal 2 se procederá con el desarrollo de los submódulos de monitoreo y de usuario, tanto para la parte de software como de hardware anteriormente diseñados y analizados, siguiendo la metodología planteada, para posteriormente iniciar con las pruebas unitarias, de integración, del sistema y de validación. Así mismo, se poblará la base de datos para hacer pruebas sobre los módulos que requieren de previos conjuntos de datos para realizar ciertos cálculos, como lo es el cálculo de promedio bimestral o mensual, sin embargo las pruebas de los periodos de tiempo cortos como el promedio diario de energía sensada o el monitoreo en tiempo real se realizará con las mediciones obtenidas por los paneles fotovoltaicos. 
\newline De igual forma, para una mejor predicción de umbrales que definen la eficiencia de la producción de los paneles fotovoltaicos, se realizarán mediciones constantes para poder estandarizar los niveles de producción de energía en la CDMX y así ser más precisos en cuanto a producción de energía en un lugar en específico, ya que recordemos, no se producirá la misma energía en todos los lugares. 
%Como parte de Trabajo terminal 2 se realizara a partir del incremento 4 de la metodología planteada en el capitulo 3, asimismo se poblara la base de datos al registrar gasolineras,usuarios y mediciones realizadas con el caudalimetro.
%Se harán pruebas del algoritmo de clasificación de gasolineras y del sistema completo funcionando.
%Ademas se buscará implementar un mejor diseño del sensor basado en la norma mexicana NOM-001-SCFI-1993, para asegurar la seguridad, de los usuarios, al momento de utilizar el sistema en el automóvil.
\\
%Asimismo para tener un mejor resultado con respecto a la medición se utilizará un apego a la norma mexicana NOM-005-SCFI-2011 la cual se refiere al uso de instrumentos y sistemas para la medición de gasolina y otros combustibles líquidos,la cual brinda la aprobación del método de medición así como de una correcta verificación del flujo de combustible.