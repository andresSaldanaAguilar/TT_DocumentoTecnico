\chapter{Trabajo futuro}\label{chapter7}

\newline A continuación se listan las áreas de mejora que encontramos con el desarrollo de este trabajo terminal:
\\
\newline Para una mejor definición de umbrales que definen la eficiencia de la producción de los paneles fotovoltaicos, se realizarán mediciones constantes para poder estandarizar los niveles de producción de energía en la CDMX y así ser más precisos en cuanto a producción de energía en un lugar en específico, ya que recordemos, no se producirá la misma energía en todos los lugares. 
\\
\newline El uso de un sistema embebido más actual, como la nueva Raspberry Pi 4 que tiene 4GB de RAM a diferencia de su antecesor que tiene solamente 1GB de RAM, con el propósito de mejorar el rendimiento de los servicios y número de paneles a monitorizar.
\\
\newline Personalizar las notificaciones de la aplicación para indicar márgenes de error y tiempo antes de crear una notificación.
\\
\newline El uso de algoritmos de predicción y detección de comportamiento de los paneles, para conocer cuando están en una etapa de deterioro. Hacer proyecciones de producción para semanas o inclusive meses posteriores con el uso de la información recaudada.
\\
\newline Envío y generación de reportes mensuales o bimestrales de producción para su comparación con el bimestral de consumo de la compañía de luz. 