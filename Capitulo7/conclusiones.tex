\chapter{Conclusiones}\label{chapter5}
El uso de fuentes energías renovables actualmente ha cobrado una importancia tremenda, y parecen ser la mejor solución a muchas problemáticas con las que el hombre de hoy en día se enfrenta, ya sean referentes a la generación de energía o a situaciones ambientales.
\newline El presente trabajo terminal, como se ha expuesto, no solo propone el uso de celdas fotovoltaicas y el aprovechamiento de una fuente de energía inagotable como lo es el Sol, sino que también permite un monitoreo constante de la generación de energía para que el usuario pueda tomar decisiones informadas sobre la utilización de la energía generada por las celdas fotovoltaicas. Como se ha presentado en los capítulos anteriores, el trabajo terminal es factible en todos los ámbitos, desde el técnico y el operativo hasta el económico.
\newline El diseño planteado para la aplicación de usuario y la tecnología usada para que se pueda llevar a cabo el monitoreo, hace de este sistema una opción no solo eficiente y útil, sino que también hace posible brindar un excelente servicio a los usuarios, haciendo su uso práctico y sin necesidad de un amplio tiempo de aprendizaje en cuanto a utilización de la aplicación se refiere, ya que el diseño de la aplicación fue pensado teniendo en cuenta la importancia que representa actualmente un dispositivo móvil para una persona.
\newline Con el análisis y diseño elaborados, se concluye que el trabajo terminal cuenta con todas las características que satisfacen los requerimientos funcionales establecidos, y por ende, el trabajo terminal. Siendo la implementación del sistema la única actividad restante para poder someter este trabajo terminal a pruebas, tanto unitarias como de integración.
%La problemática planteada sobre las fallas en el despacho de gasolina por parte de las gasolineras de la CDMX, es una problemática real que afecta a una gran cantidad de automovilistas (como lo muestran las encuestas realizadas) los cuales no cuentan con las herramientas necesarias para saber, que gasolineras presentan estás fallas y cuales no. 

%El presente trabajo terminal, como se ha expuesto, permite a los automovilistas de la CDMX conocer que gasolineras presentan estos fallos gracias a la elaboración de la clasificación de las mismas. Con lo cual, estos pueden tomar una decisión informada sobre donde cargan gasolina. Como se ha presentado en los capítulos anteriores, el trabajo terminal es factible en cada uno de los ámbitos, tanto técnico, como operativo, como económico. Además de que este, usa tecnología robusta y de vanguardia para así brindar el mejor servicio a los posibles usuarios. Bajo una arquitectura flexible y robusta como la de microservicios, tanto usando una lenguaje de programación que es estándar en el mercado como Java, el trabajo terminal cuenta con todas las características suficientes para satisfaces los requerimientos funcionales establecidos, y por ende, el objetivo del trabajo terminal.

%Finalmente, con en análisis y diseño elaborados, la implementación del sistema resulta como la única actividad pendiente para que el trabajo terminal pueda estar listo para ser sometido a pruebas, tanto unitarias como de integración.
