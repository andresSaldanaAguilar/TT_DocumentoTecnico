\chapter{Conclusiones}\label{chapter6}

El uso de fuentes de energías renovables actualmente ha cobrado una importancia tremenda, y parecen ser la mejor solución a muchas problemáticas con las que el hombre de hoy en día se enfrenta, ya sean referentes a la generación de energía o a situaciones ambientales.

\newline El presente trabajo terminal, como se ha expuesto, no solo impulsa el uso de celdas fotovoltaicas y el aprovechamiento de una fuente de energía inagotable como lo es el Sol, sino que también permite una motorización constante de la generación de energía para que el usuario pueda tomar decisiones informadas sobre la producción y comportamiento de energía generada por las celdas fotovoltaicas. Como se ha presentado en los capítulos anteriores, el trabajo terminal es factible en todos los ámbitos, desde el técnico y el operativo hasta el económico.

\newline Con el análisis y diseño elaborados, el trabajo terminal cuenta con todas las características que satisfacen los requerimientos funcionales y no funcionales establecidos que fueron pilares para su realización. Esto hace del sistema una opción no solo eficiente y útil, sino que también hace posible brindar un excelente servicio a los usuarios, como ya hemos concluido, este sistema es capaz de monitorizar hasta 10 celdas solares por servidor, atendiendo fácilmente a las necesidades de un hogar o pequeña empresa, haciendo su uso práctico y sin necesidad de un amplio tiempo de aprendizaje en cuanto a utilización de la aplicación se refiere, ya que el diseño de la aplicación fue pensado teniendo en cuenta la importancia que representa actualmente un dispositivo móvil para una persona.

\newline Finalmente corroboramos cada uno de sus módulos por medio de pruebas unitarias validando los casos de uso de cada una de las funcionalidades y de integración para validar los requerimientos no funcionales, todo esto para poder conocer mejor cual es el alcance, limitaciones y oportunidades a futuro de este trabajo terminal, dándonos un vistazo más realista del desarrollo de sistemas computacionales.
