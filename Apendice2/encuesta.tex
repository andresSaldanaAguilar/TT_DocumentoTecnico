\section{Encuesta} \label{FactibilidadOperativa}
\textbf{Gasolimetro}
\\Se lanzará al mercado una aplicación móvil que permite conocer las gasolineras que cargan con mayor exactitud el combustible solicitado por el usuario. El sistema consta de un sensor que permite medir el flujo de gasolina introducido en el vehículo.
La siguiente encuesta nos ayudará a medir los niveles de aceptación del producto. La información proporcionada solo se utilizara para fines estadísticos.
\\
\begin{enumerate}
	\item Edad: \\ a.-16 a 20 años\hspace{1cm}b.-21 a 30 años\hspace{1cm}c.-31 a 40 años\hspace{1cm}d.-mas de 40 años
	\item Delegación: \rule{20mm}{0.1mm}
	\item Ocupación:\\ a.-Estudiante\hspace{1cm}b.-Profesionista\hspace{1cm}c.-Empleado\hspace{1cm}d.-Independiente
	\item ¿Con qué tipo de automóvil cuenta?\\ a.-De combustión\hspace{1cm}b.-Híbrido\hspace{1cm}c.-Eléctrico\hspace{1cm}d.-No tengo auto
	\item ¿Qué tan seguido carga gasolina?:\\ a.-más de 3 veces por semana\hspace{1cm}b.-2 o 3 veces por semana\hspace{1cm}c.-1 vez a la semana\hspace{1cm}d.-No cargo gasolina
	\item En promedio ¿Cuánto carga de gasolina?(En pesos)\\más de \$500 \hspace{1cm}b.-Entre \$300 y \$500\hspace{1cm}c.-Entre \$100 y \$300 \hspace{1cm}d.-menos de \$100
	\item Del 1 al 5, donde 1 es poco y 5 es mucho ¿Con qué precisión considera que le cargan litros completos? \rule{20mm}{0.1mm}
	\item ¿Actualmente utiliza algún medio para medir los litros que le cargan?\\ a.-Si\hspace{1cm}b.-No
	\item Pensando en la descripción del producto y en los beneficios que este ofrece ¿Qué tan interesante le parece el producto?
	\\ a.-Muy interesante\hspace{1cm}b.-Interesante\hspace{1cm}c.-Poco interesante\hspace{1cm}d.-No me interesa
	\item ¿Cuánto considera que es un precio adecuado para su venta?
	\\ a.-menos de \$300 \hspace{1cm}b.-Entre \$300 y \$500 \hspace{1cm}c.-Entre \$500 y \$800 \hspace{1cm}d.-mas de \$800
	\item ¿En qué sistema operativo le gustaría que este producto estuviera a la venta?\\ a.-Android\hspace{1cm}b.-IOS\hspace{1cm}c.-Windows Phone
	\item ¿Qué tan útil le resulta este producto?\\ a.-Muy útil\hspace{1cm}b.-Útil\hspace{1cm}c.-Poco útil\hspace{1cm}d.-Nada útil
	\item En una escala del 1 al 5, donde 1 es poco y 5 es mucho ¿Qué probabilidad hay de que consuma este producto?\rule{20mm}{0.1mm}
\end{enumerate}