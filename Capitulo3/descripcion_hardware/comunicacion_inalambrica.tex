\subsection{Comunicación inalámbrica}\label{sec:bluetooth}
Bluetooth: \\
Bluetooth es una tecnología de conectividad inalámbrica de baja potencia que se utiliza para transmitir audio, transferir datos y transmitir información entre dispositivos. Hay dos sabores de la tecnología Bluetooth, velocidad básica / velocidad de datos mejorada (BR / EDR) y baja energía (LE)\cite{Bluetooth}. 
\newline
WiFi:\\
WiFi es la abreviatura o el nombre comercial de Wireless Fidelity  y, como su nombre lo indica, es un sistema de conexión de ordenadores completamente inalámbrico, que permite a sus usuarios compartir y transferir información utilizando ondas de radio, es decir, sin utilizar cableado alguno.
\\
De esta manera, podemos mantener comunicaciones entre ordenadores, portátiles, móviles y otros dispositivos que cuenten con tecnología de recepción inalámbrica, facilitando enormemente las comunicaciones, incluso en lugares abiertos lejos de nuestras casas y oficinas.
Las redes WiFi por lo general son de libre acceso, a menos que estén protegidas mediante contraseñas, lo cual, indicaría que son unas redes privadas utilizadas para conexiones con redes locales (LAN)\cite{WiFi}.
\\
En la tabla \ref{Comparacion_Conexiones} se muestran las características que tiene cada uno de los modos de conexión.
\begin{table}[H]
	\centering
	\begin{tabular}{|M{4cm}|M{4cm}|M{4cm}|}
		\hline
		\textbf{Característica} & \textbf{Bluetooth} & \textbf{WiFi} \\ \hline
		Frecuencia & 2.4GHz & 2.4,3.6,5 GHz \\ \hline
		Costo & Bajo & Alto \\ \hline
		Ancho de banda & 800Kbps & 11Mbps \\ \hline
		Dispositivos que lo utilizan & Dispositivos móviles, mouse, teclado, computadoras,etc. & Dispositivos móviles, computadoras, TV,etc. \\ \hline
		Requerimientos de hardware & Adaptador bluetooth entre los dispositivos conectados. & Adaptadores inalámbricos en todos los dispositivos de red o puntos de acceso. \\ \hline
		Rango & 5-30 metros & 32 metros en interiores y 95 metros en exteriores \\ \hline
		Consumo de energía & Bajo & Alto \\ \hline
		Factibilidad de uso & Sencillo de utilizar, se pueden conectar hasta 7 dispositivos a la vez. & La dificultad de implementación aumenta debido a que se requiere configurar el hardware y software. \\ \hline
		Latencia & 200 ms & 150 ms \\ \hline
	\end{tabular}
\caption{Tabla comparativa de comunicación inalámbrica}
\label{Comparacion_Conexiones}
\end{table}
Debido a que la distancia requerida para realizar el envió de datos entre el microcontrolador y el dispositivo móvil es corta, el costo que el dispositivo tiene, y a que cuenta con una implementación sencilla se decidió utilizar la conexión inalámbrica de bluetooth para realizar la comunicación.
\newline
Tras definir el tipo de comunicación es necesario elegir el dispositivo que sera utilizado, por tal motivo en la tabla \ref{Comparacion_DispBluetooth} se muestran las características con las que cuentan dos distintos tipos de modulo Bluetooth.
\\
\begin{table}[H]
	\centering
	\begin{tabular}{|M{4cm}|M{4cm}|M{4cm}|}
		\hline
		\textbf{Característica} & \textbf{HC-05} & \textbf{HC-06} \\ \hline
		Modo de configuración & Maestro/Esclavo/Esclavo con autoconexión. & Esclavo \\ \hline
		Frecuencia & 2.4GHz & 2.4GHz \\ \hline
		Modulación & GFSK & GFSK \\ \hline
		Potencia de emisión & $\leq$ 4dBm Clase 2 & $\leq$ 6dBm Clase 2 \\ \hline
		Alcance & 5m a 10m & 5m a 10m \\ \hline
		Velocidad & Asincrónica: 2.1 Mbps (max.)/160 kbps, sincrónica: 1 Mbps/1 Mbps & Asincrónica: 2 Mbps (max.)/160 kbps, sincrónica: 1 Mbps/1 Mbps \\ \hline
		Consumo de corriente & 50mA & 30 mA a 40 mA \\ \hline
		Voltaje de operación & 3.6 V a 6 V. & 3.6 V a 6 V \\ \hline
		Pines & Módulo montado en tarjeta con regulador de voltaje y 6 pines suministrando acceso a VCC, GND, TXD, RXD, KEY y status LED (STATE) & Módulo montado en tarjeta con regulador de voltaje y 4 pines suministrando acceso a VCC, GND, TXD, y RXD \\ \hline
		Precio & \$145 pesos & \$140 pesos\\ \hline
	\end{tabular}
	\caption{Tabla comparativa de dispositivos bluetooth}
	\label{Comparacion_DispBluetooth}
\end{table}
El dispositivo bluetooth a utilizar es el módulo bluetooth HC-05 ya que este tiene el modo de configuración como maestro-esclavo lo cual se ajusta a las necesidades del trabajo, ademas de que el precio es accesible y no tiene mucha variación con respecto al HC-06.
