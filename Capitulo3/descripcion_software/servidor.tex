\subsection{Servidor}
Un servidor es un ordenador u otro tipo de equipo informático encargado de suministrar información a una serie de clientes, que pueden ser tanto personas como otros dispositivos conectados a él. La información que puede transmitir es múltiple y variada: desde archivos de texto, imagen o vídeo y hasta programas informáticos, bases de datos, etc.
\textbf{Cloud Server}
Los cloud servers son unas alternativas para llevar la herramienta de los servidores al mundo virtual. La infraestructura en la nube se consigue gracias a la existencia de diversos servidores físicos controlados mediante un software, que es el encargado de virtualizar la plataforma.
\\Los servidores en la nube ofrecen a las empresas la posibilidad de tener un servidor a medida de sus necesidades, cuyos recursos y capacidades puedan ir incrementándose a conforme aumentan el tamaño y la actividad de la empresa, lo que permite un considerable ahorro para el presupuesto de las distintas corporaciones \citep{servidores}.
\\
En la tabla \ref{tabla_servidores} se tiene la comparación entre distintos cloud servers.
\begin{longtable}{|M{2.5cm}|M{4cm}|M{4.8cm}|M{3cm}|}
	\hline 
	\textbf{Cloud Server}& \textbf{Descripción} & \textbf{Características} & \textbf{Precio} \\ \hline
	Google Cloud Platform & Es una plataforma que ha reunido todas las aplicaciones de desarrollo web de Google. Google Cloud Platform es utilizada para crear soluciones a través de la tecnología almacenadas en la nube. &\begin{itemize}
		\item Permite la conexión por medio de SSH.
		\item Puedes desplegar el código directamente o mediante contenedores.
		\item Solo se paga por el tiempo utilizado.
		\item Maquinas virtuales personalizadas.
		\item Cuenta con un almacenamiento de hasta 624 GB por maquina.
		\item Proporciona almacenamiento en bloques en unidades de estado solido locales con encriptado permanente.
		\item Balanceo de carga global.
		\item Sistemas operativos: Debian, CentOS, CoreOS, SUSE, Ubuntu, Red Hat, FreeBSD o Windows 2008 R2, 2012 R2 y 2016
		\item Procesamiento por lotes.
	\end{itemize} &  Aproximadamente 0.1900 USD/hora \\ \hline
	Amazon Web Services & 
	Es una plataforma de servicios de nube que ofrece potencia de cómputo, almacenamiento de bases de datos, entrega de contenido y otra funcionalidad para ayudar a las empresas a escalar y crecer.
	&\begin{itemize}
		\item Instancias dedicadas: brindan acceso directo al procesador y a la memoria del servidor.
		\item Instancias de informática con GPU
		\item Instancias de E/S de alto desempeño: Ofrecen un desempeño de disco secuencia de hasta 16 GB/s
		\item Instancias de almacenamiento denso: Hasta 48 TB de almacenamiento.
		\item Volúmenes de almacenamiento en bloques persistente, de alta disponibilidad, constantes y de baja latencia
		\item Se paga por lo que se consuma
		\item Posibilidad de colocar las instancias en distintas ubicaciones.
		\item Auto Scaling
		\item Amazon Time Sync Service ofrece un origen de hora de alta precisión, fiabilidad y disponibilidad para los servicios de AWS
		\item Sistemas operativos: Amazon Linux, Windows Server 2012, CentOS 6.5, Debian.
	\end{itemize} & Aproximadamente 0,0832 USD por hora \\ \hline
	Heroku & Heroku es una plataforma en la nube basada en un sistema de contenedor administrado, con servicios de datos integrados y un poderoso ecosistema, para implementar y ejecutar aplicaciones modernas &\begin{itemize}
		\item Ejecuta las aplicaciones dentro de dynos:contenedores inteligentes en un entorno de tiempo de ejecución confiable y administrado.
		\item Soporta código escrito en Node, Ruby, Java, PHP, Pytho, Go, Scala y Clojure.
		\item Se puede implementar desde herramientas como Git, Github o sistemas de integración continua.
		\item Permite extender las aplicaciones con complementos.
		\item Sistema para escalar automáticamente las web dynos.
		\item Cuenta con métricas de la aplicación, alertas de umbral y escala automática.
		\item Seguridad
		\item Sistema operativo Linux
	\end{itemize} & Aproximadamente 7 USD por dyno/mes \\ \hline
	\caption{Tabla comparativa de cloud servers}
	\label{tabla_servidores}
\end{longtable}
El servicio a utilizar sera Google Cloud Platform debido a que conforme va creciendo el sistema nos ofrece el menor precio, ademas de que nos permite tener una conexión por medio de SSH y asimismo soporta el lenguaje que será utilizado para el desarrollo de la aplicación.