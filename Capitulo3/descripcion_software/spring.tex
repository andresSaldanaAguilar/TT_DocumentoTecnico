\subsection{Spring 4}
Spring Framework proporciona un completo modelo de programación y configuración para aplicaciones empresariales modernas basadas en Java, en cualquier tipo de plataforma de implementación.
Un elemento clave de Spring es el soporte de infraestructura en el nivel de la aplicación: Spring se centra en la "plomería" de las aplicaciones empresariales para que los equipos puedan enfocarse en la lógica empresarial a nivel de la aplicación, sin vínculos innecesarios con entornos de implementación específicos \citep{spring}.
\\
En la tabla \ref{tabla_frameworks} se muestra una tabla comparativa entre distintos frameworks.
\begin{longtable}{|M{3cm}|M{4.2cm}|M{4.2cm}|M{4.3cm}|}
	\hline 
	\textbf{Framework} & \textbf{Ventajas} & \textbf{Desventajas} \\ \hline
	Spring 4 & \begin{itemize}
		\item Lenguaje Java
		\item Fácil configuración
		\item Open source
		\item Estandarizado
	\end{itemize} &  \begin{itemize}
		\item Te atrae a que te adaptes al sistema.
	\end{itemize} \\ \hline
	Django & \begin{itemize}
		\item Desarrollo rápido
		\item Open source
		\item Fácil de aprender
		\item MVT
	\end{itemize} & \begin{itemize}
		\item Templates no tan robustos
		\item Se reinicia el servidor al recargar
		\item ORM no tan robusto
	\end{itemize} \\ \hline
	Node.js & \begin{itemize}
		\item Javascript
		\item High-performance
		\item Open source
		\item Asíncrono
	\end{itemize} & \begin{itemize}
		\item Limitado a una CPU
	\end{itemize} \\ \hline
	\caption{Tabla comparativa de diferentes frameworks}
	\label{tabla_frameworks}
\end{longtable}
Se eligió spring 4 como el framework para el desarrollo de la lógica del servidor, debido a que nos ofrece la posibilidad de utilizar el lenguaje Java, previamente elegido como lenguaje de programación del sistema, y nos brinda un soporte robusto de aplicaciones lo cual nos permite que el sistema funcione correctamente aun si este mismo crece.