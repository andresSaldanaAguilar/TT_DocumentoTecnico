\subsection{Herramientas de desarrollo de hardware}
Las siguientes herramientas son usadas para el desarrollo del hardware del sistema:
:
\begin{itemize}
	\item Atmel Studio 7.0: Es necesario contar con un IDE que nos permita desarrollar y probar el código que sera utilizado en la programación del microcontrolador, por tal motivo se utilizara Atmel Studio 7.0 ya que nos brinda una interfaz amigable para la programación del microcontrolador y cuenta con soporte para el microcontrolador AtMega16, ademas nos permite realizar pruebas del funcionamiento del código escrito al visualizar los datos de cada uno de los puertos del microcontrolador.
	\item Proteus 8: Para realizar los diagramas del hardware del sistema se utilizara el software Proteus, el cual nos permite simular el funcionamiento del circuito y nos proporciona una gran diversidad de componentes necesarios para realizar el diseño del hardware.
\end{itemize}

