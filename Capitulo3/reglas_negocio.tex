\section{Reglas de negocio}
Las reglas de negocio sirven para definir o restringir acciones que serán implementadas en funcionalidad. 
En la presente sección, se muestran las reglas de negocio del sistema.
\begin{enumerate}[label=RN\arabic*.]
    \item \label{RN1}
		\begin{description}
			\item[Nombre] RN1-Obtención de muestras del sistema fotovoltaico en tiempo real.
			\item[Tipo] Operación.
			\item[Objetivo] La obtención de mediciones del sistema fotovoltaico se realizará en tiempo real.
			\item[Descripción] La obtención de mediciones del sistema fotovoltaico dependerá del dispositivo de monitoreo empleado, en este caso, se trata del MCP39F521, y está determinada por la frecuencia de reloj de IIC de 400 Khz, sin embargo nuestro sistema trabajará a una frecuencia de 1 segundo.
    		\end{description}
    		
\item \label{RN2}
	\begin{description}
		\item[Nombre] RN2-Envío de muestras del sistema fotovoltaico en tiempo real.
		\item[Tipo] Operación.
		\item[Objetivo] Establecer el tiempo y frecuencia de envío de mediciones del microcontrolador por medio del módulo Wi-Fi.
		\item[Descripción] El envío de muestras del microcontrolador por medio del módulo Wi-Fi dependerá de la velocidad de transmisión UART entre el microcontrolador y el módulo, que es de 115200 baudios. Un baudio es el número de unidades de información transmitidas por segundo, que dependiendo de la configuración puede estar entre los 9600 a 11520 Bytes/Segundo, la frecuencia de envío de muestras hacia el servidor será de 1 segundo, obteniendo así una resolución de la unidad de generación de energía en Kilowatt Segundo para después ser convertida a Kilowatt Hora.
		\end{description}	
		
    \item \label{RN3}
		\begin{description}
			\item[Nombre] RN3-Creación de notificaciones por comportamiento inesperado.
			\item[Tipo] Habilitadora de acción.
			\item[Objetivo] Crear alertas que le informen al usuario el momento exacto en que el sistema fotovoltaico ha dejado de producir energía, el sistema ha dejado de obtener información de un nodo o el servidor no responde las peticiones.
			\item[Descripción] La alerta será creada y enviada como notificación al usuario de forma inmediata si y sólo si en el proceso de monitoreo para notificaciones se ha detectado alguno de los siguientes casos: la potencia activa de una muestra de un nodo es de 0 watts, si se pierde conexión con el microcontrolador o servidor.  
    		\end{description}
    		
\item \label{RN4}
		\begin{description}
			\item[Nombre] RN4-Calcular producción de energía histórica.
			\item[Tipo] Derivadora.
			\item[Objetivo] Calcular la producción de energía en el intervalo de tiempo especificado por cada nodo.
			\item[Descripción] La producción histórica de energía de un nodo es la suma de la potencia activa de cada una de las muestras obtenidas de acuerdo a la frecuencia de muestreo definida en la Regla de Negocio \ref{RN2}  
			Las opciones de periodos históricos son: última semana, último mes, mensual, bimestral y anual. Los reportes para los correspondientes periodos son:
			\\ Última semana: 
			\begin{itemize}
				\item La producción por cada día de la última semana, es decir: Lunes, Martes, Miércoles, Jueves, Viernes, Sábado y Domingo.
			\end{itemize}
			Último mes: 
			\begin{itemize}
				\item La producción por cada día del último mes.
			\end{itemize}
			Mensual: 
			\begin{itemize}
				\item La producción por cada mes del año.
			\end{itemize}
			Bimestral: 
			\begin{itemize}
				\item La producción en el periodo de tiempo comprendido del 1 de Enero al 28 o 29 de Febrero (dependiendo de si el año es bisiesto)
				\item La producción en el periodo de tiempo comprendido del 1 de Marzo al 30 de Abril
				\item La producción en el periodo de tiempo comprendido del 1 de Mayo al 30 de Junio
				\item La producción en el periodo de tiempo comprendido del 1 de Julio al 31 de Agosto
				\item La producción en el periodo de tiempo comprendido del 1 de Septiembre al 31 de Octubre
				\item La producción en el periodo de tiempo comprendido del 1 de Noviembre al 31 de Diciembre
			\end{itemize}
			Anual: 
			\begin{itemize}
				\item La producción por mes del año seleccionado.
			\end{itemize}
		\end{description}

\item \label{RN5}
		\begin{description}
			\item[Nombre] RN5- Agrupar muestras tomadas en un día por intervalos.
			\item[Tipo] Derivadora.
			\item[Objetivo] Agrupar las muestras tomadas en un día por cada nodo en intervalos.
			\item[Descripción] Las muestras diarias de cada nodo serán agrupadas en los siguientes intervalos:
			\begin{itemize}
				\item Muestras mayores o iguales al 50\% de la máxima generación del sistema fotovoltaico. Estarán representadas de color verde.  
				\item Muestras menores que el 50\% y mayores o iguales al 20\% de la máxima generación del sistema fotovoltaico. Estarán representadas de color naranja. 
				\item Muestras menores al 20\% de la máxima generación del sistema fotovoltaico. Estarán representadas de color rojo. 
			\end{itemize}
		\end{description}

\item \label{RN6}
		\begin{description}
			\item[Nombre] RN6-Periodo de toma de muestras.
			\item[Tipo] Operación.
			\item[Objetivo] Establecer el periodo de tiempo en el que las muestras serán tomadas para cada nodo.
			\item[Descripción] Las muestras para obtener el resultado de la producción de energía comenzarán a tomarse a partir de las 0:00:00 hrs hasta las 23:59:59 en la frecuencia indicada en la \ref{RN2}; adicionalmente se guardará cada minuto el valor de la potencia activa de cada nodo para la generación de la gráfica en tiempo real, dando un total de 1440 muestras diarias.		
		\end{description}
		
%Checar tipo
\item \label{RN7}
		\begin{description}
			\item[Nombre] RN7-Estándar de comunicación.
			\item[Tipo] Habilitadora de acción.
			\item[Objetivo] Establecer el estándar de comunicación entre microcontrolador-servidor y servidor-aplicación de usuario.
			\item[Descripción] El estándar de comunicación entre el servidor y la aplicación de usuario será por medio de TCP, y para la comunicación entre microcontrolador y servidor será por medio de UDP, ambos proporcionan la base para los productos con redes inalámbricas que hacen uso de la marca Wi-Fi.
		\end{description}
		
%Checar tipo		
\item \label{RN8}
		\begin{description}
			\item[Nombre] RN8-Patrón de intercambio de mensajes.
			\item[Tipo] Habilitadora de acción.
			\item[Objetivo] Establecer el patrón de intercambio de información entre módulos.
			\item[Descripción] El patrón de intercambio de información a utilizar entre módulos será petición-respuesta para el caso del protocolo TCP, para UDP solamente se hace el envío del paquete.
		\end{description}
		
%Después definir los datos exactos? o en req técnicos		
\item \label{RN9}
		\begin{description}
			\item[Nombre] RN9-Almacenamiento de datos de monitoreo.
			\item[Tipo] Operación.
			\item[Objetivo] Establecer el método de almacenamiento de la información.
			\item[Descripción] El almacenamiento en el servidor está limitado por la memoria externa con capacidad desde 8GB de espacio que le es agregada, por lo que usaremos una herramienta que trabaja con una cantidad de datos fija, sobreescribiendo la información cuando terminan e inician nuevamente los periodos definidos en la Regla de Negocio \ref{RN4}		\end{description}

%Preguntar request-reply si es RN8?

%Definí como tipo, preguntar si otra regla debemos poner el formato (bien conformado con un xsd)?
%Checar si será XML o JSON
\item \label{RN10}
		\begin{description}
			\item[Nombre] RN10-Formato de archivo para almacenamiento de datos.
			\item[Tipo] Estructura.
			\item[Objetivo] Definir el formato de archivo en que serán almacenados los datos sensados.
			\item[Descripción] El formato en que serán almacenados los datos sensados está definido por un archivo con extensión JSON, estos archivos estarán contenidos dentro de la carpeta que corresponde al número de sensor esclavo correspondiente y a su vez dentro de la carpeta con el número de serie del microcontrolador correspondiente, que está identificado por dos letras alfabéticas, por ejemplo, AC.		
		\end{description}
		
\item \label{RN11}
		\begin{description}
			\item[Nombre] RN11-Formato de datos de respuesta para la aplicación de usuario.
			\item[Tipo] Estructura.
			\item[Objetivo] Definir el formato de los datos que se enviarán como respuestas a las peticiones del cliente.
			\item[Descripción] El formato de los datos que se darán como respuesta del servidor a la aplicación móvil será en formato JSON.
			\end{description}
		
\item \label{RN12}
		\begin{description}
			\item[Nombre] RN12-Umbrales de generación
			\item[Tipo] Regla de operación.
			\item[Objetivo] Definir los umbrales de generación de energía de acuerdo a la generación en tiempo real y los intervalos definidos 
			\item[Descripción] Los umbrales se dividirán  en tres para todos los casos, generación óptima, intermedia y baja.
			El criterio a comparar será el valor de generación obtenido contra el valor de generación esperado por un panel solar, que es de 250 Watts/Hora, donde el porcentaje obtenido por la relación se clasificará de la siguiente manera:
			\begin{itemize}
			    \item 80\% - 100\% Generación Óptima, representado con el color verde.
			    \item 50\% - 79\% Generación Intermedia, representado con el color amarillo.
			    \item 0\% - 49\% Generación Baja, representado con el color rojo.
			\end{itemize}
		\end{description}
	
\item \label{RN13}
		\begin{description}
			\item[Nombre] RN13-Número de registros de notificaciones.
			\item[Tipo] Restricción.
			\item[Objetivo] Definir la cantidad máxima de registros de notificaciones a conservar en la aplicación móvil.
			\item[Descripción] El número máximo de registros de notificaciones que se permitirán conservar en la aplicación móvil tiene que ser menor o igual a 100 notificaciones. El criterio para conservar las notificaciones dependerá únicamente del tiempo de llegada, es decir, se conservarán las notificaciones más recientes, sobreescribiendo el archivo cuando se llegue al límite.   
		\end{description}
		
\item \label{RN14}
		\begin{description}
			\item[Nombre] RN14-Formato de fecha de notificaciones.
			\item[Tipo] Estructura.
			\item[Objetivo] Definir el formato de fecha de las notificaciones para la aplicación móvil.
			\item[Descripción] El formato de fecha en que serán almacenadas las notificaciones del sistema en la aplicación móvil es: dd/mm/aa.
			Donde:
			\begin{itemize}
		 		\item dd = Corresponde al número de día del mes.
		 		\item mm = Corresponde al número de mes del año.
		 		\item aa = Corresponde al año
		    \end{itemize}
		\end{description}
		
\item \label{RN15}
		\begin{description}
			\item[Nombre] RN15-Formato de hora de notificaciones.
			\item[Tipo] Estructura.
			\item[Objetivo] Definir el formato de hora de las notificaciones para la aplicación móvil.
			\item[Descripción] El formato de hora en que serán almacenadas las notificaciones del sistema en la aplicación móvil es: hh:mm.
			Donde:
			\begin{itemize}
		 		\item hh = Corresponde a la hora del día en formato de 24 horas.
		 		\item mm = Corresponde al minuto de la hora.
		    \end{itemize}
		\end{description}
%Estipular cuantas muestras se tomaran por dia y bimestralmente y si se deshecharán,
%Tipo de archivo en el que se almacenarán los datos

\item \label{RN16}
		\begin{description}
			\item[Nombre] RN16-Formato de IP válido.
			\item[Tipo] Estructura.
			\item[Objetivo] Definir el formato IP válido para poder hacer la conexión con el servidor.
			\item[Descripción] El formato de IP válido será N.N.N.N
			Donde:
			\begin{itemize}
		 		\item N = número entero menor a 255 y mayor o igual a 0
		    \end{itemize}
		\end{description}
\end{enumerate}

