\section{Reglas de negocio}
En la presente sección, se muestran las reglas de negocio del sistema.
\begin{enumerate}[label=RN\arabic*.]
    \item \label{RN1}
		\begin{description}
			\item[Nombre] RN1-Obtención de muestras del sistema fotovoltaico en tiempo real.
			\item[Tipo] Regla de operación.
			\item[Objetivo] La obtención de mediciones del sistema fotovoltaico se realizará en tiempo real.
			\item[Descripción] La obtención de mediciones del sistema fotovoltaico dependerá del dispositivo de monitoreo empleado, en este caso, se trata del MCP39F521, y está determinada por la frecuencia de reloj de IIC de 400 Khz.
    		\end{description}
    		
\item \label{RN2}
	\begin{description}
		\item[Nombre] RN2-Envío de muestras del sistema fotovoltaico en tiempo real.
		\item[Tipo] Regla de operación.
		\item[Objetivo] El envío de mediciones del microcontrolador por medio del módulo Wi-Fi se realizará en tiempo real.
		\item[Descripción] El envío de muestras del microcontrolador por medio del módulo Wi-Fi dependerá de la velocidad de transmisión UART entre el microcontrolador y el módulo, que es de 115200 baudios. Un baudio es el número de unidades de información transmitidas por segundo que, dependiendo de la configuración, puede estar entre los 9600 a 11520 Bytes/seg.
		\end{description}	
		
    \item \label{RN3}
		\begin{description}
			\item[Nombre] RN3-Creación de alertas cuando se detecta que el sistema fotovoltaico dejó de producir energía.
			\item[Tipo] Regla de estímulo y respuesta.
			\item[Objetivo] Crear alertas que le informen al usuario el momento exacto en que el sistema fotovoltaico ha dejado de producir energía.
			\item[Descripción] La alerta será creada y enviada como notificación al usuario de forma inmediata si y sólo si en el proceso de monitoreo se ha detectado que la medición de energía producida es de 0 volts.  
    		\end{description}
    		
\item \label{RN4}
		\begin{description}
			\item[Nombre] RN4-Calcular promedio de energía sensada historicamente.
			\item[Tipo] Derivadora.
			\item[Objetivo] Calcular el promedio del intervalo de tiempo requerido.
			\item[Descripción] El informe de históricos al usuario mediante la aplicación móvil puede ser el promedio de la energía sensada semanal, mensual o bimestralmente y se realizará el cálculo de este promedio con los siguientes periodos:
			\\ Semanalmente: 
			\begin{itemize}
				\item El promedio por cada día de la semana, de lunes a domingo, es decir, siete promedios.
			\end{itemize}
			\\ Mensualmente: 
			\begin{itemize}
				\item El promedio por cada semana del mes, desde la semana donde el mes inicia hasta la semana donde termina, es decir entre cinco y seis promedios.
			\end{itemize}
			\\ Bimestralmente: 
			\begin{itemize}
				\item El promedio del 1 de Enero al 28 o 29 de Febrero (dependiendo de si el año es bisiesto)
				\item El promedio del 1 de Marzo al 30 de Abril
				\item El promedio del 1 de Mayo al 30 de Junio
				\item El promedio del 1 de Julio al 31 de Agosto
				\item El promedio del 1 de Septiembre al 31 de Octubre
				\item El promedio del 1 de Noviembre al 31 de Diciembre
			\end{itemize}
		\end{description}

\item \label{RN5}
		\begin{description}
			\item[Nombre] RN5-Calcular promedio de energía sensada diariamente.
			\item[Tipo] Derivadora.
			\item[Objetivo] Calcular promedio de energía sensada diariamente
			\item[Descripción] La cantidad promedio de las 120 muestras tomadas diariamente como lo especifica la regla de negocio RN5.
		\end{description}

\item \label{RN6}
		\begin{description}
			\item[Nombre] RN6-Periodo de toma de muestras para cálculo de promedio diario de energía sensada.
			\item[Tipo] Operacional.
			\item[Objetivo] Establecer el periodo de tiempo en el que las muestras para cálculo de promedio diario de energía sensada serán tomadas.
			\item[Descripción] Las muestras comenzarán a tomarse y serán almacenadas a partir de las 8:00 a.m. hasta las 6:00 p.m. cada 5 minutos.
		\end{description}
		
%Checar tipo
\item \label{RN7}
		\begin{description}
			\item[Nombre] RN7-Estándar de comunicación.
			\item[Tipo] Derivación.
			\item[Objetivo] Establecer el estándar de comunicación entre microcontrolador-servidor y servidor-aplicación de usuario.
			\item[Descripción] El estándar de comunicación es el IEEE 802.11 que proporciona la base para los productos con redes inalámbricas que hacen uso de la marca Wi-Fi.
		\end{description}
		
%Checar tipo		
\item \label{RN8}
		\begin{description}
			\item[Nombre] RN8-Patrón de intercambio de mensajes.
			\item[Tipo] Derivación.
			\item[Objetivo] Establecer el patrón de intercambio de información entre módulos.
			\item[Descripción] El patrón de intercambio de información a utilizar entre módulos será petición-respuesta.
		\end{description}
		
%Después definir los datos exactos? o en req técnicos		
\item \label{RN9}
		\begin{description}
			\item[Nombre] RN9-Almacenamiento de datos de monitoreo.
			\item[Tipo] Regla de operación.
			\item[Objetivo] Establecer el método de almacenamiento de la información.
			\item[Descripción] El almacenamiento en el servidor está limitado por la memoria externa con capacidad de N gb de espacio que le es agregada, por lo que usaremos una herramienta que trabaja con una cantidad de datos fija, definida en el momento de crear el archivo contenedor, y un puntero al elemento actual.
		\end{description}

%Preguntar request-reply si es RN8?

%Definí como tipo, preguntar si otra regla debemos poner el formato (bien conformado con un xsd)?
%Checar si será XML o JSON
\item \label{RN10}
		\begin{description}
			\item[Nombre] RN10-Formato de datos almacenados.
			\item[Tipo] Regla de restricción.
			\item[Objetivo] Definir el formato en que serán almacenados los datos sensados.
			\item[Descripción] El formato en que serán almacenados los datos sensados está definido por un archivo con extensión XML/JSON. 
		\end{description}
		
\item \label{RN11}
		\begin{description}
			\item[Nombre] RN11-Formato de datos de respuesta para la aplicación de usuario.
			\item[Tipo] Regla de operación.
			\item[Objetivo] Definir el formato de los datos en que se enviaran las respuestas a las peticiones del cliente.
			\item[Descripción] El formato de los datos que se darán como respuesta a la aplicación será en texto plano mediante una cadena o en un arreglo de cadenas.
		\end{description}
		
\item \label{RN12}
		\begin{description}
			\item[Nombre] RN12-Umbrales de generación
			\item[Tipo] Regla de operación.
			\item[Objetivo] Definir los umbrales de generación de energía de acuerdo a la generación en tiempo real y los intervalos definidos 
			\item[Descripción] Los umbrales se dividirán en todos los casos en tres, generación óptima, intermedia y baja.
			Tiempo real: el criterio comparar sera el valor de generación obtenido contra el valor de generación esperado por un panel solar, que es de 250 Watts, donde la proporción nos dirá lo siguiente:
			80\% - 100\% Generación Óptima
			50\% - 79\% Generación Intermedia
			0\% - 49\% Generación Baja
		\end{description}
		
%Estipular cuantas muestras se tomaran por dia y bimestralmente y si se deshecharán,

%Tipo de archivo en el que se almacenarán los datos

\end{enumerate}