\section{Reglas de negocio}
En la presente sección, se muestran las reglas de negocio del sistema.
\begin{enumerate}[label=RN\arabic*.]
    \item \label{RN1}
		\begin{description}
			\item[Nombre] RN1-Monitorizar el sistema fotovoltaico en tiempo real.
			\item[Tipo] Regla de operación.
			\item[Objetivo] La obtención de mediciones del sistema fotovoltaico se realizará en tiempo real.
			\item[Descripción] La obtención de mediciones del sistema fotovoltaico dependerá del dispositivo de monitoreo empleado, en este caso, se trada del MCP39F521, y está determinada por la frecuencia de reloj de IIC de 400 Khz. 
    		\end{description}
    		
    \item \label{RN2}
		\begin{description}
			\item[Nombre] RN2-Creación de alertas cuando se detecta que el sistema fotovoltaico dejó de producir energía.
			\item[Tipo] Regla de estímulo y respuesta.
			\item[Objetivo] Crear alertas que le informen al usuario el momento exacto en que el sistema fotovoltaico ha dejado de producir energía.
			\item[Descripción] La alerta será creada y enviada como notificación al usuario de forma inmediata si y sólo si en el proceso de monitoreo se ha detectado que la medición de energía producida es de 0 volts.  
    		\end{description}
\item \label{RN3}
		\begin{description}
			\item[Nombre] RN1-Calcular promedio de energía sensada bimestralmente.
			\item[Tipo] Derivadora.
			\item[Objetivo] El informe que se notifica al usuario mediante la aplicación móvil será el promedio de la energía promedio sensada diariamente, la cual se calcula con los registros de los últimos 2 meses, únicamente se realizará el cálculo de este promedio durante los siguientes periodos:
			\begin{itemize}
				\item 1 de Enero al 28 o 29 de Febrero (dependiendo de si el año es bisiesto)
				\item 1 de Marzo al 30 de Abril
				\item 1 de Mayo al 30 de Junio
				\item 1 de Julio al 31 de Agosto
				\item 1 de Septiembre al 31 de Octubre
				\item 1 de Noviembre al 31 de Diciembre
			\end{itemize}
			\item[Descripción] Se realizará el promedio de los últimos 60 promedios diarios para el calculo bimestral
		\end{description}

\item \label{RN4}
		\begin{description}
			\item[Nombre] RN1-Calcular promedio de energía sensada diariamente.
			\item[Tipo] Derivadora.
			\item[Objetivo] La cantidad promedio de las 120 muestras tomadas diariamente como lo especifica la regla de negocios "Periodo de toma de muestras".
			\item[Descripción] Se realizará el promedio de los últimos 60 promedios diarios para el calculo bimestral
		\end{description}

\item \label{RN5}
		\begin{description}
			\item[Nombre] RN1-Periodo de toma de muestras.
			\item[Tipo] Operacional.
			\item[Objetivo] Las muestras comenzarán a tomarse y serán almacenadas a partir de las 8:00 a.m. hasta las 6:00 p.m. cada 5 minutos.
			\item[Descripción] Se realizará el promedio de los últimos 60 promedios diarios para el calculo bimestral
		\end{description}













%___________________-RN's Elioth
	\item \label{RN1}
		\begin{description}
			\item[Nombre] RN1-Calcular cantidad de combustible recargado.
			\item[Tipo] Derivadora.
			\item[Objetivo] La cantidad de combustible recargado a un automóvil se calcula con una fórmula especifica dependiente del sensor de hardware usado.
			\item[Descripción] La cantidad de combustible recargado a un automóvil, se calcula con la fórmula: $$L=(F/4.8)*T$$
			Donde:
				\begin{itemize}
					\item L=Total de litros recargados
					\item F=Frecuencia de la señal (tren de impulsos) enviada por el sensor de hardware, se calcula por la cantidad de impulsos (voltaje en alta) que se reciban en un segundo.
					\item T= Es el tiempo durante el que se recibió la señal, el tiempo se calcula al establecer una marca de tiempo en la cual se recibió el primer impulso, y establecer otra marca de tiempo en el momento en que se han dejado de recibir. Restando al segundo el primero.\cite{FS400A-G1}
				\end{itemize}
			% \item[Ejemplo]
		\end{description}

	\item \label{RN2}
		\begin{description}
			\item[Nombre] RN2-Tiempo posterior a la carga de combustible.
			\item[Tipo] Controladora.
			\item[Objetivo] Especificar la cantidad de tiempo, en la que pasado este se considera que la recarga de combustible en el automóvil haya culminado.
			\item[Descripción] Se considera que se ha dejado de cargar gasolina a un automóvil sí ha pasado más de un segundo sin que el sensor de flujo envíe ningún impulso.
			% \item[Ejemplo]
		\end{description}
	\item \label{RN3}
		\begin{description}
			\item[Nombre] RN3-Otorgar insignia a actor.
			\item[Tipo] Derivadora.
			\item[Objetivo] Determinar las condiciones bajo las cuales un actor recibe una insignia.
			\item[Descripción] Un actor recibe una insignia cada vez que acumula cien mediciones.
			% \item[Ejemplo]
		\end{description}
	\item \label{RN4}
		\begin{description}
			\item[Nombre] RN4-Registrar precio gasolina.
			\item[Tipo] Controladora.
			\item[Objetivo] Determinar que actores pueden registrar el precio de la gasolina después de haber cargado.
			\item[Descripción] Solamente un actor con insignias puede registrar el precio de la gasolina.
			% \item[Ejemplo]
		\end{description}
	\item \label{RN5}
		\begin{description}
			\item[Nombre] RN5-Asignar insignia a gasolinera.
			\item[Tipo] Controladora.
			\item[Objetivo] Determinar que actores pueden asignarle una insignia a una gasolinera.
			\item[Descripción] Solamente un actor con insignias puede asignarle insignias a una gasolinera después de carga gasolina en la misma.
			% \item[Ejemplo]
		\end{description}
	\item \label{RN6}
		\begin{description}
			\item[Nombre] RN6-Especificar bomba.
			\item[Tipo] Controladora.
			\item[Objetivo] Determinar que actores pueden especificar en que bomba cargaron gasolina.
			\item[Descripción] Solamente un actor con al menos una insignia puede especificar en que bomba cargó gasolina.
			% \item[Ejemplo]
		\end{description}
	\item \label{RN7}
		\begin{description}
			\item[Nombre] RN7-Actualizar clasificación de gasolineras.
			\item[Tipo] Controladora.
			\item[Objetivo] Determinar el tiempo que debe pasar entre cada actualización de la clasificación de gasolineras.
			\item[Descripción] La clasificación de gasolineras se actualiza cada sesenta minutos.
			% \item[Ejemplo]
		\end{description}
	\item \label{RN8}
		\begin{description}
			\item[Nombre] RN8-Algoritmo de clasificación.
			\item[Tipo] Derivadora.
			\item[Objetivo] Especificar el algoritmo que es usado para clasificar a las gasolineras con base en las mediciones registradas para cada una.
			\item[Descripción] El algoritmo de clasificación usado es descrito en está parte.
			% \item[Ejemplo]
		\end{description}
	\item \label{RN9}
		\begin{description}
			\item[Nombre] RN9-Radio cercano al actor para obtener gasolineras.
			\item[Tipo] Derivadora.
			\item[Objetivo] Especificar en que radio se obtendrán las gasolineras cercanas al actor al abrir la aplicación móvil.
			\item[Descripción] El radio es de quinientos metros a partir de la ubicación del actor.
			% \item[Ejemplo]
		\end{description}
\end{enumerate}