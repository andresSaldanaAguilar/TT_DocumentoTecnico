\section{Requerimientos funcionales}
A continuación, se presenta un listado con los requerimientos funcionales que se obtuvieron.
El listado de requerimientos funcionales se encuentra divido de acuerdo con los módulos que se identificaron:
\begin{itemize}
	\item Módulo de Microcontrolador.
	\item Módulo de Servidor.
	\item Módulo de Aplicación de usuario.
\end{itemize}

\paragraph{Módulo de Microcontrolador}
\begin{enumerate}[label=RF\arabic*.]
	\item Proporcionar un mecanismo de comunicación vía IIC entre el transductor MCP39F521 y el microcontrolador DSPIC30F4013.
	\item Proporcionar un mecanismo de comunicación vía UART entre el microcontrolador y el módulo Wi-Fi MIKROE-2542.
\end{enumerate}

\paragraph{Módulo de Servidor.}
\begin{enumerate}[label=RF\arabic*.]
	\setcounter{enumi}{12}
	\item Proporcionar una interfaz para registrar los datos usados por el usuario durante el suministro de gasolina como, la cantidad de litros de gasolina cargados y el precio por litro.
	\item Proporcionar un dispositivo electrónico que permita la medición de los litros de gasolina ingresados al automóvil.
	\item Proporcionar un mecanismo que permita la comunicación vía Bluetooth entre el dispositivo electrónico y una aplicación móvil,
	\item Proporcionar un mecanismo que permita la comunicación vía HTTP entre una aplicación móvil y un servidor que reciba los datos de las mediciones recabadas por el dispositivo electrónico.
\end{enumerate}

\paragraph{Módulo de Aplicación de usuario.}
\begin{enumerate}[label=RF\arabic*.]
	\setcounter{enumi}{16}
	\item Proporcionar un algoritmo para la clasificación de las gasolineras de acuerdo con las mediciones recabadas.
	\item Proporcionar un mecanismo para el registro, edición, consulta y eliminación de los datos de una gasolinera.
	\item Proporcionar un mecanismo para la asignación de insignias a una gasolinera.
	\item Consultar las insignias obtenidas por una gasolinera.
\end{enumerate}