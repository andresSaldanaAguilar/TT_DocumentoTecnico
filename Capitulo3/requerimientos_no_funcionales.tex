\section{Requerimientos no funcionales}
Los requerimientos no funcionales describen características o propiedades del sistema. Especifican criterios para evaluar la operación de un servicio.
\\
A continuación, se presenta un listado con los requerimientos no funcionales \citep{RNF1} que fueron identificados para el sistema :
%\begin{enumerate}[label=RNF\arabic*.]
%   \item Eficiencia: El sistema debe ser capaz de operar adecuadamente y responder a través de la aplicación móvil al usuario en a lo más 4 segundos.
	
%	\item Escalabilidad: La arquitectura modular bajo la que es construido el sistema permite que este sea escalable y pueda expandirse hacia más funcionalidades.
	
%	\item Usabilidad: La aplicación móvil cuenta con un diseño ergonómico y de fácil comprensión de tal forma que un usuario pueda utilizarla sin complicaciones, y a su vez con un tiempo de aprendizaje de a lo más dos horas.
	
%	\item Tiempo de ejecución: Cuando se envíe una petición al servidor por parte de la aplicación móvil para mostrar la información de la energía sensada, este debe responder en un tiempo máximo de 4 segundos. **
	
%	\item Disponibilidad: El sistema debe tener una disponibilidad del 99,99\% de las veces en que un usuario intente hacer uso de las funcionalidades a través de la aplicación móvil. **
	
%	\item Tolerancia a fallos: La probabilidad de que la aplicación detenga su funcionamiento debe ser máximo del 2\%. **
	%%checar n
%	\item Concurrencia: La aplicación móvil debe soportar n peticiones sin que el tiempo de respuesta por parte del servidor para todas ellas sea mayor a 10 segundos. **
%\end{enumerate}

\begin{longtable}{|M{1.6cm}|M{4.0cm}|M{6.0cm}|}
    \caption{Requerimientos no funcionales}
	\hline
	\textbf{RNF} & \textbf{Nombre} & \textbf{Descripción} \\
	\hline
 	\begin{enumerate}[label=RNF\arabic*]
 	    \item.
 	\end{enumerate}
 	& Eficiencia.
 	& El sistema debe ser capaz de operar adecuadamente y responder a través de la aplicación móvil al usuario en a lo más 4 segundos.\\
    \hline
    \begin{enumerate}[label=RNF\arabic*]
        \setcounter{enumi}{1}
 	    \item.
 	\end{enumerate}
 	& Escalabilidad.
 	& La arquitectura modular bajo la que es construido el sistema permite que este sea escalable y pueda expandirse hacia más funcionalidades.\\
    \hline
    \begin{enumerate}[label=RNF\arabic*]
        \setcounter{enumi}{2}
 	    \item.
 	\end{enumerate}
 	& Usabilidad. 
 	& La aplicación móvil cuenta con un diseño ergonómico y de fácil comprensión de tal forma que un usuario pueda utilizarla sin complicaciones, y a su vez con un tiempo de aprendizaje de a lo más dos horas.\\
    \hline
    \begin{enumerate}[label=RNF\arabic*]
        \setcounter{enumi}{3}
 	    \item.
 	\end{enumerate}
 	& Tiempo de ejecución.
 	& Cuando se envíe una petición al servidor por parte de la aplicación móvil para mostrar la información de la energía sensada, este debe responder en un tiempo máximo de 4 segundos.  \\
    \hline
    \begin{enumerate}[label=RNF\arabic*]
        \setcounter{enumi}{4}
 	    \item.
 	\end{enumerate}
 	& Disponibilidad.
 	& El sistema debe tener una alta disponibilidad del 99\% esto quiere decir que el sistema puede fallar y no estar disponible a lo máximo un acumulado de 3 días y 15 horas al año.  \\
    \hline
    \begin{enumerate}[label=RNF\arabic*]
        \setcounter{enumi}{5}
 	    \item.
 	\end{enumerate}
 	& Tolerancia a fallos.
 	& La probabilidad de que la aplicación detenga su funcionamiento debe ser máximo del 1\% anual.  \\
    \hline
    \begin{enumerate}[label=RNF\arabic*]
        \setcounter{enumi}{6}
 	    \item.
 	\end{enumerate}
 	& Concurrencia.
 	& La aplicación móvil debe soportar n peticiones sin que el tiempo de respuesta por parte del servidor para todas ellas sea mayor a 10 segundos. ** \\
    \hline
\end{longtable}