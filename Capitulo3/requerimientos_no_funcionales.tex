\section{Requerimientos no funcionales}
A continuación, se presenta un listado con los requerimientos no funcionales que fueron identificados para el sistema:
\begin{enumerate}[label=RNF\arabic*.]
    \item Eficiencia: El sistema debe ser capaz de operar adecuadamente y responder a través de la aplicación móvil al usuario en a lo más 7 segundos.
	\item Escalabilidad: La arquitectura bajo la que es construido el sistema permite que este sea escalable y pueda expandirse hacia más funcionalidades.
	\item Seguridad: Los datos almacenados en el archivo de formato de texto correspondiente serán cifrados utilizando el algoritmo AES. **
	\item Usabilidad: La aplicación móvil cuenta con un diseño ergonómico y de fácil comprensión de tal forma que un usuario pueda utilizarla sin complicaciones, y a su vez con un tiempo de aprendizaje de a lo más dos horas.
	\item Tiempo de ejecución: Cuando se envíe una petición al servidor por parte de la aplicación móvil para mostrar la información de la energía censada, este debe responder en un tiempo máximo de 5 segundos. **
	\item Versiones de android: La aplicación móvil será desarrollada para versiones android igual o mayores a la versión 5.0.
	\item Tamaño: La aplicación móvil no podrá ocupar más de 700 mb de espacio en memoria interna del teléfono celular.
	\item Disponibilidad: El sistema debe tener una disponibilidad del 99,99\% de las veces en que un usuario intente hacer uso de las funcionalidades a través de la aplicación móvil. **
	\item Tolerancia a fallos: La probabilidad de falla del sistema no podrá ser mayor a 0,05\% y cuando se presente un error, este no tardará más de 15 minutos en restaurarse a un estado válido. **
	\item Concurrencia: La aplicación móvil debe soportar 1000 peticiones sin que el tiempo de respuesta por parte del servidor para todas ellas sea mayor a 10 segundos. **
\end{enumerate}