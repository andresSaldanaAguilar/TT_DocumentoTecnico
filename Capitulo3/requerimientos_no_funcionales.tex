\section{Requerimientos no funcionales}
A continuación, se presenta un listado con los requerimientos no funcionales \citep{RNF1} que fueron identificados para el sistema :
\begin{enumerate}[label=RNF\arabic*.]
    \item Eficiencia: El sistema debe ser capaz de operar adecuadamente y responder a través de la aplicación móvil al usuario en a lo más 4 segundos.
	
	\item Escalabilidad: La arquitectura modular bajo la que es construido el sistema permite que este sea escalable y pueda expandirse hacia más funcionalidades.
	
	\item Usabilidad: La aplicación móvil cuenta con un diseño ergonómico y de fácil comprensión de tal forma que un usuario pueda utilizarla sin complicaciones, y a su vez con un tiempo de aprendizaje de a lo más dos horas.
	\item Tiempo de ejecución: Cuando se envíe una petición al servidor por parte de la aplicación móvil para mostrar la información de la energía sensada, este debe responder en un tiempo máximo de 4 segundos. **
	\item Disponibilidad: El sistema debe tener una disponibilidad del 99,99\% de las veces en que un usuario intente hacer uso de las funcionalidades a través de la aplicación móvil. **
	\item Tolerancia a fallos: La probabilidad de que la aplicación detenga su funcionamiento debe ser máximo del 2\%. **
	%%checar n
	\item Concurrencia: La aplicación móvil debe soportar n peticiones sin que el tiempo de respuesta por parte del servidor para todas ellas sea mayor a 10 segundos. **
\end{enumerate}