\subsection{Factibilidad operativa}
La factibilidad operativa tiene como objetivo comprobar que la empresa u organización será capaz de darle uso al sistema, que cuenta con el personal capacitado para hacerlo o tiene los recursos humanos necesarios para mantener el sistema\cite{factOperativa}.
\\
Para saber si el producto es factible operativamente o no, es necesario conocer los indices de aprobación del sistema entre los habitantes de la Ciudad de México; para conocer dicho parámetro se realizó una encuesta a una parte de la población, obteniendo que el sistema impacta de forma positiva a más de la mitad de la población encuestada, siendo una herramienta de utilidad para los mismos.
\\
Con la finalidad de garantizar el buen funcionamiento y uso del sistema, se planteo realizar la aplicación en Android, contando con la aprobación de los usuarios encuestados en su mayoría.
\\
La necesidad de los usuarios por conocer cuales son las gasolineras que mejor despachan la gasolina, lleva a la aceptación de la aplicación móvil ya que brinda una solución a su necesidad, ademas de que por su practicidad y su facilidad de operación lleva a que el sistema sea factible operativamente.
\\
Para conocer a detalle los resultados de la encuesta revisar el Apéndice B.
%Lo que teniamos antes
\begin{comment}
En la Tabla \ref{tabla_factibilidad_operativa}, se muestran los diferentes aspectos analizados para el desarrollo del presente trabajo.

\begin{longtable}{|M{2.5cm}|M{3cm}|M{2cm}|M{2cm}|M{2cm}|M{2cm}|}
	%\centering
		\hline
		\textbf{Personal} & \textbf{Actividades} & \textbf{Salario Mensual} & \textbf{Cantidad de Personas} & \textbf{Tiempo} & \textbf{Salario total por desarrollo} \\ \hline
		Programador Android &
		Desarrollo de la aplicación móvil.
		 & \$8 000 & 1 & 6 meses &  \$48 000 \\\hline
		Administrador de base de datos & Diseño e implementación de la base de datos & \$8 000 & 1 & 6 meses  & \$48 000 \\\hline
		Ing. en sistemas computacionales & 
		\begin{itemize}
			\item Análisis y diseño del servidor web
			\item Desarrollo del hardware del proyecto
		\end{itemize}
		& \$8 000 & 1 & 6 meses & \$48 000 \\ \hline
		\multicolumn{5}{|r|}{Total} & \$144 000 \\ \hline
	\caption{Factibilidad operativa}
	\label{tabla_factibilidad_operativa} 
\end{longtable}
\end{comment}