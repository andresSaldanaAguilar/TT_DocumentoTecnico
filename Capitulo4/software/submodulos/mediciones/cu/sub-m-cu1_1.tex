\subsubsection{SUB-M-CU1.1-Confirmar medición}\label{SUB-M-CU1.1}
Una vez que se termina la carga de combustible, y se calcula la cantidad de combustible que fue cargado, se solicita al Cliente que ingrese la cantidad de litros que ordenó.

\begin{longtable}{|J{5cm}|J{10.3cm}|}
	\hline
	\textbf{Nombre del caso de uso} &
		SUB-M-CU1.1-Confirmar medición \\ \hline
	\textbf{Objetivo} &
		Solicitar al Cliente que confirme la cantidad de gasolina que le debería de haber sido cargada en la gasolinera. \\ \hline
	\textbf{Actores} &
		Cliente \\ \hline 
	\textbf{Disparador} & 
		Se calcula la cantidad de combustible que fue cargado al automóvil en \hyperref[SUB-M-CU1]{SUB-M-CU1-Calcular cantidad de combustible cargado}. \\ \hline 
	\textbf{Entradas} & 
		\begin{itemize}
				\item Cantidad de combustible cargada al automóvil.
				\item Fecha y hora de la carga.
		\end{itemize}\\ \hline 
	\textbf{Salidas} & 
		\begin{itemize}
			\item Cantidad de gasolina que debió ser cargada al automóvil.
		\end{itemize} \\ \hline
	\textbf{Precondiciones} &
		La cantidad de combustible cargada al automóvil debe haber sido calculada.\\ \hline
	\textbf{Postcondiciones} &
		\begin{itemize}
			\item La medición calculada puede ser consultada por el Cliente.
			\item La medición se usa en el algoritmo de clasificación.
		\end{itemize} \\ \hline
	\textbf{Condiciones de término} & 
		\begin{itemize}
			\item El Cliente ingresa la cantidad de combustible que debió ser cargada.
			\item Se realiza persistencia de la información calculada e ingresada.
		\end{itemize} \\ \hline 
	\textbf{Prioridad} & 
		Alta. \\ \hline
	\textbf{Errores} & Ninguno.
		% \begin{itemize}
		% 	\item \label{SUB-M-CU1:Error1} Error 1: .
		% \end{itemize} 
		\\ \hline
	\textbf{Reglas de negocio} & Ninguna.
		% \begin{itemize}
		% 	\item \ref{RN1}.
		% \end{itemize}
		 \\ \hline
	% \caption{}
	%\label{desc:SUB-M-CU1}
\end{longtable}

\paragraph{Trayectoria principal}
	\begin{enumerate}
		\item {[Sistema]} Obtiene la cantidad de gasolina cargada, y la fecha y hora calculados en el caso de uso \hyperref[SUB-M-CU1]{SUB-M-CU1-Calcular cantidad de combustible cargado}.
		\item {[Sistema]} Verifica que el actor se encuentre dentro de la aplicación móvil.\hyperref[SUB-M-CU1.1:TA]{[Trayectoria alternativa A]}
		\item \label{SUB-M-CU1.1:Pantalla} {[Sistema]} Muestra la pantalla \hyperref[fig:sub-m-iu1.1.b]{SUB-M-IU1.1-Confirmar medición (b)}.
		\item {[Actor]} Ingresa la información solicitada por la pantalla.
		\item \label{SUB-M-CU1.1:Boton} {[Actor]} Presiona el botón \textit{Aceptar}.
		\item {[Sistema]} Incluye el caso de uso \hyperref[SUB-M-CU1.1.1]{SUB-M-CU1.1.1-Enviar medición}.
		\item \label{SUB-M-CU1.1:PantallaFinal} {[Sistema]} Muestra la pantalla \hyperref[fig:sub-m-iu1.1.c]{SUB-M-IU1.1-Confirmar medición (c)} con la información obtenida.
		\item {[Actor]} Presiona el botón \textit{Aceptar}.
		\item {[Sistema]} Muestra la pantalla \hyperref[fig:sub-c-iu2]{SUB-C-IU2-Consultar mapa}.
	\end{enumerate}
	Fin del caso de uso.

\paragraph{Trayectoria alternativa A} \label{SUB-M-CU1.1:TA}
	El actor no se encuentra usando la aplicación móvil.
	\begin{enumerate}[label=A\arabic*.]
		\item {[Sistema]} Muestra una notificación al actor como la que se observa en la pantalla \hyperref[fig:sub-m-iu1.1.a]{SUB-M-IU1.1-Confirmar medición (a)}.
		\item {[Actor]} Presiona la notificación.
		\item {[Sistema]} Continúa en el paso \ref{SUB-M-CU1.1:Pantalla} de la Trayectoria Principal.
	\end{enumerate}
	Fin de la trayectoria alternativa.

\paragraph{Puntos de extensión} \label{SUB-M-CU1.1:P}
\begin{enumerate}[label=PE\arabic*.]
	\item Caso de uso \hyperref[SUB-M-CU1.1.3]{SUB-M-CU1.1.2-Registrar precio gasolina} en el paso \ref{SUB-M-CU1.1:Boton} de la Trayectoria principal.
	\item Caso de uso \hyperref[SUB-M-CU1.1.3]{SUB-M-CU1.1.3-Obtener insignia} en el paso \ref{SUB-M-CU1.1:PantallaFinal} de la Trayectoria principal.
	\item Caso de uso \hyperref[SUB-M-CU1.1.4]{SUB-M-CU1.1.4-Asignar insignia a gasolinera} en el paso \ref{SUB-M-CU1.1:Boton} de la Trayectoria principal.
	\item Caso de uso \hyperref[SUB-M-CU1.1.5]{SUB-M-CU1.1.5-Especificar bomba} en el paso \ref{SUB-M-CU1.1:Boton} de la Trayectoria principal.
\end{enumerate}
