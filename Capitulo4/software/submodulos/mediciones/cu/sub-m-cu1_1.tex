%Dudas para revisión:
%   *Condiciones de término
%   *Errores checar si los 3 son correctos
%   *Trayectoria principal
% Agregar regla de negocio de IIC Time Out and Clock Stretching? Si se agrega, incluir la RN en la tabla
\subsubsection{SUB-M-CU1.1-Consultar medición del sensor en tiempo real}\label{SUB-M-CU1.1}
El microcontrolador consulta la medición del sensor MCP39F521 de acuerdo a la Regla de Negocio \ref{RN1} Dichas mediciones son enviadas al microcontrolador por medio del protocolo IIC.
\begin{longtable}{|J{5cm}|J{10.3cm}|}
	\hline
	\textbf{Nombre del caso de uso} &
		SUB-M-CU1.1-Consultar medición del sensor en tiempo real \\ \hline
	\textbf{Objetivo} &
		Enviar las mediciones obtenidas por el sensor hacia el microcontrolador por medio del protocolo IIC, para que éstas puedan ser procesadas por el sistema embebido. \\ \hline
	\textbf{Actores} &
	    \begin{itemize}
		    \item Microcontrolador
		    \item Sensor
		\end{itemize}\\ \hline 
	\textbf{Disparador} & 
		El sensor recibe por medio del protocolo IIC, una trama que contiene la instrucción de lectura solicitada por el microcontrolador, para obtener los valores de la medición realizada por el sensor. \\ \hline 
	\textbf{Entradas} & 
		\begin{itemize}
				\item {[Microcontrolador]} Trama que contiene la instrucción de lectura de medición del sensor.
		\end{itemize}\\ \hline 
	\textbf{Salidas} & 
	    \begin{itemize}
	        \item {[Sensor]} Trama de respuesta de lectura de medición del sensor.
	    \end{itemize}\\ \hline
	\textbf{Precondiciones} & 
		Ninguna \\ \hline
	\textbf{Postcondiciones} &
		\begin{itemize}
			\item La trama de respuesta que contiene la información de la medición obtenida es recibida por el microcontrolador.
		\end{itemize} \\ \hline
	\textbf{Condiciones de término} & 
		\begin{itemize}
		    \item El microcontrolador valida la trama de respuesta enviada por el sensor.
			%DUDA:\item El sensor recibe la trama de acuse de recibido por el microcontrolador.
		\end{itemize} 
		\\ \hline 
	\textbf{Prioridad} & 
		Alta. \\ \hline
	\textbf{Errores} &
		 \begin{itemize}
		 %DUDA: tiempo establecido por I2C Time Out and Clock Stretching?
		 	\item \label{SUB-M-CU1.1:Error1} Error 1: La trama de lectura de medición del sensor no es válida.
		 	\item \label{SUB-M-CU1.1:Error2} Error 2: La trama de respuesta del sensor no fue recibida por el microcontrolador.
		 	\item \label{SUB-M-CU1.1:Error3} Error 3: La trama de respuesta de medición del sensor no es válida.
		 \end{itemize} \\ \hline
	\textbf{Reglas de negocio} & 
	    \begin{itemize}
	      \item  \ref{RN1}
		 \end{itemize}\\ \hline
	% \caption{}
	%\label{desc:SUB-M-CU1}
\end{longtable}

\paragraph{Trayectoria principal}
\label{SUB-M-CU1.1:TP}
	\begin{enumerate}
	    \item {[Microcontrolador]} Envía una trama que contiene la instrucción de lectura de medición del sensor que se desea por medio del protocolo IIC.
	    \item {[Sensor]} Recibe la trama.
	    \item {[Sensor]} Valida que la trama sea correcta. \hyperref[SUB-M-CU1.1:TA]{[Trayectoria Alternativa A]} %Error1
		\item {[Sensor]} Envía la trama de respuesta de medición al microcontrolador por medio del protocolo de comunicación IIC.
		\item {[Microcontrolador]} Recibe la trama de respuesta de lectura de medición que solicitó al sensor.\hyperref[SUB-M-CU1.1:TB]{[Trayectoria Alternativa B]} %Error2
		\item {[Microcontrolador]} Valida que la trama sea correcta por medio del acuse del sensor. \hyperref[SUB-M-CU1.1:TC]{[Trayectoria Alternativa C]}%Error3 
		\item {[Microcontrolador]} Retorna al primer paso de \hyperref[SUB-M-CU1.1:TP]{[Trayectoria Principal]}
	\end{enumerate}
	Fin del caso de uso.

%TA? El sensor recibe una trama de no acuse de recibo por el microcontrolador. \hyperref[SUB-M-CU1.1:TP]{[Trayectoria Principal]}
\paragraph{Trayectoria alternativa A} \label{SUB-M-CU1.1:TA}
	La trama que contiene la instrucción de lectura de medición del sensor no es válida.
	\begin{enumerate}[label=A\arabic*.]
		\item {[Sensor]} Envía una trama para notificar el error al microcontrolador.  
	\end{enumerate}
	Fin de la trayectoria alternativa y retorno al último paso de \hyperref[SUB-M-CU1.1:TP]{[Trayectoria Principal]}  

\paragraph{Trayectoria alternativa B} \label{SUB-M-CU1.1:TB}
	La trama de respuesta de lectura de medición del sensor no fue recibida por el microcontrolador.
	\begin{enumerate}[label=B\arabic*.]
		\item {[Microcontrolador]} Reenvía una trama que contiene la instrucción de lectura de medición del sensor por medio del protocolo IIC.  
	\end{enumerate}
	Fin de la trayectoria alternativa. Regresa al paso 2 de \hyperref[SUB-M-CU1.1:TP]{[Trayectoria Principal]}  
	
\paragraph{Trayectoria alternativa C} \label{SUB-M-CU1.1:TC}
	La trama de respuesta de medición del sensor no es válida.
	\begin{enumerate}[label=C\arabic*.]
		\item {[Sensor]} Genera un código de error.
	\end{enumerate}
	Fin de la trayectoria alternativa y retorno al último paso de \hyperref[SUB-M-CU1.1:TP]{[Trayectoria Principal]} 