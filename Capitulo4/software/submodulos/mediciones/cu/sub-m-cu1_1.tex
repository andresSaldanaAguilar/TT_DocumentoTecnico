\subsubsection{SUB-M-CU1.1-Calcular promedio diario de energía sensada}\label{SUB-M-CU1.1}
El promedio de energía que es sensada diariamente, es calculado tomando en cuenta las especificaciones de la Regla de Negocio \ref{RN5}, este cálculo se realiza diariamente cuando el periodo diario de toma de muestras especificado en la Regla de Negocio \ref{RN6} culmina.

\begin{longtable}{|J{5cm}|J{10.3cm}|}
	\hline
	\textbf{Nombre del caso de uso} &
		SUB-M-CU1.1 Calcular promedio diario de energía sensada \\ \hline
	\textbf{Objetivo} &
		Obtener el promedio diario del total de muestras tomadas durante el día para su posterior almacenamiento. \\ \hline
	\textbf{Actores} &
		Servidor embebido \\ \hline 
	\textbf{Disparador} & 
		La última muestra del día fue tomada y el periodo de toma de muestras especificado en la Regla de Negocio \ref{RN6} ha culminado. \\ \hline 
	\textbf{Entradas} & 
		\begin{itemize}
				\item Muestras almacenadas del sistema fotovoltaico de acuerdo con la Regla de Negocio \ref{RN6} durante el día.
		\end{itemize}\\ \hline 
	\textbf{Salidas} & 
		\begin{itemize}
			\item Cantidad promedio de muestras sensadas diariamente por el sistema fotovoltaico.
			\item Fecha del día actual en la que se realiza el cálculo.
		\end{itemize} \\ \hline
	\textbf{Precondiciones} &
		Ninguna.\\ \hline
	\textbf{Postcondiciones} &
		\begin{itemize}
			\item El promedio calculado junto con la fecha del día en curso son almacenados.
		\end{itemize}\\ \hline
	\textbf{Condiciones de término} & 
		\begin{itemize}
			\item Se obtiene un número flotante que representa el promedio diario de energía que ha sido sensada.
			\item Se obtiene la fecha del día en curso del cual se realiza el promedio.
		\end{itemize} \\ \hline 
	\textbf{Prioridad} & 
		Alta. \\ \hline
	\textbf{Errores} & 
	    Ninguno
		%\begin{itemize}
		%\end{itemize} 
		\\ \hline
	\textbf{Reglas de negocio} & 
		\begin{itemize}
			\item \ref{RN5}
			\item \ref{RN6}
		\end{itemize} \\ \hline

	% \caption{}
	%\label{desc:SUB-M-CU1}
\end{longtable}

\paragraph{Trayectoria principal}
\label{SUB-M-CU1.1:TP}
	\begin{enumerate}
		\item {[Servidor embebido]} Recibe la trama con los datos de la última muestra del día.
        \hyperref[SUB-M-CU1.1:TA]{[Trayectoria Alternativa A]}
		\item {[Servidor embebido]} Realiza el promedio de las 120 muestras.
		\item {[Servidor embebido]} Obtiene la fecha actual
		\item {[Servidor embebido]} Almacena tanto el promedio, como la fecha actual.
	\end{enumerate}
	Fin del caso de uso.

\paragraph{Trayectoria alternativa A} \label{SUB-M-CU1.1:TA}
	Al menos una de las 120 muestras es nula
	\begin{enumerate}[label=A\arabic*.]
		\item {[Servidor embebido]} Realiza el promedio con las 120 muestras, ignorando los valores nulos.
	\end{enumerate}
	Fin de la trayectoria alternativa. Regresa al paso 2 de \hyperref[SUB-M-CU1.1:TP]{[Trayectoria Principal]} .
