\subsubsection{SUB-M-CU1.1-Calcular promedio diario de energía sensada}\label{SUB-M-CU1.1}
El promedio de energía que es sensada diariamente, es calculado tomando en cuenta las especificaciones de la Regla de Negocio \ref{RN5}, este cálculo se realiza diariamente cuando el periodo diario de toma de muestras especificado en la Regla de Negocio \ref{RN6} culmina.
% \begin{description}
% 	\item[Nombre del Caso de Uso] SUB-M-CU1-Calcular cantidad de combustible recargado.
% 	\item[Objetivo] Obtener las señales enviadas por el sensor y a partir de estas, calcular la cantidad de combustible que fue cargado al automóvil.
% 	\item[Actores] Sensor
% 	\item[Disparador] El microcontrolador detecta un cambio de voltaje en la salida del sensor y envía a través del módulo Bluetooth tramas indicando un pulso de voltaje en alta.
% 	\item[Entradas] \hfill
% 		\begin{itemize}
% 			\item Marca de tiempo de inicio de la recarga.
% 			\item Tramas enviadas por el módulo Bluetooth.
% 			\item Marca de tiempo de fin de la recarga.
% 		\end{itemize}
% 	\item[Salidas] \hfill 
% 		\begin{itemize}
% 			\item Cantidad de combustible recargado al automóvil.
% 			\item Fecha y hora en la que se registró la recarga de combustible.
% 		\end{itemize}
% 	\item[Precondiciones] Ninguna.
% 	\item[Postcondiciones] \hfill 
% 		\begin{itemize}
% 			\item El Cliente confirma la cantidad de combustible que le fue recargada.
% 			\item La medición calculada junto con la fecha y hora son almacenados.
% 		\end{itemize}
% 	\item[Condiciones de término] 
% 		\begin{itemize}
% 			\item Se obtiene un número flotante que representa la cantidad de combustible cargado al automóvil.
% 			\item Se obtiene la fecha y hora de la carga.
% 		\end{itemize}
% 	\item[Prioridad] Alta.
% 	\item[Errores] Ninguno.
% 	\item[Reglas de Negocio] \hfill
% 		\begin{itemize}
% 			\item \ref{RN1}.
% 			\item \ref{RN2}.
% 		\end{itemize}
% \end{description}

\begin{longtable}{|J{5cm}|J{10.3cm}|}
	\hline
	\textbf{Nombre del caso de uso} &
		SUB-M-CU1.1Calcular promedio diario de energía sensada \\ \hline
	\textbf{Objetivo} &
		Obtener el promedio diario del total de muestras tomadas durante el día para su posterior almacenamiento. \\ \hline
	\textbf{Actores} &
		Servidor embebido \\ \hline 
	\textbf{Disparador} & 
		La última muestra del día fue tomada y el periodo de toma de muestras especificado en la Regla de Negocio \ref{RN6} ha culminado. \\ \hline 
	\textbf{Entradas} & 
		\begin{itemize}
				\item Muestras almacenadas del sistema fotovoltaico de acuerdo con la Regla de Negocio \ref{RN6} durante el día.
		\end{itemize}\\ \hline 
	\textbf{Salidas} & 
		\begin{itemize}
			\item Cantidad promedio de muestras sensadas diariamente por el sistema fotovoltaico.
			\item Fecha del día actual en la que se realiza el cálculo.
		\end{itemize} \\ \hline
	\textbf{Precondiciones} &
		Ninguna.\\ \hline
	\textbf{Postcondiciones} &
		\begin{itemize}
			\item El promedio calculado junto con la fecha del día en curso son almacenados.
		\end{itemize}\\ \hline
	\textbf{Condiciones de término} & 
		\begin{itemize}
			\item Se obtiene un número flotante que representa el promedio diario de energía que ha sido sensada.
			\item Se obtiene la fecha del día en curso del cual se realiza el promedio.
		\end{itemize} \\ \hline 
	\textbf{Prioridad} & 
		Alta. \\ \hline
	\textbf{Errores} & 
		\begin{itemize}
			\item \label{SUB-M-CU1:Error1} Error 1: No se puede establecer una conexión WiFi inalámbrica entre el módulo del microcontrolador y el servidor embebido.
		\end{itemize} \\ \hline
	\textbf{Reglas de negocio} & 
		\begin{itemize}
			\item \ref{RN5}
			\item \ref{RN6}
		\end{itemize} \\ \hline

	% \caption{}
	%\label{desc:SUB-M-CU1}
\end{longtable}

\paragraph{Trayectoria principal}
	\begin{enumerate}
		\item {[Sistema]} Recibe la trama con los datos de la medición, y al ser la última muestra del día.
		\item  {[Sistema]} Consulta de la base de datos las últimas 119 muestras, ya que son las que obtuvo durante el transcurso del día actual.
		\item {[Sistema]} Habiendo obtenido las 119 muestras de la base de datos y la muestra que conserva actualmente de la última medición, realiza el promedio con las 120 muestras.
		\item {[Sistema]} Obtiene la fecha actual
		\item {[Sistema]} Almacena tanto el promedio, como la fecha actual y un estado de éxito.
		%\item {[Sistema]} Incluye el caso de uso \hyperref[SUB-M-CU1.1]{SUB-M-CU1.1-Confirmar medición}.
	\end{enumerate}
	Fin del caso de uso.

\paragraph{Trayectoria alternativa A} \label{SUB-M-CU1:TA}
	No se puede establecer conexión WiFi inalámbrica entre el módulo del sensor y el servidor embebido.
	\begin{enumerate}[label=A\arabic*.]
		\item {[Sistema]} Toma un 0 como última medición. 
		\item {[Sistema]} Habiendo obtenido las 119 muestras de la base de datos y el 0 como última medición, realiza el promedio con las 120 muestras.
		\item {[Sistema]} El sistema obtiene la fecha actual
		\item {[Sistema]} Almacena tanto el promedio, la fecha actual y un estado de fallo.
	\end{enumerate}
	Fin de la trayectoria alternativa.
