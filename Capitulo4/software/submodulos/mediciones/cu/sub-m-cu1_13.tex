\subsubsection{SUB-U-CU1.1-Ver estado de generación actual}\label{SUB-U-CU1.1}
La barra de estado de generación actual es un componente de la interfaz \hyperref[fig:monitoreo]{[IU1-Generacion Actual]}
y es la encargada de mostrar el estado de generación actual del sistema fotovoltaico, evaluándolo con la Regla de Negocios \ref{RN12}, con actualizaciones en el tiempo establecido la Regla de Negocio \ref{}, la comunicación con el servidor se logrará mediante el patrón de intercambio de mensajes de la Regla de Negocio \ref{RN8} con el formato de datos establecido en la Regla de Negocio \ref{RN11}   

\begin{longtable}{|J{5cm}|J{10.3cm}|}
	\hline
	\textbf{Nombre del caso de uso} &
		SUB-U-CU1.13-Ver Estatus de Generación Actual \\ \hline
	\textbf{Objetivo} &
		Conseguir el valor actual de generación de energía y mostrarlo al usuario. \\ \hline
	\textbf{Actores} &
		\begin{itemize}
		    \item Usuario
			\item Servidor embebido
			\item Aplicación de Usuario
		\end{itemize} \\ \hline
	\textbf{Disparador} & 
	    El usuario inicia la aplicación de usuario, dando clic en el icono correspondiente a la aplicación\\ \hline 
	\textbf{Entradas} & 
		\begin{itemize}
				\item{[Servidor]} Petición por el valor actual
		\end{itemize}\\ \hline 
	\textbf{Salidas} & 
		\begin{itemize}
			\item{[Servidor]} Respuesta con valor generación actual
		\end{itemize} \\ \hline
	\textbf{Precondiciones} &
		La aplicación de usuario debe estar previamente configurada (ver Caso de Uso \ref{}) \\ \hline
	\textbf{Postcondiciones} &
		Ninguna.\\ \hline
	\textbf{Condiciones de término} & 
		\begin{itemize}
			\item La aplicación de usuario recibe el valor y muestra el estatus de generación actual.
		\end{itemize} \\ \hline 
	\textbf{Prioridad} & 
		Media. \\ \hline
	\textbf{Errores} & 
		\begin{itemize}
		    \item \label{CU13:Error1} Error 1: Sin conexión con el servidor.
			\item \label{CU13:Error2} Error 2: La marca de tiempo requerida no existe en el servidor o es nulo su valor asociado.
		    \item \label{CU13:Error3} Error 3: Sin conexión con el cliente.
		    \item \label{CU13:Error4} Error 4: El valor de generación es -1
		\end{itemize} \\ \hline
	\textbf{Reglas de negocio} & 
		\begin{itemize}
		    \item \ref{}
			\item \ref{RN8}
			\item \ref{RN11}
		\end{itemize} \\ \hline
\end{longtable}

\paragraph{Trayectoria principal}
    \label{SUB-U-CU1.13:TP}
	\begin{enumerate}
	    \item {[Usuario]} Inicia la aplicación de usuario dando clic en el icono correspondiente a la aplicación. \hyperref[SUB-U-CU1.13:TA]{[Trayectoria alternativa A]}
	    \item {[Aplicación de Usuario]} Envía la petición para obtener el valor de generación actual. \hyperref[SUB-U-CU1.13:TB]{[Trayectoria alternativa B]} 
	    \item {[Servidor]} Recibe la petición, busca el valor actual de generación almacenado. \hyperref[SUB-U-CU1.13:TC]{[Trayectoria alternativa C]}
	    \item {[Servidor]} Envía el valor correspondiente \hyperref[SUB-U-CU1.13:TD]{[Trayectoria alternativa D]} 
	    \item {[Aplicación de Usuario]} Recibe el valor de generación actual, se muestra la interfaz de usuario con el valor actualizado de estatus de generación actual \hyperref[fig:monitoreo]{[IU1-Generacion Actual]}. \hyperref[SUB-U-CU1.13:TE]{[Trayectoria alternativa E]}
	\end{enumerate}
	Fin del caso de uso.

\paragraph{Trayectoria alternativa A} \label{SUB-U-CU1.13:TA}
	El usuario ya esta dentro de la aplicación de usuario
	\begin{enumerate}[label=A\arabic*.]
		\item {[Aplicación de Usuario]} Transcurre el tiempo marcado por la Regla de Negocio \ref{} y envía la petición para obtener el valor de generación actual. \hyperref[SUB-U-CU1.13:TB]{[Trayectoria alternativa B]} 
		\item {[Aplicación de usuario]} Regresa al punto 3 de la \hyperref[SUB-U-CU1.13:TP]{[Trayectoria Principal]}
	\end{enumerate}
	Fin de la trayectoria alternativa.

\paragraph{Trayectoria alternativa B} \label{SUB-U-CU1.13:TB}
	Sin conexión con el servidor.
	\begin{enumerate}[label=B\arabic*.]
		\item {[Aplicación de usuario]} Actualiza como estatus actual la palabra ''desconectado'' en la interfaz de usuario \hyperref[fig:monitoreo]{[IU1-Generacion Actual]},
		\item {[Aplicación de usuario]} Regresa al final de \hyperref[SUB-U-CU1.13:TP]{[Trayectoria Principal]}
	\end{enumerate}
	Fin de la trayectoria alternativa.

\paragraph{Trayectoria alternativa C} \label{SUB-U-CU1.13:TC}
	La marca de tiempo requerida tiene valor nulo.
	\begin{enumerate}[label=C\arabic*.]
		\item {[Servidor]} Asigna el valor -1 a la respuesta para la aplicación de usuario.
		\item {[Aplicación de usuario]} Regresa al punto 4 de \hyperref[SUB-U-CU1.13:TP]{[Trayectoria Principal]}
	\end{enumerate}
	Fin de la trayectoria alternativa.

\paragraph{Trayectoria alternativa D} \label{SUB-U-CU1.13:TD}
	Sin conexión con el cliente.
	\begin{enumerate}[label=D\arabic*.]
		\item {[Servidor]} Aborta envió y espera por la siguiente petición.
		\item {[Aplicación de usuario]} Al no recibir respuesta del servidor en el intervalo establecido, actualiza como estatus actual la palabra ''desconectado'' en la interfaz de usuario \hyperref[fig:monitoreo]{[IU1-Generacion Actual]},
		\item {[Aplicación de usuario]} Regresa al final de \hyperref[SUB-U-CU1.13:TP]{[Trayectoria Principal]}
	\end{enumerate}
	Fin de la trayectoria alternativa.
	
\paragraph{Trayectoria alternativa E} \label{SUB-U-CU1.13:TE}
	El valor de respuesta es -1
	\begin{enumerate}[label=E\arabic*.]
		\item {[Aplicación de usuario]} Actualiza como estatus actual la palabra ''error servidor'' en la interfaz de usuario \hyperref[fig:monitoreo]{[IU1-Generacion Actual]},
		\item {[Aplicación de usuario]} Regresa al final de \hyperref[SUB-U-CU1.13:TP]{[Trayectoria Principal]}
	\end{enumerate}
	Fin de la trayectoria alternativa.
