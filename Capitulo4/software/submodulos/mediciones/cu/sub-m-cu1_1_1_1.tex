\subsubsection{SUB-M-CU1.1.1.1-Almacenar medición en servidor}\label{SUB-M-CU1.1.1.1}
Al tener una conexión a Internet, el sistema almacena la información obtenida en la base de datos del servidor web.

\begin{longtable}{|J{5cm}|J{10.3cm}|}
	\hline
	\textbf{Nombre del caso de uso} &
		SUB-M-CU1.1.1.1-Almacenar medición en servidor \\ \hline
	\textbf{Objetivo} &
		Permitir la persistencia de la información obtenida. \\ \hline
	\textbf{Actores} &
		Cliente \\ \hline 
	\textbf{Disparador} & 
		Existe conexión a Internet y es necesario almacenar la información. \\ \hline 
	\textbf{Entradas} & 
		\begin{itemize}
				\item Cantidad de combustible cargada al automóvil.
				\item Fecha y hora de la carga.
				\item Cantidad de gasolina ingresada por el usuario
		\end{itemize}\\ \hline 
	\textbf{Salidas} & 
		\begin{itemize}
			\item Mensaje de éxito.
		\end{itemize} \\ \hline
	\textbf{Precondiciones} &
		Debe existir conexión a Internet.\\ \hline
	\textbf{Postcondiciones} &
		\begin{itemize}
			\item La medición calculada puede ser consultada por el Cliente.
			\item La medición se usa en el algoritmo de clasificación.
		\end{itemize} \\ \hline
	\textbf{Condiciones de término} & 
		\begin{itemize}
			\item Se realiza persistencia de la información calculada e ingresada.
		\end{itemize} \\ \hline 
	\textbf{Prioridad} & 
		Alta. \\ \hline
	\textbf{Errores} & Ninguno.
		% \begin{itemize}
		% 	\item \label{SUB-M-CU1:Error1} Error 1: .
		% \end{itemize} 
		\\ \hline
	\textbf{Reglas de negocio} & Ninguna.
		% \begin{itemize}
		% 	\item \ref{RN1}.
		% \end{itemize}
		 \\ \hline
	% \caption{}
	%\label{desc:SUB-M-CU1}
\end{longtable}

\paragraph{Trayectoria principal}
	\begin{enumerate}
		\item {[Sistema]} Obtiene la información que se almacenará en la base datos.
		\item {[Sistema]} Almacena la información en el servidor web.
	\end{enumerate}
	Fin del caso de uso.
