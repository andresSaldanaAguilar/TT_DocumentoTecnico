%ESTE LO TENGO QUE MODIFICAR
\subsubsection{SUB-M-CU1.2-Calcular promedio bimestral de energía sensada }\label{SUB-M-CU1.2}
El  promedio  de  energía  bimestral es calculado  tomando en cuenta las especificaciones de la Regla de Negocio \ref{RN4} para el cálculo bimestral, este cálculo se realiza con los promedios diarios calculados en los días que abarca el periodo especificado, cuando alguno de estos
periodos culmina y el promedio de energía sensada diariamente del día en curso ha sido calculado como se especifica en el caso de uso \hyperref[SUB-M-CU1.1]{SUB-M-CU1.1}.


\begin{longtable}{|J{5cm}|J{10.3cm}|}
	\hline
	\textbf{Nombre del caso de uso} &
		SUB-M-CU1.2 - Calcular promedio bimestral de energía sensada
 \\ \hline
	\textbf{Objetivo} &
		Obtener la media bimestral de los promedios  diarios de muestras tomadas durante los periodos especificados en la regla de negocios  \ref{RN4} para el cálculo bimestral, y posteriormente ser almacenado.
 \\ \hline
	\textbf{Actores} &
		Servidor embebido \\ \hline 
	\textbf{Disparador} & 
		La ́ultima muestra del día fue tomada, el periodo de toma de muestras especificado en la Regla de Negocio \ref{RN6} ha culminado y el día en curso es el día que marcan el final de uno de los periodos especificados en la regla de negocios \ref{RN4} para el cálculo bimestral. \\ \hline 
	\textbf{Entradas} & 
		\begin{itemize}
				\item Promedios diarios comprendido dentro del periodo que se va a calcular.
		\end{itemize}\\ \hline 
	\textbf{Salidas} & 
		\begin{itemize}
			\item Promedio bimestral de muestras sensadas por el sistema fotovoltaico.
			\item Fecha del día actual en la que se realiza el cálculo.
		\end{itemize} \\ \hline
	\textbf{Precondiciones} &
		Ninguna.\\ \hline
	\textbf{Postcondiciones} &
		\begin{itemize}
			\item El promedio calculado junto con la fecha del día en curso son almacenados.
		\end{itemize}\\ \hline
	\textbf{Condiciones de término} & 
		\begin{itemize}
			\item Se obtiene un número flotante que representa el promedio bimestral de energía que ha sido sensada.
			\item Se obtiene la fecha del día en curso del cual se realiza el promedio.
		\end{itemize} \\ \hline 
	\textbf{Prioridad} & 
		Alta. \\ \hline
	\textbf{Errores} & 
		Ninguno \\ \hline
	\textbf{Reglas de negocio} & 
		\begin{itemize}
			\item \ref{RN4}
			\item \ref{RN6}
		\end{itemize} \\ \hline

	% \caption{}
	%\label{desc:SUB-M-CU1}
\end{longtable}

\paragraph{Trayectoria principal}
	\begin{enumerate}
		\item {[Sensor]} El promedio diario del último día de uno de los periodos especificado en la regla de negocios \ref{RN4} para el cálculo bimestral es calculado como lo indica el caso de uso \hyperref[SUB-M-CU1.1]{SUB-M-CU1.1} y es almacenado.
		\item  {[Servidor embebido]} Consulta de la base de datos los promedios diarios de los días que abarca el periodo en curso, por ejemplo, si el día actual es 30 de Abril, se recuperarán los promedios diarios del 1 de Marzo al 30 de Abril.
		\item {[Servidor embebido]} Realiza el promedio con los promedios diarios consultados.
		\item {[Servidor embebido]} Obtiene la fecha actual
		\item {[Servidor embebido]} Almacena tanto el promedio como la fecha actual.
	\end{enumerate}
	Fin del caso de uso.
