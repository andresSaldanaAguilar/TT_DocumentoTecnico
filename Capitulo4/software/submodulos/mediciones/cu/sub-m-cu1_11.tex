\subsubsection{SUB-M-CU1.11-Almacenar Medición en el servidor}\label{SUB-M-CU1.11}
El servidor añadirá al archivo de buffer circular que almacena las muestras diarias el dato de la trama valida que tiene los valores obtenidos por el sensor. 

\begin{longtable}{|J{5cm}|J{10.3cm}|}
	\hline
	\textbf{Nombre del caso de uso} &
		SUB-M-CU1.11-Almacenar Medición en el servidor \\ \hline
	\textbf{Objetivo} &
		Almacenar los datos entrantes del modulo Wi-Fi al servidor. \\ \hline
	\textbf{Actores} &
		\begin{itemize}
			\item Servidor embebido
		\end{itemize} \\ \hline
	\textbf{Disparador} & 
	    Termina el caso de uso \hyperref[SUB-M-CU1.4]{SUB-M-CU1.4} \\ \hline 
	\textbf{Entradas} & 
		\begin{itemize}
				\item{[Servidor]} Muestra Valida
		\end{itemize}\\ \hline 
	\textbf{Salidas} & 
        Ninguna. \\ \hline
	\textbf{Precondiciones} &
		Ninguna. \\ \hline
	\textbf{Postcondiciones} &
		Ninguna.\\ \hline
	\textbf{Condiciones de término} & 
		\begin{itemize}
			\item El servidor almacena el valor en el archivo de muestras diarias.
		\end{itemize} \\ \hline 
	\textbf{Prioridad} & 
		Alta. \\ \hline
	\textbf{Errores} & 
		\begin{itemize}
		    \item \label{CU5:Error1} Error 1: 
		\end{itemize} \\ \hline
	\textbf{Reglas de negocio} & 
		\begin{itemize}
		    \item \ref{RN6}
			\item \ref{RN8}
			\item \ref{RN11}
		\end{itemize} \\ \hline

	% \caption{}
	%\label{desc:SUB-M-CU1}
\end{longtable}

\paragraph{Trayectoria principal}
    \label{SUB-M-CU1.5:TP}
	\begin{enumerate}
	    \item {[Servidor]} Actualiza el archivo de muestras diarias con el valor de la muestra.
	\end{enumerate}
	Fin del caso de uso.
