%PREGUNTAR A VH

\subsubsection{SUB-M-CU1.3-Enviar mediciones del sensor al microcontrolador en tiempo real}\label{SUB-M-CU1.3}
El dispositivo de monitoreo MCP39F521 se encarga de obtener las mediciones en tiempo real del sistema fotovoltaico; dichas mediciones son enviadas al microcontrolador por medio del protocolo IIC.
\begin{longtable}{|J{5cm}|J{10.3cm}|}
	\hline
	\textbf{Nombre del caso de uso} &
		SUB-M-CU1.3-Enviar mediciones del sensor al microcontrolador en tiempo real \\ \hline
	\textbf{Objetivo} &
		Enviar las mediciones obtenidas por el sensor hacia el microcontrolador por medio del protocolo IIC, para que éstas puedan ser procesadas por el Sistema embebido. \\ \hline
	\textbf{Actores} &
		Sensor \\ \hline 
	\textbf{Disparador} & 
		El sensor obtiene una lectura de medición del sistema fotovoltaico. \\ \hline 
	\textbf{Entradas} & 
		\begin{itemize}
				\item Medición obtenida por el sensor
		\end{itemize}\\ \hline 
	\textbf{Salidas} & Trama generada por el protocolo IIC.
		% \begin{itemize}
		% 	\item Cantidad de gasolina que debió ser cargada al automóvil.
		% \end{itemize} 
		\\ \hline
	\textbf{Precondiciones} & 
		Existe una medición del Sistema fotovoltaico.\\ \hline
	\textbf{Postcondiciones} &
		\begin{itemize}
			\item La trama que contiene la información de la medición obtenida es recibida por el microcontrolador.
		\end{itemize} \\ \hline
	\textbf{Condiciones de término} & 
		\begin{itemize}
			\item El sensor recibe la trama de acuse de recibido por el microcontrolador.
		\end{itemize} 
		\\ \hline 
	\textbf{Prioridad} & 
		Alta. \\ \hline
	\textbf{Errores} & Ninguno.
		 %\begin{itemize}
		 %	\item \label{SUB-M-CU1.3:Error1} Error 1: La información de la medición no pudo ser enviada al microcontrolador.
		 %\end{itemize} 
		\\ \hline
		% \begin{itemize}
	\textbf{Reglas de negocio} & \ref{RN1}
		% 	\item \ref{RN1}.
		% \end{itemize}
		 \\ \hline
	% \caption{}
	%\label{desc:SUB-M-CU1}
\end{longtable}

\paragraph{Trayectoria principal}
\label{SUB-M-CU1.3:TP}
	\begin{enumerate}
		\item {[Sensor]} Recibe como entrada el voltaje y la corriente que proviene del Sistema fotovoltaico.
		\item {[Sensor]} Realiza una conversión Analógica a Digital de la información recibida. 
		\item {[Sensor]} Envía la trama que contiene la información de medición al microcontrolador por medio del protocolo de comunicación IIC.
		\item {[Sensor]} Recibe trama de acuse de recibo por el microcontrolador. \hyperref[SUB-M-CU1.3:TA]{[Trayectoria Alternativa A]}
	\end{enumerate}
	Fin del caso de uso.

\paragraph{Trayectoria alternativa A} \label{SUB-M-CU1.3:TA}
	El sensor recibe una trama de no acuse de recibo por el microcontrolador.
	\begin{enumerate}[label=A\arabic*.]
		\item {[Sensor]} Regresa al punto 3 de \hyperref[SUB-M-CU1.3:TP]{[Trayectoria Principal]}  
	\end{enumerate}
	Fin de la trayectoria.
