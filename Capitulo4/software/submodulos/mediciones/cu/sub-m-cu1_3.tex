%Dudas para revisión:
%   *Condiciones de término
%   *Errores checar si los 3 son correctos
%   *Trayectoria principal
% Agregar regla de negocio de IIC Time Out and Clock Stretching? Si se agrega, incluir la RN en la tabla
\subsubsection{SUB-M-CU1.3-Enviar mediciones del sensor al microcontrolador en tiempo real}\label{SUB-M-CU1.3}
El sensor MCP39F521 se encarga de obtener las mediciones en tiempo real del sistema fotovoltaico; dichas mediciones son enviadas al microcontrolador por medio del protocolo IIC.
\begin{longtable}{|J{5cm}|J{10.3cm}|}
	\hline
	\textbf{Nombre del caso de uso} &
		SUB-M-CU1.3-Enviar mediciones del sensor al microcontrolador en tiempo real \\ \hline
	\textbf{Objetivo} &
		Enviar las mediciones obtenidas por el sensor hacia el microcontrolador por medio del protocolo IIC, para que éstas puedan ser procesadas por el Sistema embebido. \\ \hline
	\textbf{Actores} &
	    \begin{itemize}
		    \item Microcontrolador
		    \item Sensor
		\end{itemize}\\ \hline 
	\textbf{Disparador} & 
		El sensor recibe por medio del protocolo IIC, una trama que contiene la instrucción de lectura solicitada por el microcontrolador, para obtener los valores de la medición realizada por el sensor. \\ \hline 
	\textbf{Entradas} & 
		\begin{itemize}
				\item Trama que contiene la instrucción de lectura de medición del sensor.
		\end{itemize}\\ \hline 
	\textbf{Salidas} & 
	    \begin{itemize}
	        \item Trama de respuesta de lectura de medición del sensor.
	    \end{itemize}\\ \hline
	\textbf{Precondiciones} & 
		\begin{itemize}
		    \item El sensor obtiene una medición del Sistema fotovoltaico.
		\end{itemize}\\ \hline
	\textbf{Postcondiciones} &
		\begin{itemize}
			\item La trama de respuesta que contiene la información de la medición obtenida es recibida por el microcontrolador.
		\end{itemize} \\ \hline
	\textbf{Condiciones de término} & 
		\begin{itemize}
		    \item El microcontrolador valida que sea correcto el checksum de la trama de respuesta de lectura de medición enviada por el sensor.
			%DUDA:\item El sensor recibe la trama de acuse de recibido por el microcontrolador.
		\end{itemize} 
		\\ \hline 
	\textbf{Prioridad} & 
		Alta. \\ \hline
	\textbf{Errores} &
		 \begin{itemize}
		 %DUDA: tiempo establecido por I2C Time Out and Clock Stretching?
		 	\item \label{SUB-M-CU1.3:Error1} Error 1: El checksum de la trama que contiene la instrucción de lectura de medición del sensor no es válido.
		 	\item \label{SUB-M-CU1.3:Error2} Error 2: La trama de respuesta de lectura de medición del sensor no fue recibida por el microcontrolador en el tiempo establecido.
		 	\item \label{SUB-M-CU1.3:Error3} Error 3: El checksum de la trama de respuesta de lectura de medición del sensor no es válido.
		 \end{itemize} \\ \hline
	\textbf{Reglas de negocio} & 
	    \begin{itemize}
	      \item  \ref{RN1}
		 \end{itemize}\\ \hline
	% \caption{}
	%\label{desc:SUB-M-CU1}
\end{longtable}

\paragraph{Trayectoria principal}
\label{SUB-M-CU1.3:TP}
	\begin{enumerate}
	    \item {[Microcontrolador]} Envía una trama que contiene la instrucción de lectura de medición del sensor por medio del protocolo IIC.
	    \item {[Sensor]} Recibe la trama.
	    \item {[Sensor]} Valida que el checksum contenido en la trama sea correcto. \hyperref[SUB-M-CU1.3:TA]{[Trayectoria Alternativa A]} %Error1
		\item {[Sensor]} Realiza una conversión Analógica a Digital de la medición del Sistema fotovoltaico. 
		\item {[Sensor]} Envía la trama de respuesta de lectura de medición al microcontrolador por medio del protocolo de comunicación IIC.
		\item {[Microcontrolador]} Recibe la trama de respuesta de lectura de medición que solicitó al sensor.\hyperref[SUB-M-CU1.3:TB]{[Trayectoria Alternativa B]} %Error2
		\item {[Microcontrolador]} Valida que el checksum contenido en la trama sea correcto. \hyperref[SUB-M-CU1.3:TC]{[Trayectoria Alternativa C]}%Error3 
		%DUDA: \item {[Sensor]} Recibe trama de acuse de recibo por el microcontrolador. \hyperref[SUB-M-CU1.3:TA]{[Trayectoria Alternativa A]}
	\end{enumerate}
	Fin del caso de uso.

%TA? El sensor recibe una trama de no acuse de recibo por el microcontrolador. \hyperref[SUB-M-CU1.3:TP]{[Trayectoria Principal]}
\paragraph{Trayectoria alternativa A} \label{SUB-M-CU1.3:TA}
	El checksum de la trama que contiene la instrucción de lectura de medición del sensor no es válido.
	\begin{enumerate}[label=A\arabic*.]
		\item {[Sensor]} Envía una trama de respuesta con el error CSFAIL de 0x51  
	\end{enumerate}
	Fin del caso de uso.

\paragraph{Trayectoria alternativa B} \label{SUB-M-CU1.3:TB}
	La trama de respuesta de lectura de medición del sensor no fue recibida por el microcontrolador en el tiempo establecido.
	\begin{enumerate}[label=B\arabic*.]
		\item {[Microcontrolador]} Reenvía una trama que contiene la instrucción de lectura de medición del sensor por medio del protocolo IIC.  
	\end{enumerate}
	Fin de la trayectoria alternativa. Regresa al paso 2 de \hyperref[SUB-M-CU1.3:TP]{[Trayectoria Principal]}  
	
\paragraph{Trayectoria alternativa C} \label{SUB-M-CU1.3:TC}
	El checksum de la trama de respuesta de lectura de medición del sensor no es válido.
	\begin{enumerate}[label=C\arabic*.]
		\item {[Microcontrolador]} Reenvía una trama que contiene la instrucción de lectura de medición del sensor por medio del protocolo IIC.  
	\end{enumerate}
	Fin de la trayectoria alternativa. Regresa al paso 2 de \hyperref[SUB-M-CU1.3:TP]{[Trayectoria Principal]} 