\subsubsection{SUB-M-CU1.4-Respuesta a petición de cliente}\label{SUB-M-CU1.4}
La respuesta a cada petición del cliente será con los estándares especificados en las Reglas de Negocio \ref{RN7} y \ref{RN11} en el formato de la Regla de Negocio \ref{RN10} bajo el patrón de comunicación de la Regla de Negocio \ref{RN8}

\begin{longtable}{|J{5cm}|J{10.3cm}|}
	\hline
	\textbf{Nombre del caso de uso} &
		SUB-M-CU1.4 Respuesta a petición de cliente\\ \hline
	\textbf{Objetivo} &
		Dar respuesta a las peticiones del o los clientes, ya sea brindándole la información del servidor o los datos almacenados que requiera. \\ \hline
	\textbf{Actores} &
		\begin{itemize}
		    \item Servidor embebido
		\end{itemize}\\ \hline 
	\textbf{Disparador} & 
	     {[Servidor embebido]} Recibe una petición\\ \hline 
	\textbf{Entradas} & 
		\begin{itemize}
				\item Petición proveniente de la aplicación de usuario.
		\end{itemize}\\ \hline 
	\textbf{Salidas} & 
		\begin{itemize}
			\item {[Servidor embebido]} Respuesta a la petición de la aplicación de usuario.
		\end{itemize} \\ \hline
	\textbf{Precondiciones} &
		Ninguna.\\ \hline
	\textbf{Postcondiciones} &
		\begin{itemize}
			Ninguna.
		\end{itemize}\\ \hline
	\textbf{Condiciones de término} & 
		\begin{itemize}
			\item {[Servidor embebido]} Envío de respuesta a la petición de la aplicación de usuario.
		\end{itemize} \\ \hline 
	\textbf{Prioridad} & 
		Alta. \\ \hline
	\textbf{Errores} & 
	    \item {[Servidor embebido]} Falla la conexión para respuesta a la Aplicación de Usuario
		\\ \hline
	\textbf{Reglas de negocio} & 
		\begin{itemize}
			\item \ref{RN7}
			\item \ref{RN8}
			\item \ref{RN10}
			\item \ref{RN11}
		\end{itemize} \\ \hline

\end{longtable}

\paragraph{Trayectoria principal}
\label{SUB-M-CU1.4:TP}
	\begin{enumerate}
		\item {[Servidor embebido]} Crea una nueva conexión con un cliente y recibe una petición proveniente de la aplicación de Usuario
		
		\item {[Servidor embebido]} Recibe la cadena 'get data'
		\hyperref[SUB-M-CU1.4:TA]{[Trayectoria Alternativa A]}
		\hyperref[SUB-M-CU1.4:TB]{[Trayectoria Alternativa B]}
		\hyperref[SUB-M-CU1.4:TC]{[Trayectoria Alternativa C]}
		\hyperref[SUB-M-CU1.4:TD]{[Trayectoria Alternativa D]}
		\hyperref[SUB-M-CU1.4:TE]{[Trayectoria Alternativa E]}
		
		\item {[Servidor embebido]} Desarma la petición consiguiendo la dirección IP del servidor, el número de serie del microcontrolador, el número de esclavo y que archivos se van a acceder.
		
		\item {[Servidor embebido]} Envía la respuesta de los datos en formato JSON codificado como un arreglo de bytes con la información extraída del archivo o archivos \hyperref[SUB-M-CU1.4:TF]{[Trayectoria Alternativa F]}.
		
		\item {[Servidor embebido]} Cierra la conexión existente con el cliente.
	
	\end{enumerate}
	Fin del caso de uso.

\paragraph{Trayectoria alternativa A} \label{SUB-M-CU1.4:TA}
	Recibe la cadena 'get info'.
	\begin{enumerate}[label=A\arabic*.]
		\item {[Servidor embebido]} Envía la información de los microcontroladores y sus nodos de cada uno, existentes en sus registros.
		\hyperref[SUB-M-CU1.4:TA]{[Trayectoria Alternativa A]}.
	\end{enumerate}
	Fin de la trayectoria alternativa, retorna al paso 4 de  \hyperref[SUB-M-CU1.4:TP]{[Trayectoria Principal]}.
	
\paragraph{Trayectoria alternativa B} \label{SUB-M-CU1.4:TB}
	Recibe la cadena 'connect'.
	\begin{enumerate}[label=B\arabic*.]
		\item {[Servidor embebido]} Envía 'get password' si no hay una contraseña guardada en el servidor, de lo contrario envía 'set password'.
		\hyperref[SUB-M-CU1.4:TB]{[Trayectoria Alternativa B]}.
	\end{enumerate}
	Fin de la trayectoria alternativa, retorna al paso 4 de  \hyperref[SUB-M-CU1.4:TP]{[Trayectoria Principal]}.
	
\paragraph{Trayectoria alternativa C} \label{SUB-M-CU1.4:TC}
	Recibe la cadena 'set password'.
	\begin{enumerate}[label=C\arabic*.]
		\item {[Servidor embebido]} Guarda la contraseña y correo que se le envían en la petición, retorna un 'ok'; si ocurre un problema retorna un 'error'.
	\end{enumerate}
	Fin de la trayectoria alternativa, retorna al paso 4 de  \hyperref[SUB-M-CU1.4:TP]{[Trayectoria Principal]}.
	
\paragraph{Trayectoria alternativa D} \label{SUB-M-CU1.4:TD}
	Recibe la cadena 'get password'.
	\begin{enumerate}[label=D\arabic*.]
		\item {[Servidor embebido]} Verifica la contraseña en el servidor, mandando un 'ok' si es correcta y 'error' en caso contrario.
		\hyperref[SUB-M-CU1.4:TD]{[Trayectoria Alternativa D]}.
	\end{enumerate}
	Fin de la trayectoria alternativa, retorna al paso 4 de  \hyperref[SUB-M-CU1.4:TP]{[Trayectoria Principal]}.
	
\paragraph{Trayectoria alternativa E} \label{SUB-M-CU1.4:TE}
	Recibe la cadena 'forgot password'.
	\begin{enumerate}[label=E\arabic*.]
		\item {[Servidor embebido]} Envía un mensaje al correo configurado para recordarle la contraseña para emparejar al servidor.
		\hyperref[SUB-M-CU1.4:TE]{[Trayectoria Alternativa E]}.
	\end{enumerate}
	Fin de la trayectoria alternativa, retorna al paso 4 de  \hyperref[SUB-M-CU1.4:TP]{[Trayectoria Principal]}.
	
\paragraph{Trayectoria alternativa F} \label{SUB-M-CU1.4:TF}
	Falla la conexión para respuesta a la Aplicación de Usuario
	\begin{enumerate}[label=F\arabic*.]
		\item {[Servidor embebido]} Cierra la conexión existente con el cliente.
	\end{enumerate}
	Fin de la trayectoria alternativa y fin de caso de uso.
