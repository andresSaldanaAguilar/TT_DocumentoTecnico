%Dudas:
%   *Condición de término
%   *Trayectoria Principal
%   *Checar los errores
\subsubsection{SUB-M-CU1.4-Enviar muestras al Servidor embebido}\label{SUB-M-CU1.4}
El microcontrolador DSPIC30F4013 envía las muestras obtenidas del Sistema fotovoltaico a través del módulo WiFi como se especifica en la Regla de Negocio \ref{RN2} y \ref{RN7} al Servidor embebido por medio de el patrón mencionado en la Regla de Negocio \ref{RN8}.
\begin{longtable}{|J{5cm}|J{10.3cm}|}
	\hline
	\textbf{Nombre del caso de uso} &
		SUB-M-CU1.4-Enviar muestras al Servidor embebido \\ \hline
	\textbf{Objetivo} &
		Enviar las muestras obtenidas por el microcontrolador al Servidor embebido para su posterior procesamiento. \\ \hline
	\textbf{Actores} &
	    \begin{itemize}
		    \item Microcontrolador
		    \item Módulo WiFi Microcontrolador
		    \item Servidor embebido
		\end{itemize} \hline 
	\textbf{Disparador} & 
		Conexión establecida entre sensor y microcontrolador.\\ \hline
	\textbf{Entradas} & %Ninguna
		\begin{itemize}%especificar
				\item Muestra obtenida del Sistema fotovoltaico.
		\end{itemize}
		\\ \hline 
	\textbf{Salidas} & 
	    \begin{itemize}%Paquete TCP
	        \item {[Microcontrolador]} Arreglo de bytes.
	    \end{itemize}\\ \hline
	\textbf{Precondiciones} & Ninguna
		\\ \hline
	\textbf{Postcondiciones} &
		\begin{itemize}
			\item El arreglo de bytes es recibido por el Servidor embebido.
		\end{itemize} \\ \hline
	\textbf{Condiciones de término} & 
		\begin{itemize}
		    \item El microcontrolador envió el arreglo de Bytes por el módulo Wi-Fi al Servidor embebido.
		\end{itemize} 
		\\ \hline 
	\textbf{Prioridad} & 
		Alta. \\ \hline
	\textbf{Errores} &
        \item \label{SUB-M-CU1.4:Error1} Error 1: Error en envío de trama, conexión con servidor no disponible o cerrada.
		 \\ \hline
	\textbf{Reglas de negocio} & 
	    \begin{itemize}
	      \item  \ref{RN2}
	      \item  \ref{RN7}
	      \item  \ref{RN8}
		 \end{itemize}%\\ \hline
		 \hline
	% \caption{}
	%\label{desc:SUB-M-CU1}
\end{longtable}

\paragraph{Trayectoria principal}
\label{SUB-M-CU1.4:TP}
	\begin{enumerate}
    	\item {[Microcontrolador]} Configura el módulo Wi-Fi para el envío de tramas a la IP y puerto especificado por medio de comandos AT.
	    \item {[Microcontrolador]} Envía la muestra que contiene el numero de serie del microcontrolador, numero de esclavo y medición de potencia por medio de la interfaz UART al módulo WiFi del microcontrolador.
	    \item {[Módulo WiFi Microcontrolador]} Recibe la muestra de medición por medio de la interfaz UART.
	    \item {[Módulo WiFi Microcontrolador]} Envía la trama con la información de la muestra a través del protocolo UDP al módulo WiFi del Servidor embebido. \hyperref[SUB-M-CU1.4:TA]{[Trayectoria Alternativa A]}
	    \item {[Servidor Embebido]} Recibe la trama con la información de la muestra.
	    \item {[Microcontrolador] Retorna al primer paso de  \hyperref[SUB-M-CU1.3:TP]{[Trayectoria Principal]}.
	\end{enumerate}
	Fin del caso de uso.

\paragraph{Trayectoria alternativa A} \label{SUB-M-CU1.3:TA}
	Error en envío de trama al servidor embebido, conexión con servidor no disponible o cerrada.
	\begin{enumerate}[label=A\arabic*.]
		\item {[Microcontrolador]} Salta al ultimo paso de la trayectoria principal	\end{enumerate}
	Fin de trayectoria alternativa.