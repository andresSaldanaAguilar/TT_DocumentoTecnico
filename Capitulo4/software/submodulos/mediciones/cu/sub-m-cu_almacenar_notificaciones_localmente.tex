%ESTE LO TENGO QUE MODIFICAR
\subsubsection{SUB-M-CU1.1-Calcular promedio de energía sensada bimestralmente}\label{SUB-M-CU1.1}
El  promedio  de  energíıa  que  es  sensada  bimestralmente  es  calculado  tomando  en  cuenta  las especificaciones de la Regla de Negocio \ref{RN4}, este cálculo se realiza bimestralmente con los promedios diarios calculados en los días que abarca el periodo especificado, cuando alguno de estos
periodos culmina y el promedio de energía sensada diariamente del día en curso ha sido calculado como se especifica en el caso de uso Calcular promedio de energíıa sensada diariamente.


\begin{longtable}{|J{5cm}|J{10.3cm}|}
	\hline
	\textbf{Nombre del caso de uso} &
		SUB-M-CU1 - Calcular promedio de energía sensada bimestralmente
 \\ \hline
	\textbf{Objetivo} &
		Obtener  el  promedio  bimestral  de  los  promedios  diarios  de muestras tomadas durante los periodos especificados en la regla de negocios  \ref{RN4} para su posterior almacenamiento.
 \\ \hline
	\textbf{Actores} &
		Servidor embebido \\ \hline 
	\textbf{Disparador} & 
		La  ́ultima muestra del día fue tomada, el periodo de toma de muestras especificado en la Regla de Negocio \ref{RN6} ha culminado y el día en curso es el día que marcan el final de uno de los periodos especificados en la regla de negocios \ref{RN4}. \\ \hline 
	\textbf{Entradas} & 
		\begin{itemize}
				\item Promedios diarios de muestras de cada día comprendido dentro del periodo que se vaya a calcular.
		\end{itemize}\\ \hline 
	\textbf{Salidas} & 
		\begin{itemize}
			\item Cantidad promedio de muestras sensadas bimestralmente por el sistema fotovoltaico.
			\item Fecha del día actual en la que se realizá el cálculo.
		\end{itemize} \\ \hline
	\textbf{Precondiciones} &
		Ninguna.\\ \hline
	\textbf{Postcondiciones} &
		\begin{itemize}
			\item El promedio calculado junto con la fecha del día en curso son almacenados.
		\end{itemize}\\ \hline
	\textbf{Condiciones de término} & 
		\begin{itemize}
			\item Se obtiene un número flotante que representa el promedio diario de energía que ha sido sensada.
			\item Se obtiene la fecha del día en curso del cual se realiza el promedio.
		\end{itemize} \\ \hline 
	\textbf{Prioridad} & 
		Alta. \\ \hline
	\textbf{Errores} & 
		\begin{itemize}
			\item \label{SUB-M-CU1:Error1} Error 1: No se puede establecer una conexión WiFi inalámbrica entre el módulo del sensor y el servidor embebido, termina el caso de uso.
		\end{itemize} \\ \hline
	\textbf{Reglas de negocio} & 
		\begin{itemize}
		    \item \ref{RN1}
			\item \ref{RN4}
			\item \ref{RN6}
		\end{itemize} \\ \hline

	% \caption{}
	%\label{desc:SUB-M-CU1}
\end{longtable}

\paragraph{Trayectoria principal}
	\begin{enumerate}
		\item {[Sensor]} Hace una lectura del voltaje y la corriente a la última hora que especifica la regla de negocio \ref{RN6} y envía una trama indicando las cantidades medidas.\hyperref[SUB-M-CU1:Error1]{Error 1}
		\item {[Sistema]} Recibe la trama con los datos de la medición, y al ser la última muestra del día, inicia el proceso para calcular el promedio diario de energía sensada diariamente.
		\item  {[Sistema]} Consulta de la base de datos las últimas 119 muestras, ya que son las que obtuvo durante el transcurso del día actual.
		\item {[Sistema]} Habiendo obtenido las 119 muestras de la base de datos y la muestra que conserva actualmente de la última medición, realiza el promedio con las 120 muestras.
		\item {[Sistema]} El sistema obtiene la fecha actual
		\item {[Sistema]} Almacena tanto el promedio, como la fecha actual y un estado de éxito.
		%\item {[Sistema]} Incluye el caso de uso \hyperref[SUB-M-CU1.1]{SUB-M-CU1.1-Confirmar medición}.
	\end{enumerate}
	Fin del caso de uso.

\paragraph{Trayectoria alternativa A} \label{SUB-M-CU1:TA}
	No se puede establecer conexión WiFi inalámbrica entre el módulo del sensor y el servidor embebido.
	\begin{enumerate}[label=A\arabic*.]
		\item {[Sistema]} Toma un 0 como última medición. 
		\item {[Sistema]} Habiendo obtenido las 119 muestras de la base de datos y el 0 como última medición, realiza el promedio con las 120 muestras.
		\item {[Sistema]} El sistema obtiene la fecha actual
		\item {[Sistema]} Almacena tanto el promedio, la fecha actual y un estado de fallo.
	\end{enumerate}
	Fin de la trayectoria alternativa.
