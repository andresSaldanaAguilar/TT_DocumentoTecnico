%Dudas para revisión:
%   *Condiciones de término
%   *Errores checar si los 3 son correctos
%   *Trayectoria principal
% Agregar regla de negocio de IIC Time Out and Clock Stretching? Si se agrega, incluir la RN en la tabla
\subsubsection{SUB-M-CU1.5-Crear muestras para gráficas del día actual}\label{SUB-M-CU1.5}
El servidor consultará el archivo de tiempo real en el tiempo definido en la regla de negocio \ref{RN6}, con el propósito de crear una gráfica que represente la producción de energía a lo largo de un día.
\begin{longtable}{|J{5cm}|J{10.3cm}|}
	\hline
	\textbf{Nombre del caso de uso} &
		SUB-M-CU1.5-Crear gráficas de tiempo real \\ \hline
	\textbf{Objetivo} &
		Guardar las muestras en el tiempo establecido en la regla de negocio \ref{RN6} en un archivo para generar una gráfica de energía en el tiempo. \\ \hline
	\textbf{Actores} &
	    \begin{itemize}
		    \item Servidor Embebido
		\end{itemize}\\ \hline 
	\textbf{Disparador} & 
		Transcurre el tiempo indicado en la regla de negocio \ref{RN6} \\ \hline 
	\textbf{Entradas} & 
		\begin{itemize}
				\item {[Servidor Embebido]} Muestra del archivo de tiempo real
		\end{itemize}\\ \hline 
	\textbf{Salidas} & 
	    \begin{itemize}
	        \item {[Sensor]} Archivo de muestras del día actualizado
	    \end{itemize}\\ \hline
	\textbf{Precondiciones} & 
		Ninguna \\ \hline
	\textbf{Postcondiciones} &
		Ninguna \\ \hline
	\textbf{Condiciones de término} & 
		\begin{itemize}
		    \item {[Servidor Embebido]} Almacenamiento de la muestra obtenida del archivo de tiempo real en archivo de muestras del día.
		\end{itemize} 
		\\ \hline 
	\textbf{Prioridad} & 
		Alta. \\ \hline
	\textbf{Errores} &
		 \begin{itemize}
		 %DUDA: tiempo establecido por I2C Time Out and Clock Stretching?
		 	\item \label{SUB-M-CU1.5:Error1} Error 1: El archivo de tiempo real no existe.
		 	\item \label{SUB-M-CU1.5:Error2} Error 2: El archivo de tiempo real tiene una hora atrasada.
		 \end{itemize} \\ \hline
	\textbf{Reglas de negocio} & 
	    \begin{itemize}
	      \item  \ref{RN6}
		 \end{itemize}\\ \hline
	% \caption{}
	%\label{desc:SUB-M-CU1}
\end{longtable}

\paragraph{Trayectoria principal}
\label{SUB-M-CU1.5:TP}
	\begin{enumerate}
	    \item {[Servidor Embebido]} Transcurre el tiempo establecido en la regla de negocio \ref{RN6}
	    \item {[Servidor Embebido]} Obtiene por cada nodo de cada microcontrolador su valor en tiempo real. \hyperref[SUB-M-CU1.5:TB]{[Trayectoria Alternativa A]} \hyperref[SUB-M-CU1.5:TB]{[Trayectoria Alternativa B]}
	    \item {[Servidor Embebido]} Guarda el valor obtenido en las muestras del día de hoy.
	    \item {[Servidor Embebido]} Espera al término del minuto cuando sea propio.
		\item {[Servidor Embebido]} Retorna al paso 1 de \hyperref[SUB-M-CU1.5:TP]{[Trayectoria Principal]}
	\end{enumerate}
	Fin del caso de uso.

%TA? El sensor recibe una trama de no acuse de recibo por el microcontrolador. \hyperref[SUB-M-CU1.5:TP]{[Trayectoria Principal]}
\paragraph{Trayectoria alternativa A} \label{SUB-M-CU1.5:TA}
	El archivo de tiempo real no existe.
	\begin{enumerate}[label=A\arabic*.]
		\item {[Servidor Embebido]} Guarda como valor de producción en el archivo de tiempo real y de muestras del dia -1, que indica error con el servicio del microcontrolador.
	\end{enumerate}
	Fin de la trayectoria alternativa. Regresa al paso 3 de \hyperref[SUB-M-CU1.5:TP]{[Trayectoria Principal]}  

\paragraph{Trayectoria alternativa B} \label{SUB-M-CU1.5:TB}
	El archivo de tiempo real tiene una hora atrasada.
	\begin{enumerate}[label=B\arabic*.]
		\item {[Servidor Embebido]} Guarda como valor de producción en el archivo de tiempo real y de muestras del dia -1, que indica error con el servicio del microcontrolador.  
	\end{enumerate}
	Fin de la trayectoria alternativa. Regresa al paso 3 de \hyperref[SUB-M-CU1.5:TP]{[Trayectoria Principal]}  
