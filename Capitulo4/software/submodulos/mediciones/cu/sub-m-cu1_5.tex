\subsubsection{SUB-M-CU1.5-Ver Monitoreo en Tiempo Real}\label{SUB-M-CU1.5}
El monitoreo en tiempo real se realizará haciendo una petición de los valores del día actual que el servidor tiene almacenado para graficarlos en la aplicación de usuario y posteriormente actualizar la gráfica de lineas por medio de peticiones al servidor en el tiempo establecido en la Regla de Negocio \ref{RN6}.
\\ La comunicación se logrará mediante el patrón de intercambio de mensajes de la Regla de Negocio \ref{RN8} con el formato de datos establecido en la Regla de Negocio \ref{RN11}. 

\begin{longtable}{|J{5cm}|J{10.3cm}|}
	\hline
	\textbf{Nombre del caso de uso} &
		SUB-M-CU1.5-Ver monitoreo en Tiempo Real \\ \hline
	\textbf{Objetivo} &
		Enviar y recibir muestras del monitoreo para que la aplicación de usuario pueda construir una gráfica de lineas en tiempo real. \\ \hline
	\textbf{Actores} &
		\begin{itemize}
		    \item Usuario
			\item Servidor embebido
			\item Aplicación de Usuario
		\end{itemize} \\ \hline
	\textbf{Disparador} & 
	    El usuario inicia la aplicación de usuario dando clic en el icono \\ \hline 
	\textbf{Entradas} & 
		\begin{itemize}
				\item{[Servidor embebido]} Marca de tiempo
		\end{itemize}\\ \hline 
	\textbf{Salidas} & 
		\begin{itemize}
			\item{[Servidor]} Valor de la muestra
		\end{itemize} \\ \hline
	\textbf{Precondiciones} &
		Ninguna. \\ \hline
	\textbf{Postcondiciones} &
		Ninguna.\\ \hline
	\textbf{Condiciones de término} & 
		\begin{itemize}
			\item La aplicación de usuario recibe el valor o valores y construye la gráfica de lineas correspondiente.
		\end{itemize} \\ \hline 
	\textbf{Prioridad} & 
		Alta. \\ \hline
	\textbf{Errores} & 
		\begin{itemize}
		    \item \label{CU5:Error1} Error 1: Sin conexión con el servidor.
			\item \label{CU5:Error2} Error 2: La marca de tiempo requerida no existe en el servidor o es nulo su valor asociado.
		    \item \label{CU5:Error3} Error 3: Sin conexión con el cliente.
		    \item \label{CU5:Error4} Error 4: El valor de generación es -1
		    \item \label{CU5:Error5} Error 5: El valor de generación es 0
		\end{itemize} \\ \hline
	\textbf{Reglas de negocio} & 
		\begin{itemize}
		    \item \ref{RN6}
			\item \ref{RN8}
			\item \ref{RN11}
		\end{itemize} \\ \hline

	% \caption{}
	%\label{desc:SUB-M-CU1}
\end{longtable}

\paragraph{Trayectoria principal}
    \label{SUB-M-CU1.5:TP}
	\begin{enumerate}
	    \item {[Usuario]} Inicia la aplicación de usuario dando clic en el icono. \hyperref[SUB-M-CU5:TA]{[Trayectoria alternativa A]}
	    \item {[Aplicación de Usuario]} Envía la marca de tiempo actual. \hyperref[SUB-M-CU5:TB]{[Trayectoria alternativa B]} 
	    \item {[Servidor]} Recibe la petición, busca los valores almacenados hasta el momento del día de la marca de tiempo. \hyperref[SUB-M-CU5:TC]{[Trayectoria alternativa C]}
	    \item {[Servidor]} Envía los valores correspondientes \hyperref[SUB-M-CU5:TD]{[Trayectoria alternativa D]} 
	    \item {[Aplicación de Usuario]} Recibe los valores de monitoreo, muestra la interfaz de usuario \hyperref[fig:monitoreo]{[IU1-Monitoreo En Tiempo real]} con los datos obtenidos. \hyperref[SUB-M-CU5:TE]{[Trayectoria alternativa E]}
	\end{enumerate}
	Fin del caso de uso.

\paragraph{Trayectoria alternativa A} \label{SUB-M-CU5:TA}
	El usuario ya esta en la interfaz de usuario \hyperref[fig:monitoreo]{[IU1-Monitoreo En Tiempo real]}
	\begin{enumerate}[label=A\arabic*.]
		\item {[Aplicación de Usuario]} Transcurre el tiempo marcado por la Regla de Negocio \ref{RN6} y envía la marca de tiempo como petición al servidor. \hyperref[SUB-M-CU5:TF]{[Trayectoria alternativa F]} 
		\item {[Servidor]} Recibe la petición, busca en sus registros por el valor correspondiente a la marca de tiempo. \hyperref[SUB-M-CU5:TC]{[Trayectoria alternativa C]} 
		\item  {[Servidor]} Envía el valor correspondiente \hyperref[SUB-M-CU5:TG]{[Trayectoria alternativa G]}
        \item {[Aplicación de Usuario]} Recibe el valor de monitoreo, actualiza la interfaz de usuario \hyperref[fig:monitoreo]{[IU1-Monitoreo En Tiempo real]} con el dato obtenido.
        \hyperref[SUB-M-CU5:TH]{[Trayectoria alternativa H]}
        \item {[Aplicación de usuario]} Regresa al final de \hyperref[SUB-M-CU5:TP]{[Trayectoria Principal]}
	\end{enumerate}
	Fin de la trayectoria alternativa.

\paragraph{Trayectoria alternativa B} \label{SUB-M-CU5:TB}
	Sin conexión con el servidor.
	\begin{enumerate}[label=B\arabic*.]
		\item {[Aplicación de usuario]} Muestra la pantalla emergente \hyperref[fig:Error de Conexion]{[PE1-Error de Conexión]}.
		\item {[Aplicación de usuario]} Regresa al final de \hyperref[SUB-M-CU5:TP]{[Trayectoria Principal]}
	\end{enumerate}
	Fin de la trayectoria alternativa.

\paragraph{Trayectoria alternativa C} \label{SUB-M-CU5:TC}
	La marca de tiempo requerida no existe en el servidor o es nulo su valor asociado.
	\begin{enumerate}[label=C\arabic*.]
		\item {[Servidor]} Asigna el valor -1 a la respuesta para la aplicación de usuario.
		\item {[Aplicación de usuario]} Regresa al punto 4 de \hyperref[SUB-M-CU5:TP]{[Trayectoria Principal]}
	\end{enumerate}
	Fin de la trayectoria alternativa.

\paragraph{Trayectoria alternativa D} \label{SUB-M-CU5:TD}
	Sin conexión con el cliente.
	\begin{enumerate}[label=D\arabic*.]
		\item {[Servidor]} Aborta envió y espera por la siguiente petición.
		\item {[Aplicación de usuario]} Al no recibir respuesta del servidor en ese intervalo, muestra la pantalla emergente \hyperref[fig:Error de Conexion]{[PE1-Error de Conexión]}.
		\item {[Aplicación de usuario]} Regresa al final de \hyperref[SUB-M-CU5:TP]{[Trayectoria Principal]}
	\end{enumerate}
	Fin de la trayectoria alternativa.
	
\paragraph{Trayectoria alternativa E} \label{SUB-M-CU5:TE}
	El valor de respuesta es -1
	\begin{enumerate}[label=E\arabic*.]
		\item {[Aplicación de usuario]} Muestra la pantalla emergente \hyperref[fig:Error en el Servidor]{[PE2-Error en el Servidor]}.
		\item {[Aplicación de usuario]} Regresa al final de \hyperref[SUB-M-CU5:TP]{[Trayectoria Principal]}
	\end{enumerate}
	Fin de la trayectoria alternativa.
	
\paragraph{Trayectoria alternativa F} \label{SUB-M-CU5:TF}
	Sin conexión con el servidor.
	\begin{enumerate}[label=F\arabic*.]
		\item {[Aplicación de usuario]} Asigna 0 como valor de monitoreo, actualiza la interfaz de usuario \hyperref[fig:monitoreo]{[IU1-Monitoreo En Tiempo real]}  y genera la notificación \hyperref[fig:Alerta Conexion]{[N1-Alerta de Conexión]}.
		\item {[Aplicación de usuario]} Regresa al final de \hyperref[SUB-M-CU5:TA]{[Trayectoria alternativa A]}
	\end{enumerate}
	Fin de la trayectoria alternativa.
	
\paragraph{Trayectoria alternativa G} \label{SUB-M-CU5:TG}
	Sin conexión con el cliente.
	\begin{enumerate}[label=G\arabic*.]
		\item {[Servidor]} Aborta envió y espera por la siguiente petición.
		\item {[Aplicación de usuario]} Transcurre el tiempo marcado por la Regla de Negocio \ref{RN6} y al no recibir respuesta del servidor en ese intervalo, asigna 0 como valor de generación de esa marca de tiempo, actualiza la interfaz de usuario \hyperref[fig:monitoreo]{[IU1-Monitoreo En Tiempo real]} y genera la notificación \hyperref[fig:Alerta Conexion]{[N1-Alerta de Conexión]}.
		\item {[Aplicación de usuario]} Regresa al final de \hyperref[SUB-M-CU5:TA]{[Trayectoria alternativa A]}
	\end{enumerate}
	Fin de la trayectoria alternativa.
	
\paragraph{Trayectoria alternativa H} \label{SUB-M-CU5:TH}
	El valor de generación es -1
	\begin{enumerate}[label=H\arabic*.]
		\item {[Aplicación de usuario]} Asigna como valor de generación 0, actualiza la interfaz de usuario \hyperref[fig:monitoreo]{[IU1-Monitoreo En Tiempo real]} y genera la notificación \hyperref[fig:Alerta Servidor]{[N3-Alerta de Servidor]}.
		\item {[Aplicación de usuario]} Regresa al final de \hyperref[SUB-M-CU5:TA]{[Trayectoria alternativa A]}
	\end{enumerate}
	Fin de la trayectoria alternativa.
	
\paragraph{Trayectoria alternativa I} \label{SUB-M-CU5:TI}
	El valor de generación es 0
	\begin{enumerate}[label=I\arabic*.]
		\item {[Aplicación de usuario]}  Actualiza la interfaz de usuario \hyperref[fig:monitoreo]{[IU1-Monitoreo En Tiempo real]} y genera la notificación \hyperref[fig:Alerta Generacion]{[N2-Alerta de Generación]}.
		\item {[Aplicación de usuario]} Regresa al final de \hyperref[SUB-M-CU5:TA]{[Trayectoria alternativa A]}
	\end{enumerate}
	Fin de la trayectoria alternativa.