\subsubsection{SUB-C-CU2-Consultar mapa}\label{SUB-C-CU2}
Cuando el actor abre la aplicación móvil, se le muestra en un mapa su ubicación, junto con la ubicación de gasolineras que tenga cercanas. De estás gasolineras se muestra su dirección y su clasificación.

\begin{longtable}{|J{5cm}|J{10.3cm}|}
	\hline
	\textbf{Nombre del caso de uso} &
		SUB-C-CU2-Consultar mapa \\ \hline
	\textbf{Objetivo} &
		Mostrar al actor las gasolineras cercanas a él, además de la clasificación de las mismas. \\ \hline
	\textbf{Actores} &
		\begin{itemize}
			\item Cliente.
			\item Usuario no registrado.
		\end{itemize}
		 \\ \hline 
	\textbf{Disparador} & 
		El actor ingresa a la aplicación móvil. \\ \hline
	\textbf{Entradas} & 
		\begin{itemize}
				\item Ubicación del actor.
		\end{itemize}\\ \hline 
	\textbf{Salidas} & 
		\begin{itemize}
			\item Mapa que muestra las gasolineras cercanas a él, al igual que la clasificación y dirección de las mismas.
		\end{itemize} \\ \hline
	\textbf{Precondiciones} &
		Ninguna.\\ \hline
	\textbf{Postcondiciones} &
		\begin{itemize}
			\item El actor puede seleccionar una gasolinera de las que le aparecen en el mapa.
		\end{itemize} \\ \hline
	\textbf{Condiciones de término} & 
		\begin{itemize}
			\item Se muestra el mapa con la clasificación de gasolinera al actor.
		\end{itemize} \\ \hline 
	\textbf{Prioridad} & 
		Alta. \\ \hline
	\textbf{Errores} & Ninguno.
		% \begin{itemize}
		% 	\item \label{SUB-M-CU1:Error1} Error 1: .
		% \end{itemize} 
		\\ \hline
	\textbf{Reglas de negocio} & \ref{RN9}.
		% \begin{itemize}
		% 	\item \ref{RN1}.
		% \end{itemize}
		 \\ \hline
	% \caption{}
	%\label{desc:SUB-M-CU1}
\end{longtable}

\paragraph{Trayectoria principal}
	\begin{enumerate}
		\item {[Actor]} Abre la aplicación móvil.
		\item {[Sistema]} Obtiene la ubicación del usuario.
		\item {[Sistema]} Obtiene las gasolineras que se encuentran en un radio establecido por la regla de negocio \ref{RN9}.
		\item {[Sistema]} De las gasolineras cercanas al actor, obtiene la clasificación y dirección de las mismas.
		\item {[Sistema]} Consulta el api de Google Maps y marca las gasolineras.
		\item \label{SUB-C-CU2:Pantalla} {[Sistema]} Muestra la información obtenida en la pantalla \hyperref[fig:sub-c-iu2]{SUB-C-IU2-Consultar mapa}.
	\end{enumerate}
	Fin del caso de uso.

\paragraph{Puntos de extensión} \label{SUB-C-CU2:PE}
\begin{enumerate}[label=PE\arabic*.]
	\item Caso de uso \hyperref[SUB-C-CU2.1]{SUB-C-CU2.1-Realizar recorrido} en el paso \ref{SUB-C-CU2:Pantalla} de la Trayectoria principal.
	\item Caso de uso \hyperref[SUB-C-CU2.2]{SUB-C-CU2.2-Consultar gasolinera} en el paso \ref{SUB-C-CU2:Pantalla} de la Trayectoria principal.
\end{enumerate}
