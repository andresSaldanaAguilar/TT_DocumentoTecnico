\subsubsection{SUB-U-CU1.2-Ver gráfica de generación de energía en tiempo real}\label{SUB-U-CU1.2}
La gráfica de generación de energía en tiempo real es un componente de la interfaz de usuario \hyperref[fig:monitoreoReal]{[IU Ver Generación Actual]} y se obtendrá haciendo una petición al servidor para obtener la información del archivo que contiene las muestras del día actual, dichas muestras se clasificarán de acuerdo a la Regla de Negocios \ref{RN12}, posteriormente se actualizará la gráfica por medio de peticiones al servidor en el tiempo establecido en la Regla de Negocios \ref{RN23}
\\ 
La comunicación se logrará mediante el patrón de intercambio de mensajes de la Regla de Negocios \ref{RN8} con el formato de datos establecido en la Regla de Negocios \ref{RN11}  

\begin{longtable}{|J{5cm}|J{10.3cm}|}
	\hline
	\textbf{Nombre del caso de uso} &
		SUB-U-CU1.2-Ver gráfica de generación en tiempo real \\ \hline
	\textbf{Objetivo} &
		Conseguir los valores de generación de energía del día actual del nodo seleccionado para que la aplicación móvil pueda construir una gráfica de barras que se actualizará en tiempo real. \\ \hline
	\textbf{Actores} &
		\begin{itemize}
		    \item Usuario
			\item Servidor embebido
			\item Aplicación móvil
		\end{itemize} \\ \hline
	\textbf{Disparador} & 
	    El usuario inicia la aplicación desde su dispositivo móvil.\\ \hline 
	\textbf{Entradas} & 
		\begin{itemize}
				\item{[Aplicación móvil]} Identificador de servidor, microcontrolador y nodo seleccionado en la pantalla.
		\end{itemize}\\ \hline 
	\textbf{Salidas} & 
		\begin{itemize}
			\item Pantalla \hyperref[fig:monitoreoReal]{[IU Ver Generación Actual]} con la gráfica creada con la información de las muestras almacenadas del día en curso.
		\end{itemize} \\ \hline
	\textbf{Precondiciones} &
		Debe estar establecida la conexión entre la aplicación móvil y el servidor que atenderá las peticiones.
		\\ \hline
	\textbf{Postcondiciones} &
		Ninguna.\\ \hline
	\textbf{Condiciones de término} & 
		\begin{itemize}
			\item La aplicación móvil presenta la pantalla \hyperref[fig:monitoreoReal]{[IU Ver Generación Actual]} con la información de las muestras obtenidas del día en curso.
		\end{itemize} \\ \hline 
	\textbf{Prioridad} & 
		Alta. \\ \hline
	\textbf{Errores} & 
		\begin{itemize}
		    \item \label{CUU1.2:Error1} Error 1: Sin conexión con el servidor.
		\end{itemize} \\ \hline
	\textbf{Reglas de negocio} & 
		\begin{itemize}
		    \item \ref{RN8}
			\item \ref{RN11}
			\item \ref{RN12}
			\item \ref{RN23}
			\item \ref{RN24}
		\end{itemize} \\ \hline
\end{longtable}

\paragraph{Trayectoria principal}
    \label{SUB-U-CU1.2:TP}
	\begin{enumerate}
	    \item {[Usuario]} Inicia la aplicación desde su dispositivo móvil y visualiza la pantalla \hyperref[fig:monitoreoReal]{[IU Ver Generación Actual]}. \hyperref[SUB-U-CU1.2:TA]{[Trayectoria alternativa A]}
	    \item {[Aplicación móvil]} Envía una petición al servidor seleccionado; éste se carga por defecto en el primer selector que muestra la pantalla \hyperref[fig:monitoreoReal]{[IU Ver Generación Actual]} para conocer la estructura de microcontroladores y nodos que posee dicho servidor.\hyperref[SUB-U-CU1.2:TB]{[Trayectoria alternativa B]}
	    \item {[Servidor embebido]} Recibe la petición y manda como respuesta la estructura de microcontroladores y nodos que existen.
	    \item {[Aplicación móvil]} Recibe la respuesta del servidor y carga la información en los selectores de microcontroladores y nodos, mostrando el primer resultado de cada uno por defecto en su correspondiente selector de la pantalla \hyperref[fig:monitoreoReal]{[IU Ver Generación Actual]} 
	    \item {[Aplicación móvil]} Envía la petición al servidor especificando el microcontrolador y nodo seleccionado en cada uno de los selectores para obtener la información del archivo que contiene las muestras almacenadas durante el día del nodo seleccionado. \hyperref[SUB-U-CU1.2:TB]{[Trayectoria alternativa B]} 
	    \item {[Servidor embebido]} Recibe la petición, busca la información almacenada del nodo solicitado. 
	    \item {[Servidor embebido]} Envía la respuesta de la petición.
	    \item {[Aplicación móvil]} Recibe la información del nodo y la clasifica de acuerdo a la Regla de Negocios \ref{RN12} 
	    \item {[Aplicación móvil]} Muestra la pantalla \hyperref[fig:monitoreoReal]{[IU Ver Generación Actual]} con la gráfica de las muestras tomadas durante el día del nodo.
	\end{enumerate}
	Fin del caso de uso.

\paragraph{Trayectoria alternativa A} \label{SUB-U-CU1.2:TA}
	El usuario ya está dentro de la aplicación móvil y elige un nodo diferente al que se carga por defecto en los selectores de la pantalla \hyperref[fig:monitoreoReal]{[IU Ver Generación Actual]}.
	\begin{enumerate}[label=A\arabic*.]
		\item {[Aplicación móvil]} Regresa al paso 5 de la \hyperref[SUB-U-CU1.2:TP]{[Trayectoria Principal]}.
	\end{enumerate}
	Fin de la trayectoria alternativa.

\paragraph{Trayectoria alternativa B} \label{SUB-U-CU1.2:TB}
	Sin conexión con el servidor.
	\begin{enumerate}[label=B\arabic*.]
		\item {[Aplicación móvil]} Muestra el mensaje ''Consulta fallida'' en la pantalla \hyperref[fig:monitoreoReal]{[IU Ver Generación Actual]}.
	\end{enumerate}
	Fin de la trayectoria alternativa y fin del caso de uso.


