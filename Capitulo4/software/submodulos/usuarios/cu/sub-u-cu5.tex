\subsubsection{SUB-U-CU5-Autenticar usuario}\label{SUB-U-CU5}
Un actor puede autenticarse en el sistema para así registrar automóviles, sensores u editar su información.

\begin{longtable}{|J{5cm}|J{10.3cm}|}
	\hline
	\textbf{Nombre del caso de uso} &
		SUB-U-CU5-Autenticar usuario \\ \hline
	\textbf{Objetivo} &
		Permitir a un Cliente acceder a su cuenta para que pueda realizar acciones sobre la misma. \\ \hline
	\textbf{Actores} &
		Usuario no registrado. \\ \hline 
	\textbf{Disparador} & 
		El actor requiere autenticarse en el sistema. \\ \hline 
	\textbf{Entradas} & 
		\begin{itemize}
				\item Correo electrónico.
				\item Contraseña.
		\end{itemize}\\ \hline 
	\textbf{Salidas} & Ninguna.
		% \begin{itemize}
		% 	\item Cantidad de gasolina que debió ser cargada al automóvil.
		% \end{itemize} 
		\\ \hline
	\textbf{Precondiciones} &
		El actor requiere tener una cuenta activa en el sistema.\\ \hline
	\textbf{Postcondiciones} &
		\begin{itemize}
			\item El actor puede gestionar sus automóviles.
			\item El actor puede gestionar sus sensores.
			\item El actor puede gestionar su información.
			\item El actor puede reportar errores.
		\end{itemize} \\ \hline
	\textbf{Condiciones de término} & Se muestra la pantalla \hyperref[fig:sub-c-iu2]{SUB-C-IU2-Consultar mapa}
		% \begin{itemize}
		% 	\item El Cliente ingresa la cantidad de combustible que debió ser cargada.
		% 	\item Se realiza persistencia de la información calculada e ingresada.
		% \end{itemize} 
		\\ \hline 
	\textbf{Prioridad} & 
		Media. \\ \hline
	\textbf{Errores} & Ninguno.
		% \begin{itemize}
		% 	\item \label{SUB-M-CU1:Error1} Error 1: .
		% \end{itemize} 
		\\ \hline
	\textbf{Reglas de negocio} & Ninguna.
		% \begin{itemize}
		% 	\item \ref{RN1}.
		% \end{itemize}
		 \\ \hline
	% \caption{}
	%\label{desc:SUB-M-CU1}
\end{longtable}

\paragraph{Trayectoria principal}
	\begin{enumerate}
		\item {[Actor]} Selecciona la opción \textit{Iniciar sesión} del menú \hyperref[fig:menu-usuario]{Menú para Usuario no registrado}.
		\item \label{SUB-U-CU5:Pantalla} {[Sistema]} Muestra la pantalla \hyperref[fig:sub-u-iu5]{SUB-U-IU5-Autenticar usuario}.
		\item {[Actor]} Ingresa la información solicitada por la pantalla.
		\item \label{SUB-U-CU5:PresionaBoton} {[Actor]} Presiona el botón \textit{Iniciar sesión}.
		\item {[Sistema]} Verifica que la información ingresada corresponda con la información de una cuenta activa.\hyperref[SUB-U-CU5:TA]{Trayectoria alternativa A}
		\item \label{SUB-U-CU5:Pantalla} {[Sistema]} Muestra la pantalla \hyperref[fig:sub-c-iu2]{SUB-C-IU2-Consultar mapa}.
	\end{enumerate}
	Fin del caso de uso.

\paragraph{Trayectoria alternativa A} \label{SUB-U-CU5:TA}
	La información ingresada por el actor, no corresponde a la de ninguna cuenta activa.
	\begin{enumerate}[label=A\arabic*.]
		\item {[Sistema]} Muestra un mensaje indicando al actor que la información ingresada es incorrecta.
		\item {[Sistema]} Continúa en el paso \ref{SUB-U-CU5:Pantalla} de la trayectoria principal.
	\end{enumerate}
	Fin de la trayectoria alternativa.

\paragraph{Puntos de extensión} \label{SUB-U-CU5:PE}
\begin{enumerate}[label=PE\arabic*.]
	\item Caso de uso \hyperref[SUB-U-CU5.1]{SUB-U-CU5.1-Recuperar contraseña} en el paso \ref{SUB-U-CU5:PresionaBoton} de la trayectoria principal.
\end{enumerate}
