\subsubsection{SUB-U-CU1.12-Mediciones en tiempo real}\label{SUB-U-CU1.12}
El monitoreo para saber qué es lo que el sensor está midiendo en ese momento se realizará siempre que la aplicación esté en tiempo de funcionamiento, el cual es igual al periodo de tiempo comprendido en la Regla de Negocio \ref{RN6}, esto se logrará gracias a un proceso que se mantendrá corriendo y realizará constantes peticiones, el lapso que deberá trascurrir entre cada petición para saber las medidas otorgadas por el sensor será de un segundo, de esta manera se tendrá un mayor control sobre posibles anomalías y de detectarse alguna, poder notificar lo antes posible.
\\ La comunicación se logrará mediante el patrón de intercambio de mensajes de la Regla de Negocio \ref{RN8} con el formato de datos establecido en la Regla de Negocio \ref{RN11}  

\begin{longtable}{|J{5cm}|J{10.3cm}|}
	\hline
	\textbf{Nombre del caso de uso} &
		SUB-U-CU1.11-Mediciones en tiempo real \\ \hline
	\textbf{Objetivo} &
		Monitorizar cada segundo los valores de generación de energía del día actual dentro del periodo de tiempo comprendido en la Regla de Negocio \ref{RN6}, detectar algún tipo de anomalía en dichas mediciones. \\ \hline
	\textbf{Actores} &
		\begin{itemize}
		    \item Microcontrolador
			\item Servidor embebido
			\item Aplicación de Usuario
		    \item Usuario
		\end{itemize} \\ \hline
	\textbf{Disparador} & 
	    El usuario define un servidor que atenderá todas las peticiones de la aplicación de usuario y la conectividad con este es exitosa.\\ \hline 
	\textbf{Entradas} & 
		\begin{itemize}
				\item Ninguna
		\end{itemize}\\ \hline 
	\textbf{Salidas} & 
		\begin{itemize}
			\item{[Servidor]} Valor de generacion de energia
			\item{[Servidor]} Fecha y hora actual
			\item{[Aplicación de usuario]} Código de error
		\end{itemize} \\ \hline
	\textbf{Precondiciones} &
		La aplicación de usuario ha logrado una conexión exitosa con el servidor que atenderá sus peticiones como se indica en el caso de uso \hyperref[SUB-U-CU1.7]{SUB-U-CU1.7} \\ \hline
	\textbf{Postcondiciones} &
		Ninguna.\\ \hline
	\textbf{Condiciones de término} & 
		\begin{itemize}
			\item La aplicación de usuario recibe el valor y analiza que el valor no sea una falla, de ser así se procederá con lo indicado en el caso de uso \hyperref[SUB-U-CU1.11]{SUB-U-CU1.11}.
		\end{itemize} \\ \hline 
	\textbf{Prioridad} & 
		Alta. \\ \hline
	\textbf{Errores} & 
		\begin{itemize}
		    \item \label{CU5:Error1} Error 1: Sin conexión con el servidor.
		    \item \label{CU5:Error2} Error 3: Sin conexión con el cliente.
		    \item \label{CU5:Error3} Error 4: El valor de generación es -1
		    \item \label{CU5:Error4} Error 5: El valor de generación es 0
		\end{itemize} \\ \hline
	\textbf{Reglas de negocio} & 
		\begin{itemize}
		    \item \ref{RN6}
			\item \ref{RN8}
			\item \ref{RN11}
		\end{itemize} \\ \hline
\end{longtable}

\paragraph{Trayectoria principal}\label{SUB-U-CU1.12:TPr}
    \label{SUB-U-CU1.5:TP}
	\begin{enumerate}
	    \item {[Usuario]} Logra establecer una conexión con el servidor, seleccionando un servidor o ingresando su dirección, como se establece en el caso de uso \hyperref[SUB-U-CU1.7]{SUB-U-CU1.7}.
	    \item {[Aplicación de Usuario]} Envía petición para conocer el valor actualmente sensado. \hyperref[SUB-U-CU1.5:TA]{[Trayectoria alternativa A]} 
	    \item {[Servidor]} Recibe la petición, y hace la lectura de los datos enviados por el módulo WiFi como se indica en el caso de uso  \hyperref[SUB-M-CU1.4]{SUB-M-CU1.4}. \hyperref[SUB-U-CU1.5:TB]{[Trayectoria alternativa B]}
	    \item {[Servidor]} Envía el valor correspondientes a la aplicación de usuario \hyperref[SUB-U-CU1.5:TC]{[Trayectoria alternativa C]} 
	    \item {[Aplicación de Usuario]} Recibe el valor de monitoreo, almacena en una variable en la cual se sobreescribirá el siguiente valor recibido y valida si este no es un 0 o -1. \hyperref[SUB-U-CU1.5:TD]{[Trayectoria alternativa D]}
	\end{enumerate}
	Fin del caso de uso.


\paragraph{Trayectoria alternativa A} \label{SUB-U-CU1.5:TA}
	Sin conexión con el servidor.
	\begin{enumerate}[label=A\arabic*.]
		\item {[Aplicación de usuario]} Genera como salida código de error 1
	\end{enumerate}
	Fin de la trayectoria alternativa.

\paragraph{Trayectoria alternativa B} \label{SUB-U-CU1.5:TB}
	Lectura de un valor nulo.
	\begin{enumerate}[label=B\arabic*.]
		\item {[Servidor]} Asigna el valor -1 a la respuesta para la aplicación de usuario.
		\item {[Aplicación de usuario]} Genera como salida código de error 3
		\item Retorna al paso 4 de la trayectoria principal \hyperref[SUB-U-CU1.12:TPr]{[Trayectoria principal]}
	\end{enumerate}
	Fin de la trayectoria alternativa.

\paragraph{Trayectoria alternativa C} \label{SUB-U-CU1.5:TC}
	Sin conexión con el cliente.
	\begin{enumerate}[label=C\arabic*.]
		\item {[Servidor]} Aborta envió y espera por la siguiente petición.
		\item {[Aplicación de usuario]} Al no recibir respuesta del servidor en ese intervalo, muestra la pantalla emergente \hyperref[fig:Error de Conexion]{[PE1-Error de Conexión]} y almacena un 0 como valor de monitoreo.
		\item {[Aplicación de usuario]} Genera como salida código de error 2
	\end{enumerate}
	Fin de la trayectoria alternativa.
	
\paragraph{Trayectoria alternativa D} \label{SUB-U-CU1.5:TD}
	El valor de respuesta es 0
	\begin{enumerate}[label=D\arabic*.]
		\item {[Aplicación de usuario]} Genera como salida código de error 4
	\end{enumerate}
	Fin de la trayectoria alternativa.


