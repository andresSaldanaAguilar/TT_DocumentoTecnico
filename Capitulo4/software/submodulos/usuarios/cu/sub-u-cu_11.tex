\subsubsection{SUB-U-CU1.8-Obtener mediciones en tiempo real}\label{SUB-U-CU1.8}
El monitoreo para saber qué es lo que el sensor especificado está midiendo en ese momento se realizará siempre que la aplicación esté en tiempo de funcionamiento, el cual es igual al periodo de tiempo comprendido en la Regla de Negocios \ref{RN6}, esto se logrará gracias a un proceso que se mantendrá corriendo y realizará constantes peticiones al Servidor embebido; el lapso que deberá trascurrir entre cada petición para saber las mediciones otorgadas por el sensor será especificado en la Regla de negocios \ref{RN17}, de esta manera se tendrá un mayor control sobre las posibles anomalías y de detectarse alguna, poder notificar lo antes posible.
\\ La comunicación se logrará mediante el patrón de intercambio de mensajes de la Regla de Negocios \ref{RN8} con el formato de datos establecido en la Regla de Negocios \ref{RN11}  

\begin{longtable}{|J{5cm}|J{10.3cm}|}
	\hline
	\textbf{Nombre del caso de uso} &
		SUB-U-CU1.8-Obtener mediciones en tiempo real \\ \hline
	\textbf{Objetivo} &
		Monitorizar cada segundo los valores de generación de energía del día actual dentro del periodo de tiempo comprendido en la Regla de Negocios \ref{RN6}, detectar algún tipo de anomalía en dichas mediciones. \\ \hline
	\textbf{Actores} &
		\begin{itemize}
		    \item Microcontrolador
			\item Servidor embebido
			\item Aplicación de Usuario
		    \item Usuario
		\end{itemize} \\ \hline
	\textbf{Disparador} & 
	    El usuario define un Servidor embebido que atenderá todas las peticiones de la aplicación de usuario y la conectividad con este es exitosa.\\ \hline 
	\textbf{Entradas} & 
		\begin{itemize}
				\item Ninguna
		\end{itemize}\\ \hline 
	\textbf{Salidas} & 
		\begin{itemize}
			\item{[Servidor embebido]} Valor de generación de energía
			\item{[Servidor embebido]} Fecha y hora actual
			\item{[Aplicación de usuario]} Código de error
			\item{[Servidor embebido]} Identificador del sensor.
		\end{itemize} \\ \hline
	\textbf{Precondiciones} &
		La aplicación de usuario ha logrado una conexión exitosa con el Servidor embebido que atenderá sus peticiones como se indica en el caso de uso \hyperref[SUB-U-CU1.4]{SUB-U-CU1.4} \\ \hline
	\textbf{Postcondiciones} &
		Ninguna.\\ \hline
	\textbf{Condiciones de término} & 
		\begin{itemize}
			\item La aplicación de usuario recibe el valor y analiza que el valor no sea una falla, de ser así se procederá con lo indicado en el caso de uso \hyperref[SUB-U-CU1.7]{SUB-U-CU1.7}.
		\end{itemize} \\ \hline 
	\textbf{Prioridad} & 
		Alta. \\ \hline
	\textbf{Errores} & 
		\begin{itemize}
		    \item \label{CUU1.8:Error1} Error 1: Sin conexión con el Servidor embebido.
		    \item \label{CUU1.8:Error2} Error 2: Sin conexión con el cliente.
		    \item \label{CUU1.8:Error3} Error 3: El valor de generación es -1
		    \item \label{CUU1.8:Error4} Error 4: El valor de generación es 0
		\end{itemize} \\ \hline
	\textbf{Reglas de negocio} & 
		\begin{itemize}
		    \item \ref{RN6}
			\item \ref{RN8}
			\item \ref{RN11}
			\item \ref{RN17}
		\end{itemize} \\ \hline
\end{longtable}

\paragraph{Trayectoria principal}\label{SUB-U-CU1.8:TP}
    \label{SUB-U-CU1.8:TP}
	\begin{enumerate}
	    \item {[Usuario]} Logra establecer una conexión con el Servidor embebido, seleccionando un Servidor embebido o ingresando su dirección, como se establece en el caso de uso \hyperref[SUB-U-CU1.6]{SUB-U-CU1.6}.
	    \item {[Aplicación de Usuario]} Envía petición al Servidor embebido con el identificador del sensor que se desea, para conocer el valor actualmente sensado (esto para cada sensor). \hyperref[SUB-U-CU1.8:TA]{[Trayectoria alternativa A]} 
	    \item {[Servidor embebido]} Recibe la petición, y hace la lectura de los datos enviados por el módulo WiFi como se indica en el caso de uso  \hyperref[SUB-M-CU1.4]{SUB-M-CU1.4}. \hyperref[SUB-U-CU1.8:TB]{[Trayectoria alternativa B]}
	    \item {[Servidor embebido]} Envía el valor correspondiente a la aplicación de usuario \hyperref[SUB-U-CU1.8:TC]{[Trayectoria alternativa C]} 
	    \item {[Aplicación de Usuario]} Recibe el valor de monitoreo, lo almacena en una tabla de la base de datos, la cual contiene únicamente una tupla destinada a cada sensor donde se sobreescribirá el siguiente valor recibido y valida si este no es un 0. \hyperref[SUB-U-CU1.8:TD]{[Trayectoria alternativa D]}
	\end{enumerate}
	Fin del caso de uso.


\paragraph{Trayectoria alternativa A} \label{SUB-U-CU1.8:TA}
	Sin conexión con el Servidor embebido.
	\begin{enumerate}[label=A\arabic*.]
		\item {[Aplicación de usuario]} Genera como salida código de error 1
	\end{enumerate}
	Fin de la trayectoria alternativa y fin del caso de uso.

\paragraph{Trayectoria alternativa B} \label{SUB-U-CU1.8:TB}
	Lectura de un -1.
	\begin{enumerate}[label=B\arabic*.]
		\item {[Servidor embebido]} Envía el valor correspondiente a la aplicación de usuario
		\item {[Aplicación de usuario]} Recibe el valor y genera como salida código de error 3
	\end{enumerate}
	Fin de la trayectoria alternativa y fin del caso de uso.

\paragraph{Trayectoria alternativa C} \label{SUB-U-CU1.8:TC}
	Sin conexión con el cliente.
	\begin{enumerate}[label=C\arabic*.]
		\item {[Servidor embebido]} Aborta envió y espera por la siguiente petición.
		\item {[Aplicación de usuario]} Al no recibir respuesta del Servidor embebido en ese intervalo, muestra la pantalla emergente \hyperref[fig:Error de Conexion]{[PE1-Error de Conexión]} y almacena un 0 como valor de monitoreo.
		\item {[Aplicación de usuario]} Genera como salida código de error 2
	\end{enumerate}
	Fin de la trayectoria alternativa y fin del caso de uso.
	
\paragraph{Trayectoria alternativa D} \label{SUB-U-CU1.8:TD}
	El valor de respuesta es 0
	\begin{enumerate}[label=D\arabic*.]
		\item {[Aplicación de usuario]} Genera como salida código de error 4
	\end{enumerate}
	Fin de la trayectoria alternativa y fin del caso de uso.


