\subsubsection{SUB-U-CU1.3-Ver Reporte de Históricos}\label{SUB-U-CU1.3}
La visualización de reportes históricos tendrá lugar de acuerdo a los periodos establecidos en la Regla de Negocios \ref{RN4}, donde se mostrará la gráfica de barras con las especificaciones de la Regla de Negocios \ref{RN12} correspondiente al intervalo seleccionado, la comunicación se realizará mediante el patrón de intercambio de mensajes de la Regla de Negocios \ref{RN8} con el formato de datos establecido en la Regla de Negocios \ref{RN11}

\begin{longtable}{|J{5cm}|J{10.3cm}|}
	\hline
	\textbf{Nombre del caso de uso} &
		SUB-U-CU1.3-Ver Reporte de Históricos \\ \hline
	\textbf{Objetivo} &
		Graficar el periodo seleccionado por el usuario en la sección de históricos, en formato de barras. \\ \hline
	\textbf{Actores} &
		\begin{itemize}
			\item Servidor embebido
			\item Aplicación de Usuario
			\item Usuario 
		\end{itemize} \\ \hline
	\textbf{Disparador} & 
		El usuario selecciona del menú de navegación la sección de históricos. \\ \hline 
	\textbf{Entradas} & 
		\begin{itemize}
			\item{[Servidor embebido]} Periodo de tiempo seleccionado del menú desplegable de la pantalla \hyperref[fig:Opciones de Intervalo]{[IU-Opciones de Intervalo]}.
			\item{[Aplicación de usuario]} Identificador de servidor, microcontrolador y nodo seleccionado en la pantalla.
		\end{itemize}\\ \hline 
	\textbf{Salidas} & 
		\begin{itemize}
			\item{[Servidor embebido]} Valores históricos del intervalo de tiempo requerido de acuerdo a la \ref{RN6} y clasificándolos de acuerdo a la \ref{RN12}
		\end{itemize} \\ \hline
	\textbf{Precondiciones} &
		Ninguna.\\ \hline
	\textbf{Postcondiciones} &
		Ninguna.\\ \hline
	\textbf{Condiciones de término} & 
		\begin{itemize}
			\item La aplicación de usuario recibe los valores históricos requeridos y los grafica.
		\end{itemize} \\ \hline 
	\textbf{Prioridad} & 
		Media. \\ \hline
	\textbf{Errores} & 
		\begin{itemize}
		    \item \label{CUU1.3:Error1} Error 1: Sin conexión con el servidor.
		\end{itemize} \\ \hline
	\textbf{Reglas de negocio} & 
		\begin{itemize}
		    \item \ref{RN4}
		    \item \ref{RN5}
		    \item \ref{RN6}
		    \item \ref{RN8}
			\item \ref{RN11}
			\item \ref{RN12}
			\item \ref{RN24}
		\end{itemize} \\ \hline
\end{longtable}

\paragraph{Trayectoria principal}
    \label{SUB-U-CU1.3:TP}
	\begin{enumerate}
	    \item {[Usuario]} Selecciona Históricos en el menú de navegación de la pantalla \hyperref[fig:Barra de navegacion]{[IU Barra de Navegación]}. \hyperref[SUB-U-CU1.3:TA]{[Trayectoria alternativa A]} \hyperref[SUB-U-CU1.3:TE]{[Trayectoria alternativa E]} 
	    \hyperref[SUB-U-CU1.3:TF]{[Trayectoria alternativa F]} 
		\item {[Aplicación de Usuario]} Envía el periodo de tiempo que comprende al día anterior y el identificador de servidor, microcontrolador y nodo, siendo los primeros los que se tomen como valores por defecto. \hyperref[SUB-U-CU1.3:TB]{[Trayectoria alternativa B]} 
		\item {[Servidor embebido]} Recibe la petición, consigue los valores de acuerdo al periodo solicitado. 
		\item {[Servidor embebido]} Envía los valores como respuesta a la aplicación de usuario. \hyperref[SUB-U-CU1.3:TC]{[Trayectoria alternativa C]}
        \item {[Aplicación de Usuario]} Recibe la respuesta del servidor y valida los datos obtenidos.\hyperref[SUB-U-CU1.3:TD]{[Trayectoria alternativa D]}
        \item {[Aplicación de Usuario]} muestra la pantalla \hyperref[fig:Historico semanal]{[IU Histórico Semanal]} y grafica el arreglo de valores obtenidos. 
	\end{enumerate}
	Fin del caso de uso.

\paragraph{Trayectoria alternativa A} \label{SUB-U-CU1.3:TA}
    El usuario ya está en la pantalla de usuario \hyperref[fig:Historico semanal]{[IU Histórico Semanal]}  y selecciona el sensor del que desea conocer información.
	\begin{enumerate}[label=A\arabic*.]
	    \item {[Usuario]} Selecciona una opción del menú desplegable destinado a los periodos y otra destinada a los identificadores de los sensores de la pantalla de usuario \hyperref[fig:Opciones de Intervalo]{[IU Opciones de Intervalo]} 
	    \item {[Aplicación de Usuario]} Envía como petición el periodo de tiempo seleccionado por el usuario y el identificador del servidor, micro-controlador y nodo del que se desea obtener información \hyperref[SUB-U-CU1.3:TB]{[Trayectoria alternativa B]} 
	\end{enumerate}
	Fin de la trayectoria alternativa. Regresa al paso 3 de la \hyperref[SUB-U-CU1.3:TP]{[Trayectoria Principal]}.
	
\paragraph{Trayectoria alternativa B} \label{SUB-U-CU1.3:TB}
	Sin conexión con el servidor.
	\begin{enumerate}[label=B\arabic*.]
		\item {[Aplicación de usuario]} Al no obtener respuesta, se muestra en el área de graficación una etiqueta con la leyenda ''Información no disponible''
	\end{enumerate}
	Fin de la trayectoria alternativa y fin del caso de uso.

\paragraph{Trayectoria alternativa C} \label{SUB-U-CU1.3:TC}
	Sin conexión con el cliente.
	\begin{enumerate}[label=C\arabic*.]
		\item {[Servidor embebido]} Aborta envió y espera por la siguiente petición.
		\item {[Aplicación de usuario]} Al no obtener respuesta, se muestra en el área de graficación una etiqueta con la leyenda ''Información no disponible''
	\end{enumerate}
	Fin de la trayectoria alternativa y fin del caso de uso.
	
\paragraph{Trayectoria alternativa D} \label{SUB-U-CU1.3:TD}
	Algún valor de la respuesta es -1
	\begin{enumerate}[label=D\arabic*.]
		\item {[Aplicación de usuario]} Conserva ese valor como valor de generación para que se pueda visualizar en la gráfica, asignando un color distinto como se menciona en la \ref{RN12}.
	\end{enumerate}
	Fin de la trayectoria alternativa. Regresa al paso 6 de la \hyperref[SUB-U-CU1.3:TP]{[Trayectoria Principal]}.
	
\paragraph{Trayectoria alternativa E} \label{SUB-U-CU1.3:TE}
	El periodo seleccionado es ''Año''
	\begin{enumerate}[label=E\arabic*.]
		\item {[Aplicación de usuario]} Habilita el selector con la leyenda ''Año'' el cual permite navegar entre los ultimos 10 años transcurridos.
	\end{enumerate}
	Fin de la trayectoria alternativa. Regresa al paso 2 de la \hyperref[SUB-U-CU1.3:TP]{[Trayectoria Principal]}.

\paragraph{Trayectoria alternativa F} \label{SUB-U-CU1.3:TF}
	El periodo seleccionado no es ''Año''
	\begin{enumerate}[label=F\arabic*.]
		\item {[Aplicación de usuario]} El selector con la leyenda ''Año'' permanecerá inhabilitado puesto que no tiene ninguna funcionalidad para cualquier periodo que no sea ''Año''
	\end{enumerate}
	Fin de la trayectoria alternativa. Regresa al paso 2 de la \hyperref[SUB-U-CU1.3:TP]{[Trayectoria Principal]}.