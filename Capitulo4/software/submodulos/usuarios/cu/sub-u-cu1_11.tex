\subsubsection{SUB-U-CU1.8-Ver mejor marca del día}\label{SUB-U-CU1.8}
La barra de mejor marca del día se trata de un componente de la interfaz \hyperref[fig:monitoreoReal]{[IU Generación Actual]} y es la encargada de mostrar el valor de la mejor marca de producción de energía en un preciso instante de tiempo de uno de los nodos que está conectado al servidor, con posibles actualizaciones en el tiempo establecido en la Regla de Negocios \ref{RN22} siempre y cuando el usuario esté visualizando la pantalla \hyperref[fig:monitoreoReal]{[IU Generación Actual]}, la comunicación con el servidor se logrará mediante el patrón de intercambio de mensajes de la Regla de Negocios \ref{RN8}, con el formato de datos establecido en la Regla de Negocios \ref{RN11}   
\begin{longtable}{|J{5cm}|J{10.3cm}|}
	\hline
	\textbf{Nombre del caso de uso} &
		SUB-U-CU1.8-Ver mejor marca del día \\ \hline
	\textbf{Objetivo} &
		Mostrar al usuario el valor más alto de producción de energía en un instante del tiempo durante el día en curso. \\ \hline
	\textbf{Actores} &
		\begin{itemize}
		    \item Usuario
			\item Aplicación móvil
		\end{itemize} \\ \hline
	\textbf{Disparador} & 
	    El usuario inicia la aplicación desde su dispositivo móvil.\\ \hline 
	\textbf{Entradas} & 
		\begin{itemize}
				\item{[Aplicación móvil]} Identificador de servidor, microcontrolador y nodo seleccionado en la pantalla.
		\end{itemize}\\ \hline 
	\textbf{Salidas} & 
		\begin{itemize}
			\item Pantalla \hyperref[fig:monitoreoReal]{[IU Generación Actual]} con la información de la mejor marca del día del nodo seleccionado obtenida en tiempo real.
		\end{itemize} \\ \hline
	\textbf{Precondiciones} &
		Debe estar establecida la conexión entre la aplicación móvil y el servidor que atenderá las peticiones. \\ \hline
	\textbf{Postcondiciones} &
		Ninguna.\\ \hline
	\textbf{Condiciones de término} & 
		\begin{itemize}
			\item La aplicación móvil presenta la pantalla \hyperref[fig:monitoreoReal]{[IU Generación Actual]} con la información de la mejor marca del día obtenida en tiempo real.
		\end{itemize} \\ \hline 
	\textbf{Prioridad} & 
		Media. \\ \hline
	\textbf{Errores} & 
		\begin{itemize}
		    \item \label{CUU1.1:Error1} Error 1: Sin conexión con el servidor.
		\end{itemize} \\ \hline
	\textbf{Reglas de negocio} & 
		\begin{itemize}
		    \item \ref{RN8}
			\item \ref{RN11}
			\item \ref{RN22}
			\item \ref{RN24}
		\end{itemize} \\ \hline
\end{longtable}

\paragraph{Trayectoria principal}
    \label{SUB-U-CU1.8:TP}
    Este caso de uso es incluido por los casos de uso \hyperref[SUB-U-CU1.1]{[SUB-U-CU1.1-Ver estado de generación actual]} y \hyperref[SUB-U-CU1.2]{[SUB-U-CU1.2-Ver gráfica de generación de energía en tiempo real]} ya que cuando se realizan ambas consultas se pueden realizar comparaciones entre las muestras obtenidas para obtener la mejor marca del día.
	\begin{enumerate}
	    \item {[Usuario]} Inicia la aplicación desde su dispositivo móvil y visualiza la pantalla \hyperref[fig:monitoreoReal]{[IU Generación Actual]}.
	    \item {[Aplicación móvil]} Obtiene las muestras de la consulta realizada en el \hyperref[SUB-U-CU1.1]{[SUB-U-CU1.1-Ver estado de generación actual]} y \hyperref[SUB-U-CU1.2]{[SUB-U-CU1.2-Ver gráfica de generación de energía en tiempo real]}
	    \item {[Aplicación móvil]} Obtiene la muestra con el mayor valor de potencia activa de las muestras almacenadas durante el día y lo compara con el valor de generación actual del nodo. 
	    \item {[Aplicación móvil]} Muestra la pantalla \hyperref[fig:monitoreoReal]{[IU Generación Actual]} con el mayor valor de producción de energía en el instante de tiempo identificado.
	\end{enumerate}
	Fin del caso de uso.
