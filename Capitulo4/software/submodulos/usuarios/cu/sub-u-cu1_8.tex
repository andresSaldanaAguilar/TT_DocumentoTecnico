%DUDAS:
% Salidas: poner las pantallas? De ser así colocar las referencias de las imágenes
% Condiciones de término: colocar referencia a pantalla
% Trayectoria principal poner referencia de pantalla
% Poner las referencias de los CU's a los que extiende
% Primer paso poner imagen de pantalla principal y hacer la referencia
\subsubsection{SUB-U-CU1.8-Ver menú}\label{SUB-U-CU1.8}
El usuario puede ver el menú de la aplicación, desde el cual puede acceder de manera rápida a las secciones de monitoreo en tiempo real, históricos de mediciones, notificaciones y sincronización de dispositivo.

\begin{longtable}{|J{5cm}|J{10.3cm}|}
	\hline
	\textbf{Nombre del caso de uso} &
		SUB-U-CU1.8-Ver menú \\ \hline
	\textbf{Objetivo} &
		Permitirle al usuario consultar el monitoreo en tiempo real, el histórico de mediciones, las notificaciones recibidas y realizar la sincronización del dispositivo. \\ \hline
	\textbf{Actores} &
		Usuario. \\ \hline 
	\textbf{Disparador} & 
		El usuario requiere consultar el monitoreo en tiempo real o el histórico de mediciones o las notificaciones o bien realizar la sincronización del dispositivo.\\ \hline 
	\textbf{Entradas} & Ninguna.
		% \begin{itemize}
		% 		\item Nombre.
		% 		\item Imagen de perfil.
		% \end{itemize}
		\\ \hline 
	\textbf{Salidas} & 
		\begin{itemize}
			\item Pantalla de menú de la aplicación.
		\end{itemize} \\ \hline
	\textbf{Precondiciones} &
		Ninguna.\\ \hline
	\textbf{Postcondiciones} & Ninguna.
		% \begin{itemize}
		% 	\item El nombre e imagen de perfil son actualizados.
		% \end{itemize} 
		\\ \hline
	\textbf{Condiciones de término} & Se muestra la pantalla de menú de la aplicación.
		% \begin{itemize}
		% 	\item Se muestra el mensaje de éxito al actor.
		% \end{itemize} 
		\\ \hline 
	\textbf{Prioridad} & 
		Media. \\ \hline
	\textbf{Errores} & Ninguno.
		% \begin{itemize}
		% 	\item \label{SUB-M-CU1:Error1} Error 1: .
		% \end{itemize} 
		\\ \hline
	\textbf{Reglas de negocio} & Ninguna.
		% \begin{itemize}
		% 	\item \ref{RN1}.
		% \end{itemize}
		 \\ \hline
	% \caption{}
	%\label{desc:SUB-M-CU1}
\end{longtable}

\paragraph{Trayectoria principal}
	\begin{enumerate}
		\item {[Usuario]} Selecciona el ícono de \textit{Menú} de la pantalla principal %\hyperref[fig:menu-cliente]{Menú para Cliente}.
		\item {[Aplicación]} Muestra la pantalla de menú de la aplicación. %\hyperref[fig:sub-u-iu3]{SUB-U-IU3-Consultar panel de control} con la información obtenida.
	\end{enumerate}
	Fin del caso de uso.\\
	El paso 2 extiende a los Casos de Uso: 
	
	

