\subsubsection{SUB-U-CU3-Consultar panel de control}\label{SUB-U-CU3}
El Cliente puede acceder a un panel de control desde el cual puede editar información de su perfil, además de poder consultar su historial de mediciones y de insignias.

\begin{longtable}{|J{5cm}|J{10.3cm}|}
	\hline
	\textbf{Nombre del caso de uso} &
		SUB-U-CU3-Consultar panel de control \\ \hline
	\textbf{Objetivo} &
		Permitir al actor consultar el histórico de mediciones e insignias. \\ \hline
	\textbf{Actores} &
		Cliente. \\ \hline 
	\textbf{Disparador} & 
		El actor requiere consultar su histórico de insignias o de mediciones\\ \hline 
	\textbf{Entradas} & Ninguna.
		% \begin{itemize}
		% 		\item Nombre.
		% 		\item Imagen de perfil.
		% \end{itemize}
		\\ \hline 
	\textbf{Salidas} & 
		\begin{itemize}
			\item Imagen de perfil del actor.
			\item Nombre del actor.
			\item Mediciones hechas por el actor.
			\item Insignias obtenidas por el actor.
		\end{itemize} \\ \hline
	\textbf{Precondiciones} &
		Ninguna.\\ \hline
	\textbf{Postcondiciones} & Ninguna.
		% \begin{itemize}
		% 	\item El nombre e imagen de perfil son actualizados.
		% \end{itemize} 
		\\ \hline
	\textbf{Condiciones de término} & Se muestra la información correspondiente al actor.
		% \begin{itemize}
		% 	\item Se muestra el mensaje de éxito al actor.
		% \end{itemize} 
		\\ \hline 
	\textbf{Prioridad} & 
		Media. \\ \hline
	\textbf{Errores} & Ninguno.
		% \begin{itemize}
		% 	\item \label{SUB-M-CU1:Error1} Error 1: .
		% \end{itemize} 
		\\ \hline
	\textbf{Reglas de negocio} & Ninguna.
		% \begin{itemize}
		% 	\item \ref{RN1}.
		% \end{itemize}
		 \\ \hline
	% \caption{}
	%\label{desc:SUB-M-CU1}
\end{longtable}

\paragraph{Trayectoria principal}
	\begin{enumerate}
		\item {[Actor]} Selecciona la opción \textit{Perfil} del menú \hyperref[fig:menu-cliente]{Menú para Cliente}.
		\item {[Sistema]} Obtiene la imagen de perfil y nombre del actor.
		\item {[Sistema]} Obtiene la gasolinera, fecha y hora de las mediciones hechas por el actor.
		\item {[Sistema]} Obtiene las insignias obtenidas por el actor.
		\item {[Sistema]} Muestra la pantalla \hyperref[fig:sub-u-iu3]{SUB-U-IU3-Consultar panel de control} con la información obtenida.
	\end{enumerate}
	Fin del caso de uso.

% \paragraph{Trayectoria alternativa A} \label{SUB-M-CU1.1:TA}
% 	El actor no se encuentra usando la aplicación móvil.
% 	\begin{enumerate}[label=A\arabic*.]
% 		\item {[Sistema]} Muestra una notificación al actor como la que se observa en la pantalla \hyperref[fig:sub-m-iu1.1.a]{SUB-M-IU1.1-Confirmar medición (a)}.
% 		\item {[Actor]} Presiona la notificación.
% 		\item {[Sistema]} Continúa en el paso \ref{SUB-M-CU1.1:Pantalla} de la Trayectoria Principal.
% 	\end{enumerate}
% 	Fin de la trayectoria alternativa.

% \paragraph{Puntos de extensión} \label{SUB-M-CU1.1:P}
% \begin{enumerate}[label=PE\arabic*.]
% 	\item Caso de uso \hyperref[SUB-M-CU1.1.3]{SUB-M-CU1.1.3-Obtener insignia} en el paso \ref{SUB-M-CU1.1:Boton} de la Trayectoria principal.
% 	\item Caso de uso \hyperref[SUB-M-CU1.1.4]{SUB-M-CU1.1.4-Asignar insignia a gasolinera} en el paso \ref{SUB-M-CU1.1:Boton} de la Trayectoria principal.
% 	\item Caso de uso \hyperref[SUB-M-CU1.1.5]{SUB-M-CU1.1.5-Especificar bomba} en el paso \ref{SUB-M-CU1.1:Boton} de la Trayectoria principal.
% \end{enumerate}
