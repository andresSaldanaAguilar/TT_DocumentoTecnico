\subsubsection{SUB-U-CU1.1-Ver estado de generación actual}\label{SUB-U-CU1.1}
La barra de estado de generación actual es un componente de la interfaz \hyperref[fig:monitoreo]{[IU1-Generación Actual]} y es la encargada de mostrar el estado de generación actual de uno de los sensores del Sistema fotovoltaico, evaluándolo con la Regla de Negocios \ref{RN12}, con actualizaciones en el tiempo establecido en la Regla de Negocios \ref{RN17}, la comunicación con el Servidor se logrará mediante el patrón de intercambio de mensajes de la Regla de Negocios \ref{RN8} con el formato de datos establecido en la Regla de Negocios \ref{RN11}   
\begin{longtable}{|J{5cm}|J{10.3cm}|}
	\hline
	\textbf{Nombre del caso de uso} &
		SUB-U-CU1.1-Ver estado de generación actual \\ \hline
	\textbf{Objetivo} &
		Conseguir el valor actual de generación de energía y mostrarlo al usuario. \\ \hline
	\textbf{Actores} &
		\begin{itemize}
		    \item Usuario
			\item Servidor embebido
			\item Aplicación de Usuario
		\end{itemize} \\ \hline
	\textbf{Disparador} & 
	    El usuario inicia la aplicación de usuario, dando clic en el ícono correspondiente de la aplicación.\\ \hline 
	\textbf{Entradas} & 
		\begin{itemize}
				\item{[Servidor embebido]} Petición por el valor actual.
				\item{[Aplicación de usuario]} Identificador del sensor del cual se desea conocer información.
		\end{itemize}\\ \hline 
	\textbf{Salidas} & 
		\begin{itemize}
			\item Pantalla \hyperref[fig:monitoreo]{[IU1-Generación Actual]} indicando el estado de generación actual y el nivel en que se encuentra.
		\end{itemize} \\ \hline
	\textbf{Precondiciones} &
		Debe estar establecida la conexión entre la Aplicación de usuario y el Servidor que atenderá las peticiones. \\ \hline
	\textbf{Postcondiciones} &
		Ninguna.\\ \hline
	\textbf{Condiciones de término} & 
		\begin{itemize}
			\item La aplicación de usuario recibe el valor y muestra la pantalla \hyperref[fig:monitoreo]{[IU1-Generación Actual]} con el estado de generación actual.
		\end{itemize} \\ \hline 
	\textbf{Prioridad} & 
		Media. \\ \hline
	\textbf{Errores} & 
		\begin{itemize}
		    \item \label{CUU1.1:Error1} Error 1: Sin conexión con el servidor.
			\item \label{CUU1.1:Error2} Error 2: La marca de tiempo requerida no existe en el servidor.
		    \item \label{CUU1.1:Error3} Error 3: Sin conexión con el cliente.
		    \item \label{CUU1.1:Error4} Error 4: El valor de generación es -1.
		\end{itemize} \\ \hline
	\textbf{Reglas de negocio} & 
		\begin{itemize}
		    \item \ref{RN8}
			\item \ref{RN11}
			\item \ref{RN12}
			\item \ref{RN17}
		\end{itemize} \\ \hline
\end{longtable}

\paragraph{Trayectoria principal}
    \label{SUB-U-CU1.1:TP}
	\begin{enumerate}
	    \item {[Usuario]} Inicia la Aplicación de usuario dando clic en el ícono correspondiente de la aplicación. y por defecto se elige el primer sensor o el único. \hyperref[SUB-U-CU1.1:TA]{[Trayectoria alternativa A]}
	    \item {[Aplicación de Usuario]} Envía la petición para obtener el valor de generación actual. \hyperref[SUB-U-CU1.1:TB]{[Trayectoria alternativa B]} 
	    \item {[Servidor embebido]} Recibe la petición, busca el valor actual de generación almacenado. \hyperref[SUB-U-CU1.1:TC]{[Trayectoria alternativa C]}
	    \item {[Servidor embebido]} Envía el valor correspondiente. \hyperref[SUB-U-CU1.1:TD]{[Trayectoria alternativa D]} 
	    \item {[Aplicación de Usuario]} Recibe el valor de generación actual y valida que este no sea un número negativo. \hyperref[SUB-U-CU1.1:TE]{[Trayectoria alternativa E]} 
	    \item {[Aplicación de Usuario]} Muestra la pantalla \hyperref[fig:monitoreo]{[IU1-Generación Actual]} con el valor actualizado de estado de generación actual.
	\end{enumerate}
	Fin del caso de uso.

\paragraph{Trayectoria alternativa A} \label{SUB-U-CU1.1:TA}
	El usuario ya está dentro de la aplicación de usuario y elige un sensor a monitorear.
	\begin{enumerate}[label=A\arabic*.]
		\item {[Aplicación de Usuario]} Transcurre el tiempo marcado por la Regla de Negocio \ref{RN17} y envía la petición para obtener el valor de generación actual, junto con el identificador del sensor que se desea monitorear. \hyperref[SUB-U-CU1.1:TB]{[Trayectoria alternativa B]} 
	\end{enumerate}
	Fin de la trayectoria alternativa. Regresa al paso 3 de la \hyperref[SUB-U-CU1.1:TP]{[Trayectoria Principal]}.

\paragraph{Trayectoria alternativa B} \label{SUB-U-CU1.1:TB}
	Sin conexión con el servidor.
	\begin{enumerate}[label=B\arabic*.]
		\item {[Aplicación de usuario]} Actualiza como estado actual la palabra ''desconectado'' en la pantalla \hyperref[fig:monitoreo]{[IU1-Generación Actual]}.
	\end{enumerate}
	Fin de la trayectoria alternativa y fin del caso de uso.

\paragraph{Trayectoria alternativa C} \label{SUB-U-CU1.1:TC}
	La marca de tiempo requerida no existe en la base de datos del servidor.
	\begin{enumerate}[label=C\arabic*.]
		\item {[Servidor embebido]} Asigna el valor -1 a la respuesta para la Aplicación de usuario.
	\end{enumerate}
	Fin de la trayectoria alternativa. Regresa al punto 4 de la  \hyperref[SUB-U-CU1.1:TP]{[Trayectoria Principal]}

\paragraph{Trayectoria alternativa D} \label{SUB-U-CU1.1:TD}
	Sin conexión con el cliente.
	\begin{enumerate}[label=D\arabic*.]
		\item {[Servidor embebido]} Aborta envió y espera por la siguiente petición.
		\item {[Aplicación de usuario]} Al no recibir respuesta del servidor en el intervalo establecido, actualiza como estado actual la palabra ''desconectado'' en la interfaz de usuario \hyperref[fig:monitoreo]{[IU1-Generación Actual]}.
	\end{enumerate}
	Fin de la trayectoria alternativa y fin del caso de uso.
	
\paragraph{Trayectoria alternativa E} \label{SUB-U-CU1.1:TE}
	El valor de respuesta es -1
	\begin{enumerate}[label=E\arabic*.]
		\item {[Aplicación de usuario]} Actualiza como estado actual la palabra ''error servidor'' en la interfaz de usuario. \hyperref[fig:monitoreo]{[IU1-Generación Actual]}.
	\end{enumerate}
	Fin de la trayectoria alternativa y fin del caso de uso.