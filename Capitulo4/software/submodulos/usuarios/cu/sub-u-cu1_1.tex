\subsubsection{SUB-U-CU1.1-Ver estado de generación actual}\label{SUB-U-CU1.1}
La barra de estado de generación actual se trata de un componente de la interfaz \hyperref[fig:monitoreoReal]{[IU Ver Generación Actual]} y es la encargada de mostrar el estado de generación actual de uno de los nodos que está conectado al servidor, evaluándolo con la Regla de Negocios \ref{RN12}, con actualizaciones en el tiempo establecido en la Regla de Negocios \ref{RN22}, la comunicación con el servidor se logrará mediante el patrón de intercambio de mensajes de la Regla de Negocios \ref{RN8}, con el formato de datos establecido en la Regla de Negocios \ref{RN11}   
\begin{longtable}{|J{5cm}|J{10.3cm}|}
	\hline
	\textbf{Nombre del caso de uso} &
		SUB-U-CU1.1-Ver estado de generación actual \\ \hline
	\textbf{Objetivo} &
		Conseguir la muestra que contiene el valor actual de generación de energía de un nodo y mostrarlo al usuario. \\ \hline
	\textbf{Actores} &
		\begin{itemize}
		    \item Usuario
			\item Servidor embebido
			\item Aplicación móvil
		\end{itemize} \\ \hline
	\textbf{Disparador} & 
	    El usuario inicia la aplicación desde su dispositivo móvil.\\ \hline 
	\textbf{Entradas} & 
		\begin{itemize}
				\item{[Aplicación móvil]} Identificador de servidor, microcontrolador y nodo seleccionado en la pantalla.
		\end{itemize}\\ \hline 
	\textbf{Salidas} & 
		\begin{itemize}
			\item Pantalla \hyperref[fig:monitoreoReal]{[IU Ver Generación Actual]} con la información de la muestra del nodo seleccionado obtenida en tiempo real.
		\end{itemize} \\ \hline
	\textbf{Precondiciones} &
		Debe estar establecida la conexión entre la aplicación móvil y el servidor que atenderá las peticiones. \\ \hline
	\textbf{Postcondiciones} &
		Ninguna.\\ \hline
	\textbf{Condiciones de término} & 
		\begin{itemize}
			\item La aplicación móvil presenta la pantalla \hyperref[fig:monitoreoReal]{[IU Ver Generación Actual]} con la información de la muestra obtenida en tiempo real.
		\end{itemize} \\ \hline 
	\textbf{Prioridad} & 
		Media. \\ \hline
	\textbf{Errores} & 
		\begin{itemize}
		    \item \label{CUU1.1:Error1} Error 1: Sin conexión con el servidor.
		\end{itemize} \\ \hline
	\textbf{Reglas de negocio} & 
		\begin{itemize}
		    \item \ref{RN8}
			\item \ref{RN11}
			\item \ref{RN12}
			\item \ref{RN22}
			\item \ref{RN24}
		\end{itemize} \\ \hline
\end{longtable}

\paragraph{Trayectoria principal}
    \label{SUB-U-CU1.1:TP}
	\begin{enumerate}
	    \item {[Usuario]} Inicia la aplicación desde su dispositivo móvil y visualiza la pantalla \hyperref[fig:monitoreoReal]{[IU Ver Generación Actual]}. \hyperref[SUB-U-CU1.1:TA]{[Trayectoria alternativa A]}
	    \item {[Aplicación móvil]} Envía una petición al servidor seleccionado; este se carga por defecto en el primer selector que muestra la pantalla \hyperref[fig:monitoreoReal]{[IU Ver Generación Actual]} para conocer la estructura de microcontroladores y nodos que posee dicho servidor.\hyperref[SUB-U-CU1.1:TB]{[Trayectoria alternativa B]}
	    \item {[Servidor embebido]} Recibe la petición y manda como respuesta la estructura de microcontroladores y nodos que existen.
	    \item {[Aplicación móvil]} Recibe la respuesta del servidor y carga la información en los selectores de microcontroladores y nodos, mostrando el primer resultado de cada uno por defecto en su correspondiente selector de la pantalla \hyperref[fig:monitoreoReal]{[IU Ver Generación Actual]} 
	    \item {[Aplicación móvil]} Envía la petición al servidor especificando el microcontrolador y nodo seleccionado en cada uno de los selectores para obtener el valor de generación actual del nodo. \hyperref[SUB-U-CU1.1:TB]{[Trayectoria alternativa B]} 
	    \item {[Servidor embebido]} Recibe la petición, busca el valor actual de generación almacenado del nodo solicitado. 
	    \item {[Servidor embebido]} Envía la respuesta de la petición.
	    \item {[Aplicación móvil]} Recibe el valor de generación actual del nodo y lo clasifica de acuerdo a la Regla de Negocios \ref{RN12}. 
	    \item {[Aplicación móvil]} Muestra la pantalla \hyperref[fig:monitoreoReal]{[IU Ver Generación Actual]} con el valor actualizado de estado de generación actual.
	\end{enumerate}
	Fin del caso de uso.

\paragraph{Trayectoria alternativa A} \label{SUB-U-CU1.1:TA}
	El usuario ya está dentro de la aplicación móvil y elige un nodo diferente al que se carga por defecto en los selectores de la pantalla \hyperref[fig:monitoreoReal]{[IU Ver Generación Actual]}.
	\begin{enumerate}[label=A\arabic*.]
		\item {[Aplicación móvil]} Regresa al paso 5 de la \hyperref[SUB-U-CU1.1:TP]{[Trayectoria Principal]}.
	\end{enumerate}
	Fin de la trayectoria alternativa.

\paragraph{Trayectoria alternativa B} \label{SUB-U-CU1.1:TB}
	Sin conexión con el servidor.
	\begin{enumerate}[label=B\arabic*.]
		\item {[Aplicación móvil]} Muestra el mensaje ''Consulta fallida'' en la pantalla \hyperref[fig:monitoreoReal]{[IU Ver Generación Actual]}.
	\end{enumerate}
	Fin de la trayectoria alternativa y fin del caso de uso.
