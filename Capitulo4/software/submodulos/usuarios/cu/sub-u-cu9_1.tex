\subsubsection{SUB-U-CU9.1-Liberar reporte}\label{SUB-U-CU9.1}
Un administrador puede eliminar un reporte que se haya generado cuando el problema haya sido resuelto o sea desestimado.

\begin{longtable}{|J{5cm}|J{10.3cm}|}
	\hline
	\textbf{Nombre del caso de uso} &
		SUB-U-CU9.1-Liberar reporte \\ \hline
	\textbf{Objetivo} &
		Permitir al actor liberar un reporte cuando sea necesario. \\ \hline
	\textbf{Actores} &
		Administrador. \\ \hline 
	\textbf{Disparador} & 
		El error reportado en el reporte ha sido resuelto o desestimado. \\ \hline 
	\textbf{Entradas} & Reporte seleccionado.
		% \begin{itemize}
		% 		\item Cantidad de combustible cargada al automóvil.
		% 		\item Fecha y hora de la carga.
		% \end{itemize}
		\\ \hline 
	\textbf{Salidas} & Mensaje de éxito.
		% \begin{itemize}
		% 	\item Cantidad de gasolina que debió ser cargada al automóvil.
		% \end{itemize} 
		\\ \hline
	\textbf{Precondiciones} &
		Debe existir al menos un reporte.\\ \hline
	\textbf{Postcondiciones} & El reporte liberado no puede ser consultado de nuevo.
		% \begin{itemize}
		% 	\item 
		% \end{itemize}
		\\ \hline
	\textbf{Condiciones de término} & 
		\begin{itemize}
			\item El Cliente ingresa la cantidad de combustible que debió ser cargada.
			\item Se realiza persistencia de la información calculada e ingresada.
		\end{itemize} \\ \hline 
	\textbf{Prioridad} & 
		Alta. \\ \hline
	\textbf{Errores} & Ninguno.
		% \begin{itemize}
		% 	\item \label{SUB-M-CU1:Error1} Error 1: .
		% \end{itemize} 
		\\ \hline
	\textbf{Reglas de negocio} & Ninguna.
		% \begin{itemize}
		% 	\item \ref{RN1}.
		% \end{itemize}
		 \\ \hline
	% \caption{}
	%\label{desc:SUB-M-CU1}
\end{longtable}

\paragraph{Trayectoria principal}
	\begin{enumerate}
		\item {[Actor]} Selecciona un reporte de la pantalla \hyperref[fig:sub-u-iu9]{SUB-U-IU9-Consultar reporte}.
		\item {[Sistema]} Muestra la pantalla \hyperref[fig:sub-u-iu9.1]{SUB-U-IU9.1-Liberar reporte} con la información obtenida.
		\item {[Actor]} Presiona el botón \textit{Liberar}.\hyperref[SUB-U-CU9.1:TA]{Trayectoria alternativa A}
		\item {[Sistema]} Elimina el reporte seleccionado.
		\item {[Sistema]} Muestra la pantalla \hyperref[fig:sub-u-iu9]{SUB-U-IU9-Consultar reporte}.
	\end{enumerate}
	Fin del caso de uso.

\paragraph{Trayectoria alternativa A} \label{SUB-U-CU9.1:TA}
	El actor no requiere liberar el reporte.
	\begin{enumerate}[label=A\arabic*.]
		\item {[Sistema]} Muestra la pantalla \hyperref[fig:sub-u-iu9]{SUB-U-IU9-Consultar reporte}.
	\end{enumerate}
	Fin de la trayectoria alternativa.
