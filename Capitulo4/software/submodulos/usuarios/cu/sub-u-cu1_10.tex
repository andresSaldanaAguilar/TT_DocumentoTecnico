\subsubsection{SUB-U-CU1.7-Generar notificaciones}\label{SUB-U-CU1.7}
Siempre y cuando el usuario permita el monitoreo de los nodos conectados al servidor como se indica en la regla de negocios \ref{RN18} y dependiendo del acontecimiento detectado y con base en \ref{RN21}, se dará aviso al usuario a través de una notificación del dispositivo móvil el tipo de anomalía y la hora en la que fue detectada, siendo a su vez almacenada localmente en la aplicación de usuario para su posterior uso.  
\begin{longtable}{|J{5cm}|J{10.3cm}|}
	\hline
	\textbf{Nombre del caso de uso} &
		SUB-U-CU1.7 - Generar notificaciones
 \\ \hline
	\textbf{Objetivo} &
		Notificar al usuario sobre 3 posibles sucesos relevantes que puede haber en la aplicación:
		\begin{itemize}
			\item 1 - Generación de producción de energía igual a 0
			\item 2 - Problemas de obtención de la medición de uno o más nodos
			\item 3 - Sin respuesta del Servidor
		\end{itemize} \\ \hline
	\textbf{Actores} &
	    \begin{itemize}
				\item Usuario
				\item Aplicación de usuario 
				\item Servidor embebido
		\end{itemize}
		 \\ \hline 
	\textbf{Disparador} & 
		Se detectó uno de los 3 sucesos mencionados en el objetivo, ya sea por medio del valor de la muestra que se está recibiendo o por la falta de respuesta por parte del servidor al hacer una petición. \\ \hline 
	\textbf{Entradas} & 
		\begin{itemize}
				\item Valor de la muestra recibida
				\item Ausencia de respuesta por parte del servidor
		\end{itemize}\\ \hline 
	\textbf{Salidas} & 
		\begin{itemize}
		    \item Especificación del suceso detectado
			\item Fecha y hora en que se detectó el suceso.
			\item Notificación con la información acerca del suceso y hora en la que fue detectado, a través del dispositivo móvil.
			\item Identificador del servidor, microcontrolador y nodo.
		\end{itemize} \\ \hline
	\textbf{Precondiciones} &
		El usuario habilitó el servicio de notificaciones.\\ \hline
	\textbf{Postcondiciones} &
		\begin{itemize}
			\item El suceso junto con la fecha,hora del día e identificador del servidor, microcontrolador y nodo en que fue detectada son almacenados.
		\end{itemize}\\ \hline
	\textbf{Condiciones de término} & 
		\begin{itemize}
			\item Se obtiene el motivo por el cuál se provocó el suceso.
			\item Se obtiene la fecha y hora del día en que el suceso fue detectado.
			\item Se obtiene el identificador del servidor, microcontrolador y nodo en que se detectó el suceso.
		\end{itemize} \\ \hline 
	\textbf{Prioridad} & 
		Alta. \\ \hline
	\textbf{Errores} & 
		Ninguno \\ \hline
	\textbf{Reglas de negocio} & 
	    \begin{itemize}
			\item \ref{RN6}
			\item \ref{RN18}
			\item \ref{RN21}
			\item \ref{RN24}
		\end{itemize} \\ \hline 
		 

	% \caption{}
	%\label{desc:SUB-M-CU1}
\end{longtable}

\paragraph{Trayectoria principal}
	\begin{enumerate}
		\item {[Usuario]} Habilita el servicio de notificaciones como se indica en la regla de negocios \ref{RN18} al activar el switch de notificaciones que se muestra en la pantalla \hyperref[fig:Configuraciones]{[IU Configuraciones]} 
        \item {[Aplicación de usuario]} Monitorea cada 10 segundos a través de la aplicación móvil cada uno de los nodos conectados al servidor, enviando peticiones al servidor para conocer el valor de la muestra más actual.
        \item {[Servidor embebido]} Recibe petición de la aplicación móvil.
        \item {[Servidor embebido]} Envía respuesta con la muestra más actual del nodo solicitado por la aplicación de usuario.
		\item {[Aplicación de usuario]} Clasifica e identifica el suceso que se ha detectado, es decir, será "No se ha recibido respuesta del servidor" si no se obtiene respuesta por parte del servidor a una petición , ''Se ha detectado una generación de energía de 0'' En el caso de que la muestra tenga una producción de energía 0 y ''Se ha detectado una muestra perdida'' en el caso de que no haya sido posible obtener la muestra del nodo que se monitorea o el proceso encargado de consultar dichas muestras no funcione correctamente. 
		\item {[Aplicación de usuario]} Obtiene el identificador del servidor, microcontrolador y nodo en el cuál se ha detectado el suceso.
		\item {[Aplicación de usuario]} Alerta a través de una notificación del dispositivo móvil sobre el tipo de suceso, con los datos obtenidos con anterioridad.
		\item {[Aplicación de usuario]} Almacena tanto el tipo de suceso, el identificador del servidor, microcontrolador y nodo, la fecha y hora en la cual se detectó.
	\end{enumerate}
	Fin del caso de uso.
