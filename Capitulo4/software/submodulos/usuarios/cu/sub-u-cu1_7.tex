\subsubsection{SUB-U-CU1.4-Almacenar servidor en el dispositivo móvil}\label{SUB-U-CU1.4}
Una vez iniciada la aplicación, el usuario deberá registrar el servidor que atenderá todas sus peticiones por medio del dispositivo móvil, de acuerdo a la Regla de negocios \ref{RN24}
\begin{longtable}{|J{5cm}|J{10.3cm}|}
	\hline
	\textbf{Nombre del caso de uso} &
		SUB-U-CU1.4-Almacenar servidor en el dispositivo móvil \\ \hline
	\textbf{Objetivo} &
		Registrar el servidor en el dispositivo móvil para que posteriormente se configuren las peticiones hacia su dirección IP. \\ \hline
	\textbf{Actores} &
	    \begin{itemize}
			\item Aplicación de usuario
			\item Usuario administrador
			\item Usuario cliente
		    \item Servidor embebido
		\end{itemize}\\ \hline 
	\textbf{Disparador} & 
		El usuario administrador o usuario cliente quiere registrar un servidor.\\ \hline 
	\textbf{Entradas} & 
		Dirección IP del servidor con el cual se realizará la conexión, de acuerdo con la Regla de negocios \ref{RN16}\\ \hline 
	\textbf{Salidas} & 
		Servidor agregado que se mostrará en la pantalla \hyperref[fig:Configuraciones]{[IU Configuraciones]} \\ \hline
	\textbf{Precondiciones} &
		El usuario administrador o usuario cliente presiona el botón de Configuraciones que se muestra en la pantalla \hyperref[fig:Barra de navegacion]{[IU Barra de Navegación]}.\\ \hline 
	\textbf{Postcondiciones} &
		La nueva dirección IP del servidor con el que se desea tener comunicación es almacenada dentro de la aplicación móvil.\\ \hline 
	\textbf{Condiciones de término} & 
		Se logra una conexión exitosa con el nuevo servidor. \\ \hline
	\textbf{Prioridad} & 
		Alta. \\ \hline
	\textbf{Errores} & 
		\begin{itemize}
			\item \label{SUB-U-CU1.4:Error1} Error 1: No se puede establecer una conexión con el servidor.
		\end{itemize} \\ \hline
	\textbf{Reglas de negocio} & 
	    \begin{itemize}
		    \item \ref{RN16}
		    \item \ref{RN24}
		    \item \ref{RN25}
		    \item \ref{RN26}
		\end{itemize} \\ \hline
\end{longtable}

\paragraph{Trayectoria principal} \label{SUB-U-CU1.4:TP}
	\begin{enumerate}
		\item {[Usuario administrador o Usuario cliente]} Inicia la aplicación móvil, selecciona en el menú de la aplicación la opción de Configuraciones que se muestra en la pantalla \hyperref[fig:Barra de navegacion]{[IU Barra de Navegación]}
		\item {[Aplicación de usuario]} Muestra la pantalla \hyperref[fig:Configuraciones]{[IU Configuraciones]}
		\item {[Usuario administrador o Usuario cliente]} Ingresa la IP del servidor al que se desea conectar en el campo de texto que se encuentra en la sección 'IP del servidor' de la pantalla y presiona el botón de 'Enviar'.
		\item {[Aplicación de usuario]} Verifica que la IP no sea nula.\hyperref[SUB-U-CU1.4:TA]{[Trayectoria alternativa A]}
		\item {[Aplicación de usuario]} Verifica que la IP cumpla con el formato establecido en la regla de negocios \ref{RN16} \hyperref[SUB-U-CU1.4:TB]{[Trayectoria alternativa B]}
		\item {[Aplicación de usuario]} Genera la petición para conectarse al servidor solicitado y la envía. \hyperref[SUB-U-CU1.4:TC]{[Trayectoria alternativa C]} \hyperref[SUB-U-CU1.4:TH]{[Trayectoria alternativa H]}
		\item {[Servidor embebido]} Recibe la petición de conexión del usuario y verifica si ya existe un correo electrónico y una contraseña configurada, en caso negativo devuelve como respuesta a la aplicación móvil un 'set password'.
		\item {[Aplicación de usuario]} Recibe la respuesta del servidor y muestra el cuadro de diálogo solicitando al usuario que ingrese un correo electrónico y una contraseña como se muestra en la pantalla \hyperref[fig:Set password]{[IU5-Cuadro de diálogo para configuración de correo electrónico y contraseña]}
		\item {[Usuario administrador]} Ingresa un correo electrónico y una contraseña para el servidor. Oprime el botón 'Aceptar'. \hyperref[SUB-U-CU1.4:TD]{[Trayectoria alternativa D]}
		\item {[Aplicación de usuario]} Valida que tanto el campo de correo electrónico como el campo de contraseña no sean nulos. \hyperref[SUB-U-CU1.4:TE]{[Trayectoria alternativa E]}
		\item {[Aplicación de usuario]} Valida que el correo cumpla con el formato de correo electrónico establecido en la regla de negocios \ref{RN26} \hyperref[SUB-U-CU1.4:TF]{[Trayectoria alternativa F]}
		\item {[Aplicación de usuario]} Valida que la contraseña cumpla con el formato de contraseña establecido en la regla de negocios \ref{RN25} \hyperref[SUB-U-CU1.4:TG]{[Trayectoria alternativa G]}
		\item {[Aplicación de usuario]} Envía la petición al servidor mandando como parámetro el correo electrónico y la contraseña. \hyperref[SUB-U-CU1.4:TH]{[Trayectoria alternativa H]}
		\item {[Servidor embebido]} Recibe la petición de la aplicación y almacena el correo electrónico y la contraseña proporcionada por el usuario.
		\item {[Servidor embebido]} Envía como respuesta un 'Ok'. \hyperref[SUB-U-CU1.4:TI]{[Trayectoria alternativa I]}
		\item {[Aplicación de usuario]} Recibe la respuesta del servidor y almacena la IP en un archivo interno de la aplicación.
		\item {[Aplicación de usuario]} Refresca la pantalla \hyperref[fig:Configuraciones]{[IU Configuraciones]} y muestra el mensaje 'Conexión exitosa' así como actualiza el listado de servidores que se observa en la sección 'Servidores' con el servidor agregado. 
	\end{enumerate}
	Fin del caso de uso.

\paragraph{Trayectoria alternativa A} \label{SUB-U-CU1.4:TA}
	La IP es nula.
	\begin{enumerate}[label=A\arabic*.]
		\item {[Aplicación de usuario]} Muestra el mensaje 'El campo no debe estar vacío' en la pantalla \hyperref[fig:Configuraciones]{[IU Configuraciones]}
	\end{enumerate}
	Fin de la trayectoria alternativa y fin de caso de uso.
	
\paragraph{Trayectoria alternativa B} 
\label{SUB-U-CU1.4:TB}
	La IP no cumple con el formato establecido en la regla de negocios \ref{RN16}
	\begin{enumerate}[label=A\arabic*.]
		\item {[Aplicación de usuario]} Muestra el mensaje 'No cumple con el formato de IP' en la pantalla \hyperref[fig:Configuraciones]{[IU Configuraciones]}
	\end{enumerate}
	Fin de la trayectoria alternativa y fin de caso de uso.
	
\paragraph{Trayectoria alternativa C} \label{SUB-U-CU1.4:TC}
    El servidor ya tiene un correo electrónico y una contraseña configurada.
    \begin{enumerate}[label=C\arabic*.]
		\item {[Servidor embebido]} Recibe la petición de conexión del usuario y verifica si ya existe un correo electrónico y una contraseña configurada, en caso positivo devuelve como respuesta a la aplicación móvil un 'get password'.
		\item {[Aplicación de usuario]} Recibe la respuesta del servidor y muestra el cuadro de diálogo solicitando al usuario que ingrese la contraseña como se muestra en la pantalla \hyperref[fig:contrasena]{[IU Ingresar contraseña]}
		\item {[Usuario cliente]} Ingresa la contraseña para el servidor. Oprime el botón 'Aceptar'. \hyperref[SUB-U-CU1.4:TD]{[Trayectoria alternativa D]}
		\item {[Aplicación de usuario]} Valida que el campo de contraseña no sea nulo. \hyperref[SUB-U-CU1.4:TJ]{[Trayectoria alternativa J]}
		\item {[Aplicación de usuario]} Valida que la contraseña cumpla con el formato de contraseña establecido en la regla de negocios \ref{RN25} \hyperref[SUB-U-CU1.4:TK]{[Trayectoria alternativa K]}
		\item {[Aplicación de usuario]} Envía la petición al servidor mandando como parámetro la contraseña. \hyperref[SUB-U-CU1.4:TH]{[Trayectoria alternativa H]}
		\item {[Servidor embebido]} Recibe la petición de la aplicación y verifica que la contraseña proporcionada por el usuario sea correcta.
		\item {[Servidor embebido]} Envía como respuesta un 'Ok'. \hyperref[SUB-U-CU1.4:TI]{[Trayectoria alternativa I]}
		\item {[Aplicación de usuario]} Recibe la respuesta del servidor y almacena la IP en un archivo interno de la aplicación.
		\item {[Aplicación de usuario]} Refresca la pantalla \hyperref[fig:Configuraciones]{[IU Configuraciones]} y muestra el mensaje 'Conexión exitosa' así como actualiza el listado de servidores que se observa en la sección 'Servidores' con el servidor agregado. 
	\end{enumerate}
	Fin de la trayectoria alternativa y fin del caso de uso.
	
\paragraph{Trayectoria alternativa D} \label{SUB-U-CU1.4:TD}
    El usuario oprime el botón cancelar
    \begin{enumerate}[label=D\arabic*.]
		\item {[Usuario administrador o usuario cliente]} Oprime el botón cancelar.
		\item {[Aplicación de usuario]} Cierra el cuadro de diálogo.
	\end{enumerate}
	Fin de la trayectoria alternativa y fin del caso de uso.
	
\paragraph{Trayectoria alternativa E} \label{SUB-U-CU1.4:TE}
    El campo de correo electrónico o contraseña es nulo
    \begin{enumerate}[label=D\arabic*.]
		\item {[Aplicación de usuario]} Muestra el mensaje 'El campo no debe estar vacío' en el campo nulo.
	\end{enumerate}
	Fin de la trayectoria alternativa. Regresa al paso 8 de la \hyperref[SUB-U-CU1.4:TP]{[Trayectoria Principal]}
	
\paragraph{Trayectoria alternativa F} \label{SUB-U-CU1.4:TF}
    El campo de correo electrónico no cumple con el formato establecido en la regla de negocios \ref{RN26}
    \begin{enumerate}[label=D\arabic*.]
		\item {[Aplicación de usuario]} Muestra el mensaje 'No cumple con formato de correo electrónico' en el campo del correo.
	\end{enumerate}
	Fin de la trayectoria alternativa. Regresa al paso 8 de la \hyperref[SUB-U-CU1.4:TP]{[Trayectoria Principal]}
	
\paragraph{Trayectoria alternativa G} \label{SUB-U-CU1.4:TG}
    El campo de contraseña no cumple con el formato establecido en la regla de negocios \ref{RN25}
    \begin{enumerate}[label=D\arabic*.]
		\item {[Aplicación de usuario]} Muestra el mensaje 'Debe tener entre 8-10 caracteres: mayúscula, minúscula, número, y @\#\$\%\textasciicircum{}\&+=-\_.' en el campo del correo.
	\end{enumerate}
	Fin de la trayectoria alternativa. Regresa al paso 8 de la \hyperref[SUB-U-CU1.4:TP]{[Trayectoria Principal]}
	
\paragraph{Trayectoria alternativa H} \label{SUB-U-CU1.4:TH}
    No se pudo establecer conexión con el servidor
    \begin{enumerate}[label=D\arabic*.]
		\item {[Aplicación de usuario]} Muestra el mensaje 'Conexión fallida' en pantalla
	\end{enumerate}
	Fin de la trayectoria alternativa y fin del caso de uso

\paragraph{Trayectoria alternativa I} \label{SUB-U-CU1.4:TI}
    El servidor envía como respuesta 'error'
    \begin{enumerate}[label=D\arabic*.]
		\item {[Servidor embebido]} Envía como respuesta un 'error'.
		\item {[Aplicación de usuario]} Recibe la respuesta del servidor.
		\item {[Aplicación de usuario]} Muestra el mensaje 'Error al conectar' en pantalla
	\end{enumerate}
	Fin de la trayectoria alternativa y fin del caso de uso
	
\paragraph{Trayectoria alternativa J} \label{SUB-U-CU1.4:TJ}
    El campo de contraseña es nulo
    \begin{enumerate}[label=D\arabic*.]
		\item {[Aplicación de usuario]} Muestra el mensaje 'El campo no debe estar vacío' en el campo nulo.
	\end{enumerate}
	Fin de la trayectoria alternativa. Regresa al paso 3 de la \hyperref[SUB-U-CU1.4:TC]{[Trayectoria Alternativa C]}
	
\paragraph{Trayectoria alternativa K} \label{SUB-U-CU1.4:TK}
    El campo de contraseña no cumple con el formato establecido en la regla de negocios \ref{RN25}
    \begin{enumerate}[label=D\arabic*.]
		\item {[Aplicación de usuario]} Muestra el mensaje 'Debe tener entre 8-10 caracteres: mayúscula, minúscula, número, y @\#\$\%\textasciicircum{}\&+=-\_.' en el campo del correo.
	\end{enumerate}
	Fin de la trayectoria alternativa. Regresa al paso 3 de la \hyperref[SUB-U-CU1.4:TC]{[Trayectoria Alternativa C]}