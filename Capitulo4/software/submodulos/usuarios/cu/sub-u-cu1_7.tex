\subsubsection{SUB-U-CU1.4-Emparejar dispositivo}\label{SUB-U-CU1.4}
Una vez iniciada la aplicación deberá conectarse con el Servidor que atenderá todas las peticiones que esta haga, y de ser necesaria la configuración de dicho servidor, se podrán realizar cambios para poder establecer una nueva conexión.
\begin{longtable}{|J{5cm}|J{10.3cm}|}
	\hline
	\textbf{Nombre del caso de uso} &
		SUB-U-CU1.4-Emparejar dispositivo \\ \hline
	\textbf{Objetivo} &
		Establecer una conexión con el servidor que es capaz de atender todas las peticiones que la aplicación móvil pueda hacer como cliente. \\ \hline
	\textbf{Actores} &
	    \begin{itemize}
			\item Aplicación de usuario
			\item Usuario
		    \item Servidor embebido
		\end{itemize}\\ \hline 
	\textbf{Disparador} & 
		El usuario inicia por primera vez la aplicación de usuario o desea hacer alguna modificación con respecto al servidor al cual quiere conectarse. \\ \hline 
	\textbf{Entradas} & 
		Dirección IP del servidor con el cual se realizará la conexión, puede que se encuentre como opción predefinida o que deba ingresarse por el usuario de acuerdo con la Regla de negocios \ref{RN16}\\ \hline 
	\textbf{Salidas} & 
		Pantalla \hyperref[fig:Emparejamiento Dispositivos]{[IU3.1-Emparejamiento de Dispositivos]} \\ \hline
	\textbf{Precondiciones} &
		Se inicia por primera vez la aplicación de usuario o se ha dado clic en el botón Dispositivos con el símbolo de configuración de la pantalla \hyperref[fig:Barra de navegacion]{[IU5-Barra de Navegación]}.\\ \hline 
	\textbf{Postcondiciones} &
		La nueva dirección IP del servidor con el que se desea tener comunicación es modificada.\\ \hline 
	\textbf{Condiciones de término} & 
		Se logra una conexión exitosa con el nuevo servidor. \\ \hline
	\textbf{Prioridad} & 
		Alta. \\ \hline
	\textbf{Errores} & 
		\begin{itemize}
			\item \label{SUB-U-CU1.4:Error1} Error 1: No se puede establecer una conexión con el servidor.
		\end{itemize} \\ \hline
	\textbf{Reglas de negocio} & 
	    \begin{itemize}
		    \item \ref{RN16}
		\end{itemize} \\ \hline
\end{longtable}

\paragraph{Trayectoria principal} \label{SUB-U-CU1.4:TP}
	\begin{enumerate}
		\item {[Usuario]} Inicia por primera vez la aplicación móvil \hyperref[SUB-U-CU1.4:TA]{[Trayectoria alternativa A]} 
		\item {[Aplicación de usuario]} Muestra la pantalla \hyperref[fig:Emparejamiento Dispositivos]{[IU3.1-Emparejamiento de Dispositivos]}
		\item  {[Usuario]} Selecciona del combo box una dirección IP disponible, como se muestra en el ejemplo de la pantalla \hyperref[fig:Seleccion de Disposotivo]{[IU3.2-Selección de Dispositivo]}. \hyperref[SUB-U-CU1.4:TB]{[Trayectoria alternativa B]}
		\item {[Aplicación de usuario]} Valida si la conexión con el servidor es posible. \hyperref[SUB-U-CU1.4:TC]{[Trayectoria alternativa C]}
		\item {[Aplicación de usuario]} Almacena la dirección IP del nuevo servidor.
		\item {[Aplicación de usuario]} Muestra un mensaje de conexión exitosa, como se muestra en la pantalla emergente \hyperref[fig:Exito Emparejamiento]{[PE3- Éxito de Emparejamiento con Dispositivo]}.
	\end{enumerate}
	Fin del caso de uso.

\paragraph{Trayectoria alternativa A} \label{SUB-U-CU1.4:TA}
	No es la primera vez que se accede a la aplicación, por lo tanto no pasamos por la etapa de configuración.
	\begin{enumerate}[label=A\arabic*.]
		\item {[Usuario]} Da clic en el botón 'Dispositivos' con el símbolo de configuración de la pantalla \hyperref[fig:Barra de navegacion]{[IU5-Barra de Navegación]}
	\end{enumerate}
	Fin de la trayectoria alternativa. Regresa al paso 2 de la \hyperref[SUB-U-CU1.4:TP]{[Trayectoria Principal]}.
	
\paragraph{Trayectoria alternativa B} \label{SUB-U-CU1.4:TB}
	No selecciona ninguna opción del combo box y en su lugar el usuario prefiere ingresar una dirección manualmente.
	\begin{enumerate}[label=B\arabic*.]
		\item {[Usuario]}  Ingresa una dirección IP manualmente, en el campo de la pantalla \hyperref[fig:Seleccion de Disposotivo]{[IU3.2-Selección de Dispositivo]}.
		\item {[Aplicación móvil]} Valida si lo ingresado tiene un formato de dirección IP válido, según la regla de negocios \ref{RN16} \hyperref[SUB-U-CU1.4:TC]{[Trayectoria alternativa C]} \hyperref[SUB-U-CU1.4:TD]{[Trayectoria alternativa D]} 
	\end{enumerate}
	Fin de la trayectoria alternativa. Regresa al paso 4 de la \hyperref[SUB-U-CU1.4:TP]{[Trayectoria Principal]}.

\paragraph{Trayectoria alternativa C} \label{SUB-U-CU1.4:TC}
    La IP cumple con el formato estipulado por la regla de negocios \ref{RN16}, sin embargo no es posible realizar la conexión con el servidor.
    \begin{enumerate}[label=C\arabic*.]
		\item {[Aplicación móvil]} Muestra un mensaje de error en la conexión, como se muestra en la pantalla emergente \hyperref[fig:Error de Conexion]{[PE1- Error de Conexión]}.
		\item {[Aplicación de usuario]} Muestra la pantalla \hyperref[fig:Emparejamiento Dispositivos]{[IU3.1-Emparejamiento de Dispositivos]}
	\end{enumerate}
	Fin de la trayectoria alternativa y fin del caso de uso.
	
\paragraph{Trayectoria alternativa D} \label{SUB-U-CU1.4:TD}
    La IP no cumple con el formato estipulado por la Regla de negocios \ref{RN16}
    \begin{enumerate}[label=D\arabic*.]
		\item {[Aplicación móvil]} Muestra un mensaje de formato de IP inválido.
		\item {[Aplicación de usuario]} Muestra la pantalla \hyperref[fig:Emparejamiento Dispositivos]{[IU3.1-Emparejamiento de Dispositivos]}
	\end{enumerate}
	Fin de la trayectoria alternativa y fin del caso de uso.