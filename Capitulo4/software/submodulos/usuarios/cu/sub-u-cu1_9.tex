%DUDAS:
% Especificar que intervalo de notificaciones al inicio.
% Definir RN de número de notificaciones máximas a conservar?
% Agregar que extiende del CU de Ver Menú
% Entradas? ninguna
% Tprincipal cuando lo hace desde el menu y TAlternativa cuando lo hace desde la barra superior de notificaciones del celular?
%Poner las referencias a las pantallas que falten en las trauectorias
% Poner referencia de la trayectoria alternativa 

\subsubsection{SUB-U-CU1.9-Ver notificaciones}\label{SUB-U-CU1.9}
El usuario puede consultar las notificaciones que ha recibido.%especificar cuantas

\begin{longtable}{|J{5cm}|J{10.3cm}|}
	\hline
	\textbf{Nombre del caso de uso} &
		SUB-U-CU1.9-Ver notificaciones \\ \hline
	\textbf{Objetivo} &
		Permitirle al usuario consultar las notificaciones que ha recibido. \\ \hline
	\textbf{Actores} &
		Usuario. \\ \hline 
	\textbf{Disparador} & 
		%\begin{itemize}
		 El usuario quiere ver la información de las notificaciones recibidas.
		 %\end{itemize}
		 \\ \hline 
	\textbf{Entradas} & Ninguna.
		% \begin{itemize}
		% 		\item Nombre.
		% 		\item Imagen de perfil.
		% \end{itemize}
		\\ \hline 
	\textbf{Salidas} & 
	    Registros de notificaciones:
		\begin{itemize}
			\item Descripción de la notificación.
			\item Fecha y hora de la notificación.
			\item Dirección IP del dispositivo que notificó.
		\end{itemize} 
		\\ \hline
	\textbf{Precondiciones} &
		El actor debe haber recibido al menos una notificación.\\ \hline
	\textbf{Postcondiciones} & Ninguna.
		% \begin{itemize}
		% 	\item El nombre e imagen de perfil son actualizados.
		% \end{itemize} 
		\\ \hline
	\textbf{Condiciones de término} & Se muestran los registros de las notificaciones recibidas.
		% \begin{itemize}
		% 	\item Se muestra el mensaje de éxito al actor.
		% \end{itemize} 
		\\ \hline 
	\textbf{Prioridad} & 
		Baja. \\ \hline
	\textbf{Errores} & Ninguno.
		% \begin{itemize}
		% 	\item \label{SUB-M-CU1:Error1} Error 1: .
		% \end{itemize} 
		\\ \hline
	\textbf{Reglas de negocio} & Ninguna.
		% \begin{itemize}
		% 	\item \ref{RN1}.
		% \end{itemize}
		 \\ \hline
	% \caption{}
	%\label{desc:SUB-M-CU1}
\end{longtable}

\paragraph{Trayectoria principal}
	\begin{enumerate}
		\item {[Usuario]} Selecciona la opción de \textit{Notificaciones} que se muestra en la pantalla Ver menú% \hyperref[fig:sub-u-iu3]{SUB-U-IU3-Consultar panel de control}. AQUI PONER TRAYECTORIA ALTERNATIVA ?
		\item {[Sistema]} Obtiene los registros de notificaciones enviadas al dispositivo.
		\item {[Sistema]} Muestra la pantalla ver notificaciones %\hyperref[fig:sub-u-iu6]{SUB-U-IU6-Consultar medición}.
		%\item {[Actor]} Presiona el botón \textit{Aceptar}.
		%\item {[Sistema]} Muestra la pantalla \hyperref[fig:sub-u-iu3]{SUB-U-IU3-Consultar panel de control}.
	\end{enumerate}
	Fin del caso de uso.

\paragraph{Trayectoria alternativa A} \label{SUB-M-CU1.9:TA}
 	El usuario abre el centro de notificaciones de su dispositivo móvil.
 	\begin{enumerate}[label=A\arabic*.]
 	    \item {[Usuario]} Selecciona la notificación que recibió del Sistema desde el centro de notificaciones de su dispositivo móvil.
 		\item {[Sistema]} Continúa en el paso 2 de la Trayectoria Principal. %\ref{SUB-M-CU1.1:Pantalla} 
 	\end{enumerate}
 	Fin de la trayectoria alternativa.
