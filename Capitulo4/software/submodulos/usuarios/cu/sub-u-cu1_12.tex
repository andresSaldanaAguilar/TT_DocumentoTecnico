\subsubsection{SUB-U-CU1.9-Gestionar servidores almacenados}\label{SUB-U-CU1.9}
Si el usuario tiene al menos un servidor almacenado en la aplicación móvil puede realizar 3 operaciones sobre dicho servidor de acuerdo con la Regla de negocios \ref{RN20}
\begin{longtable}{|J{5cm}|J{10.3cm}|}
	\hline
	\textbf{Nombre del caso de uso} &
		SUB-U-CU1.9-Gestionar servidores almacenados \\ \hline
	\textbf{Objetivo} &
		Permitirle al usuario gestionar los servidores que tiene almacenados en la aplicación móvil. \\ \hline
	\textbf{Actores} &
	    \begin{itemize}
			\item Aplicación de usuario
			\item Usuario
			\item Servidor embebido
		\end{itemize}\\ \hline 
	\textbf{Disparador} & 
		El usuario quiere gestionar sus servidores almacenados.\\ \hline 
	\textbf{Entradas} & 
		Ninguna \\ \hline
	\textbf{Salidas} & 
	    Dependiendo el caso
		\begin{itemize}
		    \item Correo electrónico para recuperación de contraseña.
			\item Servidor con nombre modificado.
			\item Eliminación de servidor en la lista de servidores.
		\end{itemize}
		 \\ \hline
	\textbf{Precondiciones} &
		El usuario presiona el botón de Configuraciones que se muestra en la pantalla \hyperref[fig:Barra de navegacion]{[IUBarra de Navegación]}.\\ \hline 
	\textbf{Postcondiciones} &
		El usuario pudo realizar alguna de las operaciones permitidas para los servidores almacenados.\\ \hline 
	\textbf{Condiciones de término} & 
		Se concluye la operación realizada de manera exitosa. \\ \hline
	\textbf{Prioridad} & 
		Media. \\ \hline
	\textbf{Errores} & 
		No hay \\ \hline
	\textbf{Reglas de negocio} & 
	    \begin{itemize}
		    \item \ref{RN20}
		    \item \ref{RN24}
		\end{itemize} \\ \hline
\end{longtable}

\paragraph{Trayectoria principal} \label{SUB-U-CU1.9:TP}
	\begin{enumerate}
		\item {[Usuario]} Inicia la aplicación móvil, selecciona en el menú de la aplicación la opción de Configuraciones que se muestra en la pantalla \hyperref[fig:Barra de navegacion]{[IUBarra de Navegación]}
		\item {[Aplicación de usuario]} Muestra la pantalla \hyperref[fig:Configuraciones]{[IU Configuraciones]}
		\item {[Usuario]} Elige la opción 'Obtener contraseña' de uno de los servidores que se muestra en la lista 'Servidores' de la pantalla  \hyperref[SUB-U-CU1.9:TA]{[Trayectoria alternativa A]} \hyperref[SUB-U-CU1.9:TB]{[Trayectoria alternativa B]} 
		\item {[Aplicación de usuario]} Genera la petición para recuperar la contraseña del servidor solicitado y la envía.  \hyperref[SUB-U-CU1.9:TC]{[Trayectoria alternativa C]}
		\item {[Servidor embebido]} Recibe la petición del usuario y envía un correo electrónico con la contraseña configurada. 
		\item {[Servidor embebido]} Envía un 'Ok' como respuesta en caso de éxito. \hyperref[SUB-U-CU1.9:TD]{[Trayectoria alternativa D]}
		\item {[Aplicación de usuario]} Recibe la respuesta del servidor y muestra el mensaje 'Correo enviado exitosamente' en la pantalla \hyperref[fig:Configuraciones]{[IU3.1-Configuraciones]}
	\end{enumerate}
	Fin del caso de uso.

\paragraph{Trayectoria alternativa A} \label{SUB-U-CU1.9:TA}
	El usuario elige la opción de 'Editar'
	\begin{enumerate}[label=A\arabic*.]
		\item {[Aplicación de usuario]} Muestra el cuadro de diálogo \hyperref[fig:Editar servidor]{[IU3.1-Editar servidor]} solicitando al usuario el nombre que le quiera asignar al servidor seleccionado.
		\item {[Usuario]} Ingresa el nuevo nombre para el servidor y selecciona la opción 'Aceptar'. \hyperref[SUB-U-CU1.9:TE]{[Trayectoria alternativa E]} \hyperref[SUB-U-CU1.9:TF]{[Trayectoria alternativa F]}
		\item {[Aplicación de usuario]} Almacena el nuevo nombre y muestra la pantalla \hyperref[fig:Configuraciones]{[IU Configuraciones]} con el nombre actualizado del servidor.
	\end{enumerate}
	Fin de la trayectoria alternativa y fin de caso de uso.
	
\paragraph{Trayectoria alternativa B} \label{SUB-U-CU1.9:TB}
	El usuario elige la opción de 'Eliminar'
	\begin{enumerate}[label=A\arabic*.]
		\item {[Aplicación de usuario]} Muestra el cuadro de diálogo \hyperref[fig:Eliminar servidor]{[IU Eliminar servidor]} preguntándole al usuario si está seguro de que quiere eliminar el servidor seleccionado.
		\item {[Usuario]} Selecciona la opción 'Aceptar'. \hyperref[SUB-U-CU1.9:TF]{[Trayectoria alternativa F]}
		\item {[Aplicación de usuario]} Elimina el servidor almacenado y muestra la pantalla \hyperref[fig:Configuraciones]{[IU Configuraciones]}; el servidor que fue eliminado ya no aparece en la lista de 'Servidores'.
	\end{enumerate}
	Fin de la trayectoria alternativa y fin de caso de uso.
	
\paragraph{Trayectoria alternativa C} \label{SUB-U-CU1.9:TC}
    No se pudo establecer conexión con el servidor
    \begin{enumerate}[label=D\arabic*.]
		\item {[Aplicación de usuario]} Muestra el mensaje 'Error al recuperar contraseña' en pantalla
	\end{enumerate}
	Fin de la trayectoria alternativa y fin del caso de uso

\paragraph{Trayectoria alternativa D} \label{SUB-U-CU1.9:TD}
    El servidor envía como respuesta 'error'
    \begin{enumerate}[label=D\arabic*.]
		\item {[Servidor embebido]} Envía como respuesta un 'error'.
		\item {[Aplicación de usuario]} Recibe la respuesta del servidor.
		\item {[Aplicación de usuario]} Muestra el mensaje 'Error al recuperar contraseña' en pantalla
	\end{enumerate}
	Fin de la trayectoria alternativa y fin del caso de uso
	
\paragraph{Trayectoria alternativa E} \label{SUB-U-CU1.9:TE}
    El campo de nombre de servidor es nulo
    \begin{enumerate}[label=D\arabic*.]
		\item {[Aplicación de usuario]} Muestra el mensaje 'El campo no debe estar vacío' en el campo nulo.
	\end{enumerate}
	Fin de la trayectoria alternativa. Regresa al paso 1 de la \hyperref[SUB-U-CU1.9:TA]{[Trayectoria alternativa A]}
	
\paragraph{Trayectoria alternativa F} \label{SUB-U-CU1.9:TF}
    El usuario oprime el botón cancelar
    \begin{enumerate}[label=D\arabic*.]
		\item {[Usuario]} Oprime el botón cancelar.
		\item {[Aplicación de usuario]} Cierra el cuadro de diálogo.
	\end{enumerate}
	Fin de la trayectoria alternativa y fin del caso de uso.