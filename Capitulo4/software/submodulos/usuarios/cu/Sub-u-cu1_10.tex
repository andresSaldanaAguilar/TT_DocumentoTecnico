\subsubsection{SUB-U-CU1.11-Generar notificaciones}\label{SUB-U-CU1.11}
Partiendo del código de error generado por el caso de uso \hyperref[SUB-U-CU1.5]{SUB-U-CU1.5} se dará aviso, a través de una notificación del dispositivo móvil, el tipo de anomalía y la hora en la que fue detectada, siendo a su vez almacenada localmente en la aplicación de usuario para su posterior uso.  


\begin{longtable}{|J{5cm}|J{10.3cm}|}
	\hline
	\textbf{Nombre del caso de uso} &
		SUB-U-CU1.11 - Generar notificaciones
 \\ \hline
	\textbf{Objetivo} &
		Notificar al usuario sobre los 4 tipos posibles de anomalías que puede presentar el sistema:
		\begin{itemize}
			\item 1 - Sin conexión con el servidor
			\item 2 - Sin conexión con el cliente
			\item 3 - Valor de generación igual a -1
			\item 4 - Valor de generación igual a 0
		\end{itemize} \\ \hline
 \\ \hline
	\textbf{Actores} &
		Aplicación de usuario \\ \hline 
	\textbf{Disparador} & 
		Se detectó una anomalía en el funcionamiento del sistema por medio de lo especificado en el caso de uso \hyperref[SUB-U-CU1.5]{SUB-U-CU1.5}  \\ \hline 
	\textbf{Entradas} & 
		\begin{itemize}
				\item Código de error proveniente del caso de uso \hyperref[SUB-U-CU1.5]{SUB-U-CU1.5}
		\end{itemize}\\ \hline 
	\textbf{Salidas} & 
		\begin{itemize}
		    \item Tipo de anomalía
			\item Fecha y hora en que se detectó la anomalía.
			\item Notificación con la información acerca del tipo de anomalía y hora en la que fue detectada, a través del dispositivo móvil
		\end{itemize} \\ \hline
	\textbf{Precondiciones} &
		El caso de uso \hyperref[SUB-U-CU1.5]{SUB-U-CU1.5} detectó un mal funcionamiento del sistema.\\ \hline
	\textbf{Postcondiciones} &
		\begin{itemize}
			\item El tipo de anomalía junto con la fecha y hora del día en que fue detectada son almacenados.
		\end{itemize}\\ \hline
	\textbf{Condiciones de término} & 
		\begin{itemize}
			\item Se obtiene el motivo por el cuál se provocó una anomalía.
			\item Se obtiene la fecha y hora del día en que la anomalía fue detectada.
		\end{itemize} \\ \hline 
	\textbf{Prioridad} & 
		Alta. \\ \hline
	\textbf{Errores} & 
		\begin{itemize}
		\end{itemize} \\ \hline
	\textbf{Reglas de negocio} & 
		\begin{itemize}
			%\item \ref{RN4}
		\end{itemize} \\ \hline

	% \caption{}
	%\label{desc:SUB-M-CU1}
\end{longtable}

\paragraph{Trayectoria principal}
	\begin{enumerate}
		\item {[Usuario]} Monitorea en tiempo real por medio de la aplicación móvil y genera el código de error para notificación a través de lo especificado en el caso de uso \hyperref[SUB-U-CU1.5]{SUB-U-CU1.5} 
		\item {[Aplicación de usuario]} Valida qué número de código error se ha generado e identifica el tipo de anomalía, es decir, será "Sin conexión con el servidor" si el error tiene el código 1, "Sin conexión con el cliente" si tiene el código 2, "Valor de generación igual a -1" si tiene el código 3 y "Valor de generación igual a 0" si tiene el código 4.
		\item {[Aplicación de usuario]} Obtiene la fecha y hora actual.
		\item {[Aplicación de usuario]} Alerta a través de una notificación  del dispositivo móvil sobre el tipo de anomalía y la fecha y hora obtenida con aterioridad.
		\item {[Aplicación de usuario]} Almacena tanto el tipo de anomalía como la fecha y hora actual con un respectivo identificador de notificación.
	\end{enumerate}
	Fin del caso de uso.
