%%%%%%%%%%%%%%%%%%%%%%%%%%%%%%%%%%%%%%%%%%%%%%%%%%%%%%%%%%%%%%%%%%%%%%%%%
%                           Presentación                                %
%%%%%%%%%%%%%%%%%%%%%%%%%%%%%%%%%%%%%%%%%%%%%%%%%%%%%%%%%%%%%%%%%%%%%%%%%

\section{Antecedentes} % section headings are printed smaller than chapter names
El uso de fuentes de energía renovable ha cobrado cada vez más importancia hoy en día, debido a que ofrecen múltiples beneficios durante la elaboración de las actividades diarias del hombre. 
\paragraph{}
Entre las fuentes de energía renovable destaca el uso de las celdas fotovoltaicas, ofreciendo ventajas a los usuarios tales como: el aprovechamiento del Sol, el cual se trata de una fuente de energía inagotable; generación de energía limpia (no contaminante), así como ahorro económico en el servicio de electricidad. Dicha energía puede ser aprovechada en diferentes sectores, desde los domésticos hasta los industriales. Algunos de sus usos y aplicaciones son: el almacenamiento de la energía generada en baterías electrolíticas para su futura utilización, la electrificación de zonas rurales en las cuales no es posible hacer llegar la red eléctrica convencional, así como ser utilizada por dispositivos convencionales como televisiones, radios, lámparas fluorescentes, entre otros, mediante un inversor de corriente eléctrica \citep{Pre1}.   
%Buscar usos y aplicaciones de celdas fotovoltaicas (enfocado en domestico)
\paragraph{}
El grado de generación de energía eléctrica para su uso en las instalaciones deseadas depende del efecto fotovoltaico, es decir, cuando sobre los materiales semiconductores que componen a las células fotovoltaicas incide la radiación solar produciendo la electricidad, también se consideran diversos factores físicos para su producción. De hecho, se han realizado varias investigaciones con el propósito de obtener avances en las células fotovoltaicas, de los cuales destaca el Instituto de Investigación Fraunhofer en Europa, que se ha enfocado en mejorar su eficiencia y rendimiento, tomando como punto de referencia al llamado "límite de eficiencia", el objetivo consiste en obtener un potencial de eficiencia de hasta el 50 por ciento bajo luz solar concentrada en los nuevos paneles solares, para posteriormente asegurar garantías de calidad, eficiencia y eficacia del comportamiento de los paneles solares en la generación de energía eléctrica frente a factores y variables como el calor, la arena, el polvo, las tormentas, entre otros elementos que afecten al rendimiento \citep{Pre1}. Aunado a esto, el hombre se ha interesado por medir la producción de energía, así como ver el reflejo del beneficio obtenido por la implementación de un sistema de energía limpia. 
%Buscar factores que influyen en la produccion de energia solar de celdas fotovoltaicas

%\citep{Pre1}. Esto es para poner la referencia de la bibliografía
\paragraph{}
La solución planteada en este trabajo terminal consiste en la creación de un sistema que ofrece monitorear de manera remota la producción de energía de fuentes fotovoltaicas haciendo uso de nodos sensores que tienen un dispositivo de alta integración para el procesamiento de las señales, que además realizará la transmisión de las señales medidas por medio de Wi-Fi, haciendo posible agregar una gran cantidad de nodos sensor a la red para su monitoreo y almacenamiento en un servidor dentro de un sistema embebido, haciendo uso de estos datos con la aplicación móvil desarrollada para el usuario.