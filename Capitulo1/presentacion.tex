%%%%%%%%%%%%%%%%%%%%%%%%%%%%%%%%%%%%%%%%%%%%%%%%%%%%%%%%%%%%%%%%%%%%%%%%%
%                           Presentación                                %
%%%%%%%%%%%%%%%%%%%%%%%%%%%%%%%%%%%%%%%%%%%%%%%%%%%%%%%%%%%%%%%%%%%%%%%%%

\section{Presentación} % section headings are printed smaller than chapter names
La implementación de fuentes de energía renovable ha cobrado cada vez más importancia hoy en día, debido a que ofrecen múltiples beneficios durante la elaboración de las actividades diarias del hombre. 
\paragraph{}
Entre las fuentes de energía renovable destaca el uso de las celdas fotovoltaicas, ofreciendo algunas ventajas a los usuarios tales como: el aprovechamiento del Sol, el cual se trata de una fuente de energía inagotable; generación de energía limpia (no contaminante), así como ahorro económico en el servicio de electricidad. Dicha energía puede ser aprovechada en diferentes sectores, desde los domésticos hasta los industriales. %Buscar usos y aplicaciones de celdas fotovoltaicas (enfocado en domestico)
\paragraph{}
El proceso de generación de energía eléctrica para su uso en las instalaciones deseadas depende de diversos factores, desde el sistema de energía fotovoltaico, el grado de exposición a los rayos solares, hasta su transformación para la obtención de energía eléctrica. Aunado a esto, el hombre se ha interesado por medir su consumo eléctrico, así como ver el reflejo del beneficio obtenido por la implementación de un sistema de energía limpia. %Buscar factores que influyen en la produccion de energia solar de celdas fotovoltaicas

%\citep{Pre1}. Esto es para poner la referencia de la bibliografía
\paragraph{}
La propuesta de solución planteada en este trabajo terminal consiste en la creación de un sistema que ofrece monitorizar de manera remota la producción de energía de fuentes fotovoltaicas haciendo uso de nodos sensores y un dispositivo de alta integración para el procesamiento de las señales, la cual además tendrá que realizar la transmisión de las señales medidas para su monitoreo y almacenamiento por un servidor dentro de un sistema embebido, comunicándose con la plataforma desarrollada para el usuario.