%%%%%%%%%%%%%%%%%%%%%%%%%%%%%%%%%%%%%%%%%%%%%%%%%%%%%%%%%%%%%%%%%%%%%%%%%
%                           Motivación y estado del arte                %
%%%%%%%%%%%%%%%%%%%%%%%%%%%%%%%%%%%%%%%%%%%%%%%%%%%%%%%%%%%%%%%%%%%%%%%%%
\section{Motivación}
El incremento en el uso de fuentes de energía renovable en la actualidad ha cobrado gran importancia debido a que actualmente tiene un costo inferior de obtención que los combustibles fósiles gracias al progreso que se ha logrado en este campo. Siendo así la energía fotovoltaica una de las fuentes renovables de energía más accesibles y eficientes para su aplicación en proyectos de pequeña o gran magnitud \citep{Not1}.
Es importante tener en continua observación estos sistemas de generación de energía con el objetivo de atender algún fallo o anomalía en su funcionamiento lo mas pronto posible y así disminuir las pérdidas de energía; Por esta razón, el presente trabajo terminal ofrece una aplicación que permitirá a los usuarios de sistemas fotovoltaicos consultar y monitorear las variables de voltaje y corriente de su instalación para que estos tengan conocimiento del rendimiento de su instalación o si este tiene algún problema. 
