%%%%%%%%%%%%%%%%%%%%%%%%%%%%%%%%%%%%%%%%%%%%%%%%%%%%%%%%%%%%%%%%%%%%%%%%%
%                           Motivación y estado del arte                %
%%%%%%%%%%%%%%%%%%%%%%%%%%%%%%%%%%%%%%%%%%%%%%%%%%%%%%%%%%%%%%%%%%%%%%%%%
\section{Motivación}
El incremento del uso de fuentes de energía renovable en la actualidad ha cobrado gran importancia debido a que actualmente tiene un costo inferior de obtención que los combustibles fósiles gracias al progreso que se ha logrado en este campo, siendo la energía fotovoltaica una de las fuentes que es mas accesibles y eficientes para su aplicación en México \citep{Not1},como en cualquier sistema de generación de energía es importante monitorear la energía generada a partir del uso de dicha tecnología con el propósito de conocer su estado en cualquier momento.
 
El presente trabajo terminal pretende ofrecer una herramienta que le permita a aquellos usuarios que implementen un sistema de generación de energía fotovoltaica monitorear este mismo con el fin de que el usuario pueda consultar por medio de una aplicación, el voltaje y la corriente generada por el sistema de generación de energía fotovoltaica.

 \citep{Not2}
https://es.weforum.org/agenda/2017/06/asi-estan-las-energias-renovables-en-el-mundo/

https://www.forbes.com.mx/energia-solar-fotovoltaica/