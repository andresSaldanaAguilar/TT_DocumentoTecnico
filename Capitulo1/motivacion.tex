%%%%%%%%%%%%%%%%%%%%%%%%%%%%%%%%%%%%%%%%%%%%%%%%%%%%%%%%%%%%%%%%%%%%%%%%%
%                           Motivación y estado del arte                %
%%%%%%%%%%%%%%%%%%%%%%%%%%%%%%%%%%%%%%%%%%%%%%%%%%%%%%%%%%%%%%%%%%%%%%%%%
\section{Motivación}
El incremento en el uso de fuentes de energía renovables en la actualidad tiene cada vez mayor relevancia debido a que en países como México, tiene un costo inferior de obtención que los combustibles fósiles \citep{Not1}, gracias al progreso que se ha logrado en este campo, la utilización de energía solar incrementó en un trece porciento en México en el 2018 \citep{Not2}, siendo así la energía fotovoltaica una de las fuentes renovables más accesibles y eficientes para su aplicación en proyectos de pequeña o gran magnitud.
El presente trabajo terminal ofrece un sistema que permitirá a los usuarios de módulos fotovoltaicos consultar y monitorear mediante de una aplicación las variables de voltaje y corriente para que estos tengan conocimiento de la producción de energía de la instalación.
