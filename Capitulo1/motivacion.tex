%%%%%%%%%%%%%%%%%%%%%%%%%%%%%%%%%%%%%%%%%%%%%%%%%%%%%%%%%%%%%%%%%%%%%%%%%
%                           Motivación y estado del arte                %
%%%%%%%%%%%%%%%%%%%%%%%%%%%%%%%%%%%%%%%%%%%%%%%%%%%%%%%%%%%%%%%%%%%%%%%%%
\section{Motivación}
La posibilidad de pagar por litros de gasolina que estén incompletos ya es un riesgo que existe en todo el país. En los últimos años las autoridades han confirmado irregularidades en gasolinerías de las 32 entidades federativas. En promedio, cada tres días se descubre y sanciona una nueva gasolinera. \citep{Not1}
Entre 2005 y 2010 tres de cada 10 gasolineras presentaban alguna falla. En los últimos años esta relación se invirtió por completo: ahora, de cada 10 verificaciones, sólo en tres Profeco no halla irregularidades. \citep{Not2}
Teniendo en cuenta que no todas las gasolineras ofrecen litros de a litro a los consumidores se decidió realizar una aplicación móvil la cual permita a los usuarios conocer cuáles son las gasolineras que ofrecen realmente la cantidad de gasolina solicitada y a su vez cual es la cantidad de gasolina que está siendo cargada a su vehículo.