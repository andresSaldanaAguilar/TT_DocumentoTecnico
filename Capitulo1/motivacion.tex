%%%%%%%%%%%%%%%%%%%%%%%%%%%%%%%%%%%%%%%%%%%%%%%%%%%%%%%%%%%%%%%%%%%%%%%%%
%                           Motivación y estado del arte                %
%%%%%%%%%%%%%%%%%%%%%%%%%%%%%%%%%%%%%%%%%%%%%%%%%%%%%%%%%%%%%%%%%%%%%%%%%
\section{Motivación}
El incremento en el uso de fuentes de energía renovables en la actualidad ha cobrado gran importancia debido a que tiene un costo inferior de obtención que los combustibles fósiles, gracias al progreso que se ha logrado en este campo \citep{Not1}, en México, la utilización de energía solar incremento en un trece porciento en el 2018 \citep{Not2}, siendo así la energía fotovoltaica una de las fuentes renovables de energía más accesibles y eficientes para su aplicación en proyectos de pequeña o gran magnitud.
El presente trabajo terminal ofrece una aplicación que permitirá a los usuarios de sistemas fotovoltaicos consultar y monitorear las variables de voltaje y corriente de su instalación para que estos tengan conocimiento de la eficiencia de su instalación o si presenta un problema. 
