%%%%%%%%%%%%%%%%%%%%%%%%%%%%%%%%%%%%%%%%%%%%%%%%%%%%%%%%%%%%%%%%%%%%%%%%%
%                   Planteamiento del problema                          %
%%%%%%%%%%%%%%%%%%%%%%%%%%%%%%%%%%%%%%%%%%%%%%%%%%%%%%%%%%%%%%%%%%%%%%%%%

\section{Planteamiento del problema}

La necesidad que atenderá la aplicación es monitorear la potencia activa que se está generando en tiempo real para que el usuario pueda consultar dicha información con el objeto de tener en continua observación los sistemas de generación de energía para atender los fallos, accidentes o anomalías que puedan afectar su correcto funcionamiento, con lo que se podrán disminuir pérdidas cuando se trata de una de las principales fuentes de abastecimiento, así como brindar un registro histórico de la producción de energía.

