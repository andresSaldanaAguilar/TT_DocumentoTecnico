%%%%%%%%%%%%%%%%%%%%%%%%%%%%%%%%%%%%%%%%%%%%%%%%%%%%%%%%%%%%%%%%%%%%%%%%%
%                   Planteamiento del problema                          %
%%%%%%%%%%%%%%%%%%%%%%%%%%%%%%%%%%%%%%%%%%%%%%%%%%%%%%%%%%%%%%%%%%%%%%%%%

\section{Planteamiento del problema}

La necesidad que atenderá la aplicación es monitorear la potencia activa que el sistema fotovoltaico está generando en tiempo real, para que el usuario pueda consultar dicha información con el objeto de tener en continua observación al sistema fotovoltaico y así conocer cual es la generación de energía en distintos periodos de tiempo y atender de manera inmediata los posibles fallos que puedan interrumpir la generación de energía; contribuyendo como consecuencia a la disminución de pérdida de producción de energía, así como brindar un registro histórico de la producción de energía del sistema fotovoltaico en diferentes periodos que mas adelante definimos.

