%%%%%%%%%%%%%%%%%%%%%%%%%%%%%%%%%%%%%%%%%%%%%%%%%%%%%%%%%%%%%%%%%%%%%%%%%
%                   Planteamiento del problema                          %
%%%%%%%%%%%%%%%%%%%%%%%%%%%%%%%%%%%%%%%%%%%%%%%%%%%%%%%%%%%%%%%%%%%%%%%%%

\section{Planteamiento del problema}

La necesidad que atenderá la aplicación es monitorear la potencia activa que el sistema fotovoltaico está generando en tiempo real, para que el usuario pueda consultar dicha información con el objeto de tener en continua observación al sistema fotovoltaico y así atender de manera inmediata los posibles fallos, accidentes o anomalías que puedan interrumpir la generación de energía dentro del horario establecido; contribuyendo como consecuencia a la disminución de pérdida en cuanto a la producción de energía, así como brindar un registro histórico de los datos obtenidos por el monitoreo del sistema fotovoltaico.

