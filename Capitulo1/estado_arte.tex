%%%%%%%%%%%%%%%%%%%%%%%%%%%%%%%%%%%%%%%%%%%%%%%%%%%%%%%%%%%%%%%%%%%%%%%%%
%                           Estado del arte                              %
%%%%%%%%%%%%%%%%%%%%%%%%%%%%%%%%%%%%%%%%%%%%%%%%%%%%%%%%%%%%%%%%%%%%%%%%%
\section{Estado del arte}
Actualmente, existen interfaces que permiten la conexión e incorporación de diversas fuentes de generación de energía a la red eléctrica, a tales interfaces se les conoce con el nombre de microrredes \citep{Microrredes}. Este nuevo esquema de generación se caracteriza por la flexibilidad y autonomía con la que operan estas microrredes. Es decir, que, en caso de fallos de la red de distribución, estas puedan proporcionar energía directamente al usuario, siendo con esto más flexibles que los esquemas de distribución de energía ya existentes.\\

Para realizar el monitoreo de las microrredes se pueden emplear tecnologías de comunicación alámbricas e inalámbricas. En la división de tecnologías inalámbricas se encuentran las redes de sensores inalámbricas (WSN – Wireless Sensor Network).\\

Una WSN consiste en nodos de sensores autónomos implementados en zonas de interés que tienen en común características como:
\begin{itemize}
	\item Procesamiento de datos
	\item Capacidad de almacenamiento
	\item Interfaces de comunicación inalámbrica
	\item Consumo de energía limitada
\end{itemize}

Estos nodos sensores son un sistema computacional diseñado con hardware y software especialmente para realizar tareas específicas logrando así obtener beneficios en desempeño, costo y usabilidad del sistema, por lo que se le denomina un sistema embebido \citep{SistemaEmbebido}. A continuación, se hace mención de algunos de los trabajos de investigación que hacen referencia al uso de estas tecnologías:\\

\begin{itemize}
	\item En \citep{EstadoDelArte1} se propone el monitoreo autónomo de voltaje y corriente con el propósito de alertar fallas de cortocircuito, descargas o sobrealimentación de voltaje, también tiene la capacidad de controlar circuitos de protección cuando estas fallas acontezcan, esto es logrado colocando transductores (en este caso, sensores de efecto Hall) en las áreas de interés de la microrred , el acondicionamiento de la señal y procesamiento están a cargo de un microcontrolador, el envío de los datos adquiridos por medio de un arduino uno y un módulo de internet y finalmente enviando los datos a la plataforma de IoT Ubidots. 

    \item En \citep{EstadoDelArte2} se propone el monitoreo y control de una microrred simulada, donde el modelo de microrred es simulado por medio de MATLAB en una computadora que solo tendrá esa tarea, se ocupa un DAQ (Acondicionador de señales  y convertidor Analógico-Digital) para obtener los datos generados y son enviadas a un PLC (Controlador Lógico Programable) que está conectado al DAQ, este se encarga de guardar y escanear continuamente las variables de la microrred para ser monitoreadas en el sistema SCADA (Supervisión, Control y Adquisición de Datos) Honeywell Experion PKS.

    \item En \citep{EstadoDelArte3} se propone un sistema de gestión de microrredes basado en WiFi, donde el sistema de control y sensado de la microrred es manejado por un software ya existente y sensores de potencia eléctrica (En conjunto denominado como Dispositivo de control) que está conectado por medio de un cable ethernet a un módulo inalámbrico que enviará los datos a un Router que se encargará de capturar los datos de los módulos conectados a la WLAN (Red de Área Local Inalámbrica) para que el servidor los almacene en la base de datos y puedan ser accedidos posteriormente. 

    \item En \citep{EstadoDelArte4} el sistema de control y monitoreo propuesto, está conformado por un subsistema capaz de monitorear hasta 18 circuitos sobre un rango de voltajes y corrientes, y es capaz de almacenar la información de manera interna o bien cargarla a un servidor remoto. Cabe señalar que el almacenamiento de la información se realiza en intervalos de 5 minutos en un archivo CSV, el cual es enviado a través de HTTP Push a una Raspberry Pi la cual tiene la función de controlador principal. A su vez, en la Raspberry Pi se ejecuta Tornado, un programa de servidor web que implementa la API Restful, de esta forma es como se acepta la información recolectada de los sensores, así como hacer llamadas RESTful para comunicarse con el resto de la arquitectura.
\end{itemize}