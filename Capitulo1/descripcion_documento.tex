%%%%%%%%%%%%%%%%%%%%%%%%%%%%%%%%%%%%%%%%%%%%%%%%%%%%%%%%%%%%%%%%%%%%%%%%%
%                    Descripción del documento                           %
%%%%%%%%%%%%%%%%%%%%%%%%%%%%%%%%%%%%%%%%%%%%%%%%%%%%%%%%%%%%%%%%%%%%%%%%%
\section{Resumen}
El sistema que se describe y diseña a continuación tiene como objetivo el monitoreo de generación de energía de un sistema fotovoltaico mediante el uso de una aplicación móvil bajo el concepto del internet de las cosas, que se refiere a una interconexión digital de objetos cotidianos con internet.

\section{Descripción del documento}
Este documento contiene la terminología y conceptos fundamentales como base del presente trabajo terminal, para después hacer una descripción detallada del análisis y diseño. 

\\
En los capítulos siguientes se expresará a mayor profundidad el Trabajo Terminal, en el \textit{Capítulo \ref{chapter2} Marco conceptual} se presenta una referencia general, al ámbito bajo el cual es desarrollado el sistema, dando así, una introducción a los temas que son de interés para la correcta compresión de las diversas temáticas que abarca el Trabajo Terminal.

\\
En el \textit{Capítulo \ref{chapter3} Análisis del sistema} se redactan los distintos factores tomados en cuenta para las selecciones de tecnología, se explican los motivos por los cuales fueron usados ciertos dispositivos de hardware y ciertas tecnologías.

\\
El \textit{Capítulo \ref{chapter4} Diseño del sistema} se presenta el diseño de los distintos submódulos que componen al sistema, mostrando para la parte de hardware, los distintos diagramas de los circuitos que son usados, además, para la parte de software, se muestran los diagramas UML que permiten la construcción del sistema, entre los cuales se incluyen, diagramas de casos de uso y diagramas de clases. Finalmente, en este capítulo se muestra la arquitectura del sistema. %y el diagrama relacional de la base de datos.
\\

