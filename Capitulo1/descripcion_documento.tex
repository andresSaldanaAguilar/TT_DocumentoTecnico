%%%%%%%%%%%%%%%%%%%%%%%%%%%%%%%%%%%%%%%%%%%%%%%%%%%%%%%%%%%%%%%%%%%%%%%%%
%                    Descripción del documento                           %
%%%%%%%%%%%%%%%%%%%%%%%%%%%%%%%%%%%%%%%%%%%%%%%%%%%%%%%%%%%%%%%%%%%%%%%%%
\section{Resumen}
El Sistema descrito y diseñado a continuación tiene como objetivo el monitoreo de la generación de energía de un Sistema fotovoltaico mediante el desarrollo de: un módulo de monitoreo que internamente está conformado por el dispositivo de sensado, el cual se comunica a su vez con el microcontrolador; un sistema embebido que permitirá el procesamiento y almacenamiento de la información monitoreada y una aplicación móvil que permita mantener informado al usuario sobre el estado del Sistema fotovoltaico, bajo el concepto del internet de las cosas, que se refiere a una interconexión digital de objetos cotidianos con Internet.

\section{Palabras clave}
Sistema embebido, IoT, Sistema Fotovoltaico, Nodo sensor, Red de sensores.

\section{Descripción del documento}
Este documento contiene la terminología y conceptos fundamentales tomados como base del presente trabajo terminal, para después hacer una descripción detallada del análisis y diseño. 
\\
En los capítulos siguientes se expresará a mayor profundidad el Trabajo Terminal, en el \textit{Capítulo \ref{chapter2} Marco conceptual} se presenta una referencia general, al ámbito bajo el cual es desarrollado el sistema, dando así, una introducción a los temas que son de interés para la correcta compresión de las diversas temáticas que abarca el Trabajo Terminal.
\\
En el \textit{Capítulo \ref{chapter3} Análisis del sistema} se redactan los distintos factores tomados en cuenta para la selección de tecnología, se explican los motivos por los cuales fueron usados ciertos dispositivos de hardware y tecnologías.
\\
En el \textit{Capítulo \ref{chapter4} Diseño del sistema}, se presenta el Diagrama del proceso general del Sistema, el cual informa sobre el flujo entre las tareas que se llevarán acabo dentro de él; también se incluyen los Modelos de información del Sistema que se realizarán para el almacenamiento de la información en el Sistema embebido así como el almacenamiento en la Aplicación de usuario; se presenta el diagrama de Hardware respecto a la parte que comprende el módulo de monitoreo del Sistema. Finalmente se incluye el diseño de los submódulos de monitoreo y de usuario para la parte de software, en el que se muestran los diagramas UML que permiten la construcción del sistema, mediante los diagramas de casos de uso.
\\
En el \textit{Capítulo \ref{chapter5} Conclusiones}, 
se realiza un resumen de los puntos más relevantes que fueron expuestos en cada uno de los capítulos del presente documento como resultado de la investigación, el análisis y el diseño del Sistema que será desarrollado.   
\\
En el \textit{Capítulo \ref{chapter6} Trabajo futuro}, se describirán los avances que tenemos programados para la segunda etapa del Trabajo Terminal.

