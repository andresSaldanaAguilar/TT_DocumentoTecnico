%%%%%%%%%%%%%%%%%%%%%%%%%%%%%%%%%%%%%%%%%%%%%%%%%%%%%%%%%%%%%%%%%%%%%%%%%
%                    Descripción del documento                           %
%%%%%%%%%%%%%%%%%%%%%%%%%%%%%%%%%%%%%%%%%%%%%%%%%%%%%%%%%%%%%%%%%%%%%%%%%
\section{Resumen}
El sistema descrito y diseñado a continuación tiene como objetivo el monitoreo en tiempo real de la generación de energía de un sistema fotovoltaico mediante el desarrollo de: un módulo de monitoreo que internamente está conformado por el dispositivo de sensado, el cual se comunica a su vez con el microcontrolador, denominando a este conjunto como nodo sensor; un sistema embebido que permitirá la recepción, almacenamiento y envío de la información monitoreada y una aplicación móvil que permita mantener informado al usuario sobre el estado y producción de energía  del sistema fotovoltaico, bajo el concepto del internet de las cosas, que se refiere a una interconexión digital de objetos cotidianos con Internet.

\section{Palabras clave}
Sistema embebido, IoT, Sistema Fotovoltaico, Nodo sensor, Red de sensores, Tiempo real.

\section{Descripción del documento}
Este documento contiene la terminología y conceptos fundamentales tomados como base para el presente trabajo terminal, posteriormente se realiza una descripción detallada del análisis,  diseño e implementación. 
\\
En los capítulos siguientes se expresará a mayor profundidad el Trabajo Terminal, en el \textit{Capítulo \ref{chapter2} Marco conceptual} se presenta una referencia general, al ámbito bajo el cual es desarrollado el sistema, dando así, una introducción a los temas que son de interés para la correcta comprensión de las diversas temáticas que abarca el Trabajo Terminal.
\\
En el \textit{Capítulo \ref{chapter3} Análisis del sistema} se redactan los distintos factores tomados en cuenta para la selección de tecnología y se explican los motivos por los cuales serán utilizados los dispositivos de hardware indicados.
\\
En el \textit{Capítulo \ref{chapter4} Diseño del sistema}, se presenta el diagrama del proceso general del sistema, el cual informa sobre el flujo entre las tareas que se llevarán acabo; también se incluyen los modelos de información del sistema que se realizarán para el almacenamiento de la información en el sistema embebido así como el almacenamiento en la aplicación de usuario; se presenta el diagrama de hardware correspondiente al módulo de monitoreo del sistema. Finalmente se incluye el diseño de los submódulos de monitoreo y de aplicación de usuario para la parte de software, en el que se muestran los diagramas UML que permiten la construcción del sistema, mediante los diagramas de casos de uso.
\\
En el \textit{Capítulo \ref{chapter5} Desarrollo del sistema}, se explica de forma detallada y por módulos la implementación del diseño del sistema que fue descrito en el \textit{capítulo \ref{chapter4} Diseño del sistema}.
\\
En el \textit{Capítulo \ref{chapter6} Pruebas del sistema}, se da una descripción de las pruebas que fueron realizadas por cada uno de los módulos, las pruebas de integración así como de operación del sistema, mostrando sus respectivos resultados.
\\
En el \textit{Capítulo \ref{chapter7} Conclusiones}, 
se realiza un resumen de los puntos más relevantes que fueron expuestos en cada uno de los capítulos del presente documento; como resultado de la investigación, el análisis, el diseño, la implementación y los resultados de las pruebas realizados en el sistema que fue desarrollado.   
\\
En el \textit{Capítulo \ref{chapter8} Trabajo futuro}, se plantean las posibles ideas que permitirán darle continuidad a este proyecto, con el objetivo de mejorarlo y/o escalarlo agregándole aún más valor.

