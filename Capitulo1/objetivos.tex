%%%%%%%%%%%%%%%%%%%%%%%%%%%%%%%%%%%%%%%%%%%%%%%%%%%%%%%%%%%%%%%%%%%%%%%%%
%                           Objetivo                                    %
%%%%%%%%%%%%%%%%%%%%%%%%%%%%%%%%%%%%%%%%%%%%%%%%%%%%%%%%%%%%%%%%%%%%%%%%%

\section{Objetivos}

\subsection{Objetivo general}
Este trabajo tiene por objetivo desarrollar un prototipo de aplicación móvil para el sistema operativo Android en su versión 5 o posteriores, la cual permita consultar una clasificación de las gasolineras de la CDMX, elaborada mediante la comparación de la cantidad de gasolina que fue cargada a un automóvil con la cantidad de gasolina que el conductor del mismo solicitó.
\subsection{Objetivos específicos}
\begin{enumerate}[label=\arabic*.]
    \item Desarrollar una aplicación móvil que permita la recepción a través del puerto Bluetooth del celular, de los datos provenientes de un sensor que mida el flujo de gasolina que entra a un tanque de un automóvil.
	\item Desarrollar una aplicación móvil que permita la comunicación con un servidor web a través de un API basada en la arquitectura REST.
	\item Desarrollar un servidor web que permita realizar la clasificación de las gasolineras registradas en el sistema con base en los datos recibidos a través de una aplicación móvil.
	\item Desarrollar una aplicación móvil que permita el registro de usuarios en el sistema, así como la asociación de estos con sus respectivos sensores.
	\item Desarrollar una aplicación móvil que indique las gasolineras más cercanas y con las mejores clasificaciones de acuerdo con la geolocalización de los usuarios.
	\item Implementar un dispositivo electrónico el cual permita medir el flujo de gasolina que entra al tanque de un automóvil al momento de recargar gasolina.
	\item Implementar un dispositivo electrónico que transmita información a una aplicación móvil a través de un módulo Bluetooth.
\end{enumerate}