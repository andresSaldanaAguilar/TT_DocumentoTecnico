%%%%%%%%%%%%%%%%%%%%%%%%%%%%%%%%%%%%%%%%%%%%%%%%%%%%%%%%%%%%%%%%%%%%%%%%%
%                           Objetivo                                    %
%%%%%%%%%%%%%%%%%%%%%%%%%%%%%%%%%%%%%%%%%%%%%%%%%%%%%%%%%%%%%%%%%%%%%%%%%

\section{Objetivos}

\subsection{Objetivo general}
Este trabajo tiene por objetivo desarrollar un prototipo de sistema que le permita a los usuarios de sistemas fotovoltaicos realizar el monitoreo de la potencia activa producida por el sistema fotovoltaico en tiempo real, haciendo uso de un sensor para la adquisición de los datos, un microprocesador para el envío y recepción de la información que consigue el sensor, un servidor de datos alojado en un sistema embebido a partir de una tarjeta de desarrollo y una aplicación móvil para el consumo de esta información.

\subsection{Objetivos específicos}
\begin{enumerate}[label=\arabic*.]
    \item Implementar un nodo sensor que permita monitorear una fuente de energía fotovoltaica.
    \item Implementar el módulo para la transmisión de datos obtenidos por el sensor hacia el servidor, haciendo uso de un microcontrolador con módulo Wi-Fi. 
    \item Almacenar la información obtenida del monitoreo de la fuente de energía fotovoltaica en el servidor.
    \item Desarrollar una aplicación que permita al usuario visualizar los datos históricos obtenidos, resultado del monitoreo de la fuente de energía fotovoltaica.
    \item Notificar fallas por generación de energía o conectividad al usuario por medio de la aplicación.
    \item Documentar el trabajo realizado.
\end{enumerate}
